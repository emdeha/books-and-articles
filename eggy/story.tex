\documentclass[ebook,openany,12pt]{memoir}
\renewcommand{\baselinestretch}{1.15}
\renewcommand{\familydefault}{\sfdefault}
\usepackage[margin=1in]{geometry}
\usepackage[utf8]{inputenc}
\inputencoding{utf8}
\usepackage[bulgarian]{babel}
\usepackage{indentfirst}
\selectlanguage{bulgarian}
\hfuzz=2.000pt
\raggedbottom

\makeatletter
\let\ps@plain\ps@empty
\makeatother

\title{Опасни човечета}
\date{}
\author{}


\begin{document}

\clearpage\maketitle
\thispagestyle{empty}

\part{Историята започва\ldots}

\chapter{Ново начало}

\section*{1.}


Алармата звънна. Денят проблесна с хубавото усмихнато слънце, което все още имаше следи от снощния запой с луната. То се надигна само с няколко милиметра. Надигна се още, почна да му става лошо и реши да остане в това си положение. От другата страна на кръглия свят се намираше луната. Тя дори не можеше да стане и поради тая причина си постоя на едно място. Хората се почудиха, някои пророци дори се смаяха, а хипитата по парковете пушеха трева. Алармата спря да звъни. Беше почти размазана от костеливата ръка на Станчо. 

Станчо знаеше, че днешният ден ще бъде готин. Той подаваше молба за напускане от работа, която обичаше, бе платен добре и скоро дори го очакваше повишение. Въпреки това, той си тръгваше без угризения, а само с желание. Станчо щеше да създаде новия свят.

Седна и разтърка очи. Изрева нещо нечленоразделно и се протегна като котка, която очакваше пържола от стопанина си. По-скоро приличаше на опърпано куче. Отиде до кухнята, направи си чай, а докато го чакаше да изстине се изкъпа. Обръсна се, изми си зъбите, сети се да спре порното, на което бе заспал и продължи с пиенето на горещия чай.

Беше топло, за това реши да отиде на терасата. Забеляза болното изражение на слънцето или по-точно твърде яркия блясък за сезона. Върна се в спалнята, взе си слънчевите очила, и излезе отново. Сега бе по-добре. Приличаше на опърпано куче с очила, но му беше по-добре. Не мислеше за нищо, просто се наслаждаваше на изгрева, на деня. Стоя така около час и после се приготви за последния си работен ден.

Облече се в черно. Влезе в черния си Мерцедес и отпраши към генния институт ``Аспартам ЕООД''. Винаги се бе чудил защо са го кръстили така. Може би, защото основателят пиеше много кола. Или бе нещо свързано с луната. ``Да!'' – дойде му на ума. – ``Ми, тя като са го основавали е блеснала в бутилката кола! Точно!'' Почувства се умен. Засия с усмивка и направи денят си още по-щастлив.

Станчо бе генен инженер пълен с късмет и празен с мозък. Да, беше тъп, но по някакво вселенско стечение на обстоятелствата, бе прогресирал страшно много. Училище, MIT, после работа в САЩ, а после се завърнал в България, щото се влюбил в една котка, на която климатът в щатите не ѝ понасял. Бързо го наели в ``Аспартам'' и вече пета година работеше там. Котката умряла скоро след преместването. Май въздухът не ѝ се понравил.

Станчо пристигна, паркира като за двама и с горда походка на петел с очила стъпи в асансьора. Стъпвайки на етажа срещна новата лаборантка. Тя не бе много красиво момиче, противно на очакванията ви. Станчо се радваше, че напуска и вече няма да я вижда. Вярно, че се бе появила наскоро, но преди нея имаше друга. Станчо беше магнит за непривлекателни момичета. Понякога се напиваше заради това.

"--* Добро утро, г-н Стаматов – започна момичето с опит за усмивка.

"--* Бе, добро, лошо все едно – измърмори Стаматов и се опита да я отмине.

"--* Г-н Стаматов, ДНКто, което вчера оставихте да се реплики\ldots 

"--* Имам работа, после момиче\ldots – отвърна и забърза към офиса си. Влезе в последния момент и заключи.

"--* Г-н Стаматов! – чу се отчаян вик.

"--* После, г-це, после – провикна се и той и това май го спаси.

Чакаха го куп хартия, два молива и едно шише ракия с бележка на него. ``Интересно започва тази сутрин'' – помисли си той. На бележката пишеше две думи – ``С любоф''. Усмихна се предизвикателно, отиде до огледалото, съблече си сакото и ризата и почна да стяга мускули.

Скоро прекъсна това богохулно действие, облече се и се запъти към офиса на началника. В едната си ръка държеше молбата за напускане, а в другата шишето с ракия. Ходи по дългия коридор около половин минута. Стигна до една бяла врата, с черен надпис – ``Стамат Иванов – началник отдел генни модификации''. Почука три пъти на вратата, после се провикна: ``Стаматьеееей!'' Чуха се стъпки. Една космата ръка хвана дръжката на вратата и я натисна надолу. От страната на Станчо дръжката се завъртя и вратата се открехна. Двамата – шеф и подчинен – се спогледаха. Иванов видя ракията, усмихна се показвайки два реда чисти, бели зъби, които бе платил с четвърт заплата и каза:

"--* Ех, Станчо, Станчо\ldots Ех! Винаги ще оправиш сутринта на човек.

"--* Да, да\ldots – отговори Стаматов. – Виж, Стами, тря\-бва да приказваме по един въпрос.

"--* Разбира се, разбира се, влез – подкани го, даже го хвана за ръка и като детенце го заведе до ниската холна масичка и го седна на дивана. – Кажи сега. Не, чакай! Първо аз имам да ти казвам нещо! Много важно! Ама, много!

"--* Но\ldots

"--* Не, без но – отговори Стами. Станчо се чувстваше като в клиширан филм. Толкова често се случваше това по тези кадри.

"--* Не, чуй ме сега – аз напускам! – изстреля тези думи и оплю малко Иванов.

Докато Стамат го гледаше учудено и невярващо, Ста\-нчо си извади молбата и му я подаде. Постави я на масичката и зачака. Притесни се малко за бившия си шеф, който си беше добре угоено прасенце. Понякога сънуваше как си го готви за Коледа. Или на Нова година се готвеше? Обичаше го като човек и началник, но това не му пречеше да му се смее вътрешно. Такъв си беше Станчо.

"--* Но\ldots – на свой ред отговори Стамат.

"--* Чао. Взимам си едномесечна отпуска, докато изтече срокът за приемане на молбата – отговори Станчо, ходейки вече и даже отваряйки вратата. 

"--* Щях да те повиша! Да ти дам всичко. Станчооо! Неее! – провикна се Стами, а накрая и изгрухтя. После се строполи на дивана и си наля конска доза ракия.

По някакъв начин Станчо успя да излезе без да срещне лаборантката. Миналата вечер бе изнесъл всичко от офиса си, така че сега направо се качи в Мерцедеса (черният) и отпраши.

\section*{2.}

В Люлин пак беше студено. Слънцето печеше, но трудно се справяше със задачата да затопли квартала. Явно снощният запой наистина му се бе отразил. И нормално, щото и то бе в криза, което го бе принудило да купи евтина водка от някакъв западнал преди десетки векове свят. Тая водка бе правена от едно растение, което просто бе затрупало тая кръгла топка скали и пръст. Това я правеше евтина и гадна. Но какво да се прави, на всеки се случваше понякога да няма пари.

В един скъп апартамент, в една скъпа кооперация, се събуди една не толкова скъпа персона. Това бе Иван Радомирски. Той беше предприемач наскоро продал фирмата си за пет милиона, от които му бяха останали два. С тях той мислеше да направи още нещо велико, но още чакаше подходящия момент, или по-скоро подходящия човек. 

Иванчо не беше скъпа персона, щото беше прост. Прост в лошия смисъл на думата. Бе, май по-правилното беше простак. И самата му осанка предполагаше тая простота. Набит, мутренски настроен, остриган до кожа. Не, че всички така изглеждащи хора бяха негодници, но той специално напълно отговаряше на стереотипа. Беше собственик на тъмно минало, малко кокаин за лична употреба, този скъп апартамент и едно БМВ.

Събуди се на някакво легло. Неговото си легло де, ама още не можеше да осъзнае къде се намира. Полежа си блажено няколко минути, а после се надигна и отиде в кухнята. Там го чакаха няколко мазни банички с палмово масло и една боза с толкова ``Е''-та, че сигурно бе способна да убие тиранозавър. Ванката си направи сандвич от две банички и трета между тях, сипа бозата в една голяма чаша и прибави 50 грама текила. Това бе закуска за шампиони, помисли си той. Хапна, пийна и се чувстваше готов за новия ден. Поне половината от него – вече бе три след обед. 

Вечерта щеше да се проведе фест на кокошката. На него щяха да се ядат страшно много яйца и също толкова много пилешко. Иванчо щеше да ходи там. Но до тогава оставаха около пет часа. Трябваше да запълни това време по някакъв начин. 

Седна пред телевизора и почна да гледа телешки в една точка. Мислеше. Влизаха, излизаха някакви мисли, накрая мозъкът му се почувства доста използван и реши да каже на собственика си да пусне телевизора и да бози в него. Междувременно едно прасе изгрухтя някъде в далечината. Петел изкукурига и всичко стана много селско. Поне на Иван така му се стори. В същност това бяха първите наченки на неговото психическо разстройство.

По телевизора почна филм. Някакъв екшън с Арни. Онзи Арни от Терминатор, дето после стана сенатор, а след това май пенсионер. Стреляха се, биеха се. Арни бе от добрите и само размазваше лошите. Радомирски се подкефи на това. Даже по едно време толкова се вглъби, че почна да заляга, да се прави, че стреля, да вика, да пъшка. Абе, стана си професионален актьор. После се усмири, отвори си една бира, докато траеше поредната реклама и запали цигара. Изведнъж филмът му стана адски тъп. Всичко му стана адски тъпо, а имаше още цели два часа до заминаването. Зачуди се как да ги запълни. Звънна на няколко от своите хора, но всичките бяха заети. Изпсува благородно и се опъна на дивана.

Неусетно заспа.

Събуди се към осем без пет. Беше толкова точен, че направо да му се чуди човек как бе толкова тъп. Но те двете сигурно не е проблем да съществуват заедно\ldots Изми си очите, избръсна се, изпи една малка текила, облече се и излезе.

Отпред го чакаше БМВ-то му. Лъснато, черно, ново. Някой пак бе пльоснал яйце върху задния прозорец. Иван, въпреки осанката си, бе спокоен човек. Изпсува, извика силно този, който го е направил да слезе да се разберат, изчака няколко минути и след като никой не дойде отегчено хвана една кърпа (извади я от джоба си) и почна да размазва всичко. Сдоби се с безплатно затъмнено задно стъкло. Това малко го зарадва.

Запали и отпраши с пълна газ.

\chapter{Срещата}

Колите на Станчо и Иван спряха една до друга. Двамата се разминаха и влязоха, почти едновременно, в огромния и миришещ на курник ресторант, където щеше да се проведе събитието. 

Естествено, всяко такова светско събитие не минаваше без жива музика. Хващаш един оркестър, такъв от улицата с по-мургави участници, някаква скъпо платена, надарена проститутка и един качествен запис на плейбек и всичко е готово. Хората се забавляват, бизнесът върви. ``Така се изкарваха пари -- редовно казваше Радомир Маслинчев, собственикът на заведението, -- а не с някви там стартъпи, мартъпи и прочие нескопосани предприемаченца пробващи да ``оправят'' света.''

Музиката дънеше, храната направо изчезваше от чиниите, пък за алкохола да не говорим. 

За пореден път Станчо се отправи към шведската маса за да си вземе порция пилешко с пържени яйца. Същото направи и Иван. Те се бяха толкова забъркали от алкохол и ядене, че не се видяха и се сблъскаха точно пред масата. Температурата в заведението, която и без това си беше висока, се качи с още поне пет градуса. Ванката почервеня, Станчо малко побеля. Получи се една реакция, която бе бързо потушена от гръмкия смях на първия. 

"--* Ех, ех, ех! – почна Иван. – Пак не гледаме къде вървим. Ама, айде, то нали е купон. Яде се, пие се. Случва се.

"--* Да, да, да! – отвърна Станчо. – Какво ще си взимаме?

"--* Ами, тука съм си го харесал това панирано пиленце. Тия, хапките, де. Направо много апетитно изглеждат.

"--* Да, бе. Готвачите са си свършили работата качествено. Не е като по другите събития, дето всички стоят гладни и преживят на алкохол.

"--* Ха, ха, ха, ха! – изсмя се Иванчо от сърце. Дълбок, тежък смях. – Имаш го чувството! Имаш гоо!

Иван му подаде ръка зарадван и те се здрависаха. На Станчо почти му изскочиха очите от силата, с която бе смазана ръката му. Благодарение на количеството алкохол, което бе изпил, той се справи с положението и мъжката отвърна на здрависването.

"--* Аз съм Станчо – поде Стаматов.

"--* Аз пък съм Иван. Как е?

"--* Мии добре. Тук малко разтоварвам. Оглеждам се за познати лица. И така\ldots Тии с кво се занимаваш?

"--* Аз ли? Аз се занимавах с бизнес. Сега работя върху себе си. Ха, ха, ха – отвърна Ванката.

"--* Бизнес? Звучи готино! Кво правеше?

"--* Наркотици – отговори Радомирски толкова сериозно, че Станчо леко се ококори и кимна назад с глава. – Не, бе, споко. Хахаха. Имах тука една компания, продадох я. Сега търся нови възможности. 

"--* Ааааа. Готино. То и аз нещо подобно правя\ldots Ще почвам да правя, де. Само чакам малко да мине време, да почина.

"--* Та, от кво ше почиваш? Той живота си минава, ти ше почиваш – отвърна Иванчо.

"--* Да, да\ldots -- каза Стаматов. – А какви възможности точно търсиш?

"--* Ами, нещо интересно и бързо печелившо. Каквото всички търсят. Хаха. 

"--* Аха. Добре, готино\ldots С компания ли си тук? – след малко замисляне запита бившият генен инженер. 

"--* Ми, не. Те всички са по някви задачи. Работници, заети хора. Хаха. Не са като мен да почиват\ldots Ела, да седнем там на оная маса.

Те тръгнаха към една интимна маса за двама. Имаше свещ по средата, а до нея малка ваза с три червени рози. Столовете бяха леко разкривени от всичките дебели задници, които са сядали върху тях. Да, скъп беше ресторантът, но не и хората, които ходеха там. Поради тая причина Маслинчев не се грижеше много за обзавеждането. Най-важното бе храната – мазна, сочна, вкусна, в огромни порции. А масите подражаваха и на яденето, и на консуматорите му – огромни, омазнени, масивни, пропити с разлят алкохол. Една покривка стоеше под вазата и свещта.

Радомирски сложи своята порция пред себе си и почна да яде направо с ръце. От време на време изтриваше мазните си пръсти в кърпа, пропита с петна и тъга. Стаматов беше по-културен – той ядеше с вилица и нож. Направо истинска аристократичност струеше от неговата консумираща осанка. От време на време глътваха по малко ракия и я замезваха със случайно появилата се отнякъде пастърма. Интересно се появяваха нещата в този ресторант. В един момент ядеш, пиеш – само твоята порция -- и изведнъж – пържола! Доволно, наистина. Естествено, после тези случайно появили се меса, салати и прочие се прибавяха към сметката. Естествено, никой не забелязваше или акo, би се засрамил твърде много от това да се стисне за една луканка. ``Бизнес, момче, бизнес!'' -- често повтаряше Радомир Маслинчев на сина си – едно надуто хлапе, което единствено почиташе баща си.

Привършиха с поредната порция, поизтриха се, наляха си още ракия и продължиха спрелия разговор.

"--* Та -- вече леко завалено почна Стаматов, – идеята ми е – преди да изпие това количество алкохол, той не смяташе да споделя нищо – да променя света! Чакай, чакай! Не с тоя смях. Виж сега, аз съм генен инженер. Наскоро на мойта работа разработихме един продукт. Успяхме да създадем цветни пилета. Избираме си сами цвета на перата им! Но това не е нищо! Щом успяхме да променим перата, ще можем да променим и друго, по-сериозно – мозъкът им! Не е ли невероятно?

"--* И кво от тва? Браво? – отвърна Радомирски отегчено. – Някви учени тука променили някви пера и ся мозъци ще правят\ldots И кви пари от тва? Нищо. Тя тая наука освен да харчи, нищо друго не може да прави\ldots 

"--* Не, чакай. Чуй сега. От това може да се изкара много. Представи си – тук Иванчо включи на режим представяне и почна да гледа като теленце – умни кокошки. Сами си гледат храната, сами си ядат, сами се грижат за себе си. И накрая, ако наистина успеем, се самозаколват. Това ще спести адски много пари. Ние подбиваме цените на другите и ставаме адски богати. Но чакай, не бързай. Ми другите животни? Смятай какво можем да направим! Целият свят може да е наш! А, какво мислиш?

Мутрата на Ванката беше сериозно изкривена. Мисловната му дейност протичаше дори по космите в носа му. Неговият организъм осъзнаваше, че става нещо интересно, че е чул нещо, от което могат да станат пари. Оставаше да се справи с тежката задача да осъзнае всичко. Лек ветрец подухваше от прозореца до масата. Около тях почти настана тишина. Човек можеше да чуе скърцането на хлебарките в кухнята на ресторанта. Поне така си мислеше Иван. Много филми гледаше напоследък и това често го вкарваше в някакви фантазии – още един симптом на започващото му психично заболяване.

"--* Добре – почна и спря. Нещо не му се струваше наред.

След секунда-две мълчание се сети какво беше объркано.

"--* А имаш ли някакъв по-точен план на всичките тия истории?

"--* Хах\ldots – зарадван и леко опиянен отвърна Станчо. – Ми, не съм се замислял за по-точен план. Всичко е просто идея. Пък и ако вземем финансиране всичко ще си дойде на мястото. Важното е да имаме главната концепция и първата стъпка на лице. А ние ги имаме – умни кокошки и наемане на учени.

"--* Хм, та за кво ни е финансиране?! Аз имам пари, ти имаш със сигурност. То тва да не са нужни десетки милиони за него?!

"--* Ами, Ванка, как да ти кажа\ldots Нужни са. Само апаратурата е два милиона. Отделно заплати, офиси, наеми\ldots Доста работи са.

"--* Ехеее! А как ше го вземем тва финансиране? Аз тия пари да не ги бера по дърветата?!

"--* Спокойно! Ще измислим нещо – каза Станчо.

Те продължиха да приказват на различни теми. Пиха и ядоха. Накрая се изпратиха взаимно. Залитаха и падаха, радваха се. Надяваха се да направят нещо. Естествено, не направиха. Поне първите няколко месеца след срещата двамата си продължиха стария начин на живот. Станчо харчеше остатъка от заплатата си главоломно, а Иван проспиваше живота си и се занимаваше с мутрата си. Живееха си.

\chapter{Раждането}

\section*{1.}

На една далечна планета, обикаляща около едно далечно слънце заваля дъжд. Капки киселина падаха върху разядената повърхност на кръглото, скалисто тяло. Отдолу, под пластове скала, кости и алкохолни пари, се намираха няколко жилища. В тези жилища, на тая планета, живееха същества много подобни на хората. Те водеха прост живот, далеч от слънчевата светлина. Когато не валеше ходеха навън за да се радват на изгарящата топлота предизвикана от тяхната звезда. Обикновено си изнасяха с тях разни синтетични вещества за да се разсънят. И така седяха под жаркия пек и си говореха. Спомняха си за тяхното минало изпълнено с щастие и саблезъби тигри, които те не наричаха така. 

Та, беше се случила война и бяха умрели повечето. Тези, които оцеляха се затвориха надълбоко. Тези, които умряха се въплътиха в птици с леки мозъчни изкривявания и литнаха към небето само за да бъдат изгорени от киселината, която преди войната утоляваше жаждата им. Кръговратът на живота бе такъв и те го знаеха. Нямаше честност, нямаше награди – само това, което сам си предоставиш ти бе давано.

Дъждът капеше, а съществата тъкмо се събуждаха. Те спяха на кръгли огромни легла с меки завивки и топли дюшеци. Бяха в една голяма обща стая, която побираше цялото население на тази планета. Тя също бе кръгла. Лек, пулсиращ обръч минаваше по стената на стаята. Той успокояваше и изпълваше с апатия, но също така и контролираше. Казваше на човекоподобните какво да правят, занимаваше се с чувствата им, играеше си с живота им.

Беше време за ставане. Човекоподобните се разсъниха, разтъркаха очи и машинално се надигнаха от леглата си. Днешният ден започваше някак си странно.

Всички се отправиха към също толкова огромната столова. Тя бе друга кръгла стая, с друг лек, пулсиращ обръч. Там имаше кръгли маси с по четири стола. Насядаха. Днес бяха с двама по-малко. Единият се бе напил с цветен сок и бе почнал да повръща дъги, от което бе сменил няколко краски, бе се смекчил и бе паднал разтопен и мъртъв. Другият се бе опитал да счупи обръча, което само му докара счупена ръка. Сега лежеше в лазарета. 

Беше им сервирана обикновената закуска от райски гъби и афгански хляб. За всеки много точно бе определена порцията, която му се полага. Закусиха. Сетиха се защо денят започва странно – днес имаше раждане. Това бе нещо страшно рядко и страшно обичано от тях. Щяха да се попълнят - нова работна сила щеше да замести старата.

Отправиха се към залата за ритуали. Тя бе кубична, със страшно висок таван и страшно сиви стени. Там обръч нямаше, защото ритуалите сами поддържаха мира. Пък и този куб бе свещен, а никое същество не смееше да води битки на свещено място. Наредиха се до стените. Всеки бе на точно определеното си място. 

Всички почнаха да пеят. Красива мелодия пулсираща и гънеща се между техните усти. Тя бе преплетена с барабанни удари, правени с крака. Бе прекъсната от зверския вик на тази, която раждаше.

И сега се забеляза, че по средата на този огромен куб имаше паралелепипед. По средата на паралелепипеда бе легнала тази, която създаваше новото индивидче. Зверски облещени очи, с жълти, зелени и сини нишки, проблясваха в сивотата. Изкривеното от последния вик лице, леко петносано с цветове, бе сменено от спокойствие. Това спокойствие пак се замени от кривота. И така имаше един пулсиращ цикъл на промяна. Това бе нормално, това бе красиво. Тези човекоподобни бяха красиви до органелите на клетките си. Тялото на раждащата бе облечено с бяла пелерина, жертвено и самотно. Гърч след гърч преминаваше по него и сформираше вълни чиито пулсации ритмично тресяха стаята – такава бе силата на тоя ритуал. 

Постепенно тя се измени. Пелерината почна да избледнява, кожата ѝ изтъняваше, косата изчезваше. Красиви червено-жълти нюанси проблясваха под плътта ѝ. Жената се смаляваше, вече бе девойка. Лицето придобиваше все по-детски черти. Крачетата се свиваха в зародишна поза. Ръчичките станаха крехки. Очите се затвориха, а косата беше толкова рядка, че се виждаше скалпа, а под него – недорасналите гънки на мозъка на бебето. Всичко спря. Рев прокънтя из залата. Новото се бе родило.

\section*{2.}

Ритуалът приключи толкова бързо, колкото бе започнал. Съществата отидоха по задълженията си. Новото бе оставено на грижите на паралелепипеда.

То се търкаляше по спиралата между живота и смър\-тта. Скачаше от брънка на брънка, изминаваше път, непосилен дори за най-опитните. Спъна се в един камък. От камъка изскочи една пеперуда, която от своя страна се блъсна в челото на Новото. То се усмихна, разпери ръка и замахна. Халоса се по малкото челце и заплака. Един еднорог подскочи към него, легна и го сгуши в себе си. Отнякъде дойде статуя, която с непроницаема сила заби лакътя си в рога на конеподобното. Той се строши и съществото се изпари. Бронзовата фигура се устреми към Новото. То я видя, подскочи и запълзя към появилото се дърво. Дървото разпери клони в усмивка и погълна детето. Там беше топло, то заспа.

Събуди го звукът на течаща вода. Вече бе порасло. Излезе от пастта на растението само за да види пустош. Дървото бе изсъхнало и свито. Векове на пустиня заобикаляха всичко. Поточе течеше отстрани с кърваво-червени води. От него се разперваха вадички, прилични на капиляри. А човек, ако погледнеше отгоре, можеше да забележи, че самото поточе извираше от река. Тази река струеше от една планина, която туптеше. Това беше величие несравнимо по своята сложност. С всяко туп се захранваха водите на тази планета. С всяко захранване съществата процъфтяваха. Всяко съкращение траеше години. Този път бяха спрели. Паралелепипедът бе подготвил изпитание за Новото.

То направи първата си крачка. Стави, почивали с векове, се свиха. Кости, не напрягани от стотици лета, осигуриха опора. Мускули се съкратиха и кръв забуча. Първото действие бе изпълнено. Оставаха още стотици. Леко полеко Новото се придвижи, видя природата и заплака. След плача дойде смях. След смеха – безразличие. След безразличието се изредиха чувства на загуба, романтика, носталгия, тъга, радост, здраве, спомен, хубост, любов, грижа, страдание, болка, смърт, жестокост, омраза, безпаричие, богатство, апатия. После се задвижи мозъкът. Той почна да мисли. Пресметна обиколката на тази жива планета, изчисли скоростта на светлината във вакуум, определи кръг чрез линия и транспортир, разбра защо дърветата живеят и защо слушаме музика, нарисува същество, а после го задраска и сътвори света. След мисълта дойде съвкупността. Забърза времето и го сложи в кораб, а кораба пусна да върви бързо, а бързината бе светлинна. Погледна времето и разбра, че му изглежда еднакво и така си показа следствие от една теорема. Спря и се усмихна. Бе готов, бе създал, бе творец.

Продължи да върви, а пред него имаше стълба издигаща се към просторите на светлите небеса, но идваща от дълбините на червеното ядро. Изкачи я, но падна. Хвана се по средата за изгорения парапет. Бе страшно близо до смъртта. Изпита уплаха, която му се струваше неестествена след онова величие. Спря, видя пейка и се метна върху нея. Тя гореше, но той медитираше. Проясни се, почна да се издига и стигна до Горното. Новото бе до висините на величието. Спря, събуди се.

\chapter{Финансирането}

\section*{1.}

Кафенето беше опушено, затънтено и тъмно. Една муха, \`{о}пита от алкохолните пари, се ориентираше към ръката на Станчо. Тя кацна, потърка предните си крачета и почна да смуче напосоки с хобота си. Стаматов обърна глава, погледна я и замахна. Мухата вече представляваше пихтия чернота и кръв. Той сви палец и среден пръст на буквата ``О'' и изстреля останките на летящата твар на пода.

Кафенето всъщност бе кръчма. Сега бе вечер. Станчо и Иван бяха седнали на една маса край един мазен прозорец. Върху масата имаше две порции пържоли. До чинията на Радомирски имаше текила, а до тази на Станчо – запотена чаша бира. Двамата ядяха и отпиваха.

"--* Вчера някакъв се блъснал с БМВ по Цариградско. Литнал на 20 метра от колата. Не приличал на нищо – започна разговорът Станчо.

"--* Еба\ldots – замислено отговори Ванката. – Такъв е живота, бате, кво да праиш. Днес си караш БМВ-то, утре те карат на небето. Аз за тва не превишавам много скоростта. Ама като я натиснеш тая кола\ldots Малеее\ldots

"--* Дам\ldots Да знаеш вчера кви пържоли си направих! Направо да си оближеш пръстите!

"--* Че ти и готвиш ли?! Хахаха\ldots Всеки ден ме учудваш – с насмешка отговори Иван.

"--* Ами, да\ldots Харесва ми, пък и е по-икономично\ldots

"--* Аааа начи кризата те хвана и тебе. Ти тука няколко години на яката заплата и ся като свършат спестяванията ше го ядеш!

"--* Бе, надявам се да ни потръгне с бизнеса. Ама, то като не действаме няма как да потръгне\ldots Дай да почваме да движим нещата вече. Не може така\ldots

"--* Ами, да почваме, де! Аз откога ти казвам, ти не та не\ldots Ама, не може така. Ти, все едно аз съм виновен, че не сме почнали\ldots Баси! – отвърна Ванката.

"--* Спокойно сега. Дай да помислим откъде може да вземем пари.

"--* Ми как откъде бе? Банката! – каза Ванката.

"--* Банката?! Ами ако фалираме? Ако не успеем да ги върнем? Ванка, това са много пари. Не е да кажеш няколко хиляди\ldots Тук говорим за милиони\ldots

"--* Е, като не ти харесва, предложи ти нещо де! Айде да те видим кат си толкова отворен\ldots

"--* Ами, четох това онова, говорих с приятели и научих за няколко начина. Първият е инвеститорски фонд. Разказваме си идеята на едни хора и ако ни харесат, ще ни дадат пари.

"--* Добре, супер. Ай, да започваме – нетърпеливо го прекъсна Иван.

"--* Чакай, чакай. Освен това може да направим една кампания в интернет. По същия начин описваме това, което ще създадем, обаче го записваме на видео. И там на един сайт слагаме да ни гледат и те ще дават пари ако им хареса идеята – обясни Станчо.

"--* Ми айде, почваме.

"--* Чакай ся, много бързаш. Според мен чрез инвеститор е по-добре, щото можем да вземем повече пари. Обаче, той пък ще вземе дял от фирмата\ldots Трябва да го измислим. Той дела е малък, ама после ще трябва да го откупуваме. Но айде, ще мислим за това като му дойде времето.

"--* А, като му дойде времето\ldots То като му дойде времето, вече ще е късно. Тва са пари, Станчо. Не може ей така да си играем с тях. Малък дял, ама от няколко милиарда\ldots Виж ся, ако може да се договорим нещо с тях\ldots Да кажем, като успее бизнесът и те да си дадат дела безвъзмездно\ldots А, кво ше каеш?

"--* Ванка, те не са овци. Кой по дяволите, на тоя свят, ей така ще си даде дела от успешен бизнес?! Помисли малко\ldots Десет процента да са, дори двайсет, ако сме бързи ще ги откупим на безценица. Вярно, ще сме навътре с няколко милиона, но после? Тва е бизнес за трилиони! С това може да владеем света! Помисли малко, просто помисли. Тея пари, с които сме се увътрили, после ще са наши, щото инвеститорите ще са наши. Всичко ще е наше!

По лицето на Станчо блестеше лека лудост. Приятна усмивка почна да се заформя по крайчетата на устата му. Ванката се опитваше да асимилира как, аджеба, ще стане всичко това и беше превключил на телешкия поглед. Тишината беше като на филм преди съдбоносна сцена. Човек можеше да почувства как не само те двамата мислят напрегнато, а как целият свят се опитва да им помогне да решат проблема. 

"--* Добре, добре\ldots Да намерим тоя инвеститор – каза Иван.

\section*{2.}

Беше един от онези снежни дни преди празниците. Леки, бели парцали се сипеха от посивялото небе и покриваха земята. Маса хора тъпчеха по магазините, притеснени от това с какво ще зарадват другите, какво ще си помислят хората за подаръците им, как да осигурят по-хубаво ядене за празниците и други такива прости, човешки мисли. На Витошка, едно момиченце с опърпани дрехи и скалъпен шал беше извадило посиняла ръчичка в жест, който умоляваше за някоя стотинка. Хората подминаваха, някои незабелязващи, други смятащи, че всеки си има своите проблеми, трети състрадаващи без действие, четвърти мислещи, че това е някаква схема за крадене на пари\ldots Един млад човек погледна към нея, спря за малко и посегна към портмонето си. После сякаш му щукна по-добра идея и отмина. След няколко минути се върна с плик, в който имаше няколко сандвича, бутилка минерална вода и два големи шоколада. Те бяха прегърнати от едно топло, плюшено мече. 

Младежът срамежливо пристъпи към момиченцето и ѝ подаде плика. 

"--* Заповядай – едвам промълви той.

Момиченцето взе плика, разтвори го и ококори очи. Погледна пак нагоре, но човекът бе отпрашил. То взе единия от сандвичите и почна да го яде с ентусиазъм. Свърши с него и пак протегна ръка в искащ жест.

Това бе драмата на днешния живот. Детето не знаеше какво е благодарност, просто защото никой до сега не е бил добър към него. Щом не знаеш какво е доброта, няма как да разбереш, че трябва да си благодарен за нея. Този жест, който ние бихме сметнали за възвишен и рядък, бе за нея просто заплата за работата ѝ на улицата. Нито я стопли, нито оправи живота ѝ, а просто продължи жалкото ѝ съществуване. 

Покрай тази случка минаха Маслинчо Людмилов и Люси Панайотова. Те бяха основатели на един инвестиционен фонд, който за няколко години бе попълнил портфолиото си с доста апетитни фирми. Апетитни, не щото всичките изкарваха милиони, а щото повечето бяха свързани с храна. Виждате ли, двамата инвеститори бяха леко закръглени и бяха леко увлечени по това да си хапват вкусни неща. Нормално бе, понеже търсеха щастието в живота, да се занимават с ядене и на работа.

Люси беше облечена с черно, плътно и дълго палто. Около врата си носеше бял шал, дискретно прикриващ втората ѝ гушка. Имаше също и черно, отскоро пак навлизащо на мода, клоше на главата. Ботушите ѝ бяха сиви, невръзващи се с останалите цветове на дрехите ѝ. Все пак, колкото и богат да е човек, ако няма чувство за мода, ще изглежда зле. 

Маслинчо бе със сив костюм и тъмно-бежав шлифер. Кръгли очила без рамки стояха на носа му. Косата му бе побеляла от снега, просто защото не знаеше каква шапка да си сложи, така че да му отива на останалото облекло. Бе достатъчно умен обаче, за да си обуе ботуши вместо традиционните официални обувки. 

Те се бяха запътили в посока НДК, където щяха да отседнат в едно приятно и спокойно заведение за да си поговорят за текущата ситуация във фирмата. От няколко месеца попадаха само на неуспешни стартиращи компании. Бяха изгубили доста от парите си и трябваше да действат по въпроса.

Стигнаха до мястото. Пред вратата ги чакаше едно опърпано куче, което веднага се втурна да душка краката на Люси. Тя се ужаси и почна да му се кара. Не било учтиво така, как може това куче тука да посреща и да душка? Ми ако носи болести? Ужас! Закле се, че повече няма да стъпят в това кафе. Маслинчо се попритесни и тръгна да я успокоява, да се извинява и да моли персонала по-бързо да им намери маса и да донесе обичайното. Да, бяха като женени, само дето не водеха съвместен живот и нямаше никакви други чувства освен тези свързани с бизнеса.

Настаниха ги на една маса, със запалена свещ до един светъл прозорец. Отвън можеха да гледат минувачите и да се любуват на сипещия се сняг. Това успокояваше и прокарваше топли чувства, свързани с пържоли, баници, сладкиши и образа на мама радваща се на нейното семейство.

Сервитьорката дойде и им донесе кафе и тирамису. Двамата почнаха да похапват от сладкото и да отпиват от събуждащата течност.

"--* Тортичките са хубави тази сутрин – забеляза Маслинчо.

"--* Да, наистина. Постарали са се. Де да се стараеха така всяка сутрин\ldots 

"--* Е, поне е спокойно тук. Можеш да обсъждаш, да мислиш. Не е като някое скъпо и претрупано кафене\ldots

"--* О, да, определено. Както и да е\ldots Дай да видим какво ще правим с компанията – започна Люси. – Нещата са зле – не сме имали успешни стартъпи от три месеца\ldots

"--* Мислиш, че не знам? И какво да направим? Твърде много храна се навъди в тоя свят. То не бяха фаст фууд, ийт уел, хам-бург и прочие\ldots Не мислиш ли, че трябва да сменим стратегията? Нещо технологично може би?

"--* Технологията е голям балон, готов да се пукне всеки момент. Там е още по-зле\ldots Но виж, не сме инвестирали в наука\ldots

"--* Там са много пари\ldots -- отвърна Маслинчо. – Не бихме си избили инвестициите и рискът е твърде голям.

"--* Нали за това сме инвеститори? Ако не рискувахме, нямаше да успеем. Пък и парите се избиват ако намерим нещо с реално приложение. 

"--* Е, да, ама\ldots

"--* Какво ама? Станал си много притеснителен и нерешителен напоследък\ldots

"--* Е, как да не стана, бе, Люси? Как да не стана? Нещата са зле, скоро можем да фалираме, ако\ldots

"--* Я недей с тия мисли! От теб най-малко съм очаквала да се отчаеш\ldots Всичко ще се нареди. Спомняш ли си – почнахме със страшно малко и инвестирахме в много рискован бизнес. Тогава не се притесняваше\ldots

"--* Тогава бях млад! – отвърна Людмилов.

"--* И какво? Хората на по четирийсет години стават милиардери\ldots Я стига, станал си дете! 

"--* Дете или не, сега имам задължения. Тогава не ми пукаше и просто исках да експериментирам. Люси, искам да уседна вече, да си уредя пенсията\ldots

"--* Няма да го направиш ако фалираме. А това е доста вероятно ако не сменим стратегията си.

"--* Ми дай да видим как успяват другите. Всичко живо инвестира в софтуерни компании. Повечето успяват\ldots

"--* Повечето? Една на пет успяват. Ние нямаме толкова пари за този нисък процент на успех\ldots Като ще фалираме, дай да влезем с рогата напред и да пробваме с нещо нечувано. И двамата сме заделили по нещо за черни дни\ldots Ще изплуваме някакси после.

"--* Ами, ако не изплуваме\ldots Люси, не знам, просто не знам. Писна ми от това\ldots

"--* С оплакване няма да стигнеш до никъде – раздразнено му отвърна тя.

"--* Добре, какво предлагаш? Айде кажи сега\ldots

"--* Има тук едни момчета, наскоро ги забелязах. Разполагат с пари, идея и някакъв процент ум – тя преглътна тортичка и продължи. – Можем да отидем да поговорим с тях, да ги видим какво предлагат. Ще разпитаме, ще помислим хубаво и ако ни харесат, ще им направим оферта, която не могат да откажат.

"--* Те нашите оферти са едни\ldots

"--* Престани! – Люси вече се ядосваше. – Чуй ме сега – ще сме внимателни и всичко ще се нареди. Щом съм ги забелязала, значи имам нещо предвид. 

"--* Добре, добре\ldots Айде, ще ги видим\ldots

Останалата част от престоя им премина в мълчание и тук-там случайни думи. Те станаха, оставиха нормалния бакшиш и потеглиха по работа. Имаше да се върши доста днес. Ако искаха да успеят трябваше да вършат доста. Такава беше формулата – постоянство, щастие и работа.

\section*{3.}

Отново бе сутрин. Станчо спеше и сънуваше как леприконите завземат трезвото слънце. От време на време потръпваше с крак и измърморваше нещо. В съня беше генерал на защитната армия на деня. Даваше заповеди и крещеше по войниците да ги изпълняват. Беше облечен в униформа с цветовете на дъгата, на главата си имаше конусовидна шапка. Един леприконски шрапнел литна към главата му. Оглушителен алармен звън полетя в мозъка му. Той се размърда, почна да рипа опитвайки да се спаси и след малко осъзна, че този шрапнел беше звънящият му телефон. Изрева, измърмори и протегна ръка към устройството. Едвам го напипа. Придърпа го към ръба на нощното шкафче и за малко не го изпусна на земята. Накрая телефонът се озова в кокалестата му хватка. Плъзна с палец слушалката за отговор и допря говорителя до ухото си. 

"--* Армпф – измърмори.

"--* Добро утро. Г-н Стаматов?

"--* Арм\ldots Да. Да, аз съм. Май\ldots Добро утро, де – постепенно почваше да осъзнава ситуацията около него.

"--* Ох, да не би да ви събудих? Извинявайте много – Станчо се опита да проговори нещо, но гласът не го пусна. – Аз ви звъня във връзка с една бизнес възможност\ldots

"--* Бизнес възможност? За какво ми е бизнес възможност? Та, аз съм генерал на деня! – още не бе осъзнал изцяло ситуацията.

"--* Ъм, добре\ldots Харесва ми, че сънувате! Разсънете се малко, измийте си очите и след няколко минутки ще ви звънна пак – ``клъп'' се чу от отсрещния край.

Станчо стана по някакъв начин, огледа се и пусна една пословична сутрешна пръдня. Постоя малко седнал, втренчен в земята и невярващ на обкръжилата го стая.

"--* Потъмъръмпф! – промърмори той. – Паръмпаръмпф! Оооо, що пих толкова снощи\ldots 

Хвана се за главата, постоя още малко така и отиде в банята да си измие очите. Изми си зъбите за да премахне смрадта на алкохол от устата си и постоя малко опрян на мивката и втренчен в огледалото. Телефонът звънна отново. Той се запъти към него.

"--* По-добре ли сте господине? – обади се миналия глас. – Добре, супер. Та, аз съм Люси Панайотова, инвеститор в хранителната индустрия – Стаматов се опита да каже нещо, но тя го прекъсна. – Чакайте малко, сега ще ме чуете и после ще говорите. 

"--* Но\ldots – накрая успя да каже той.

"--* Така, забелязахме ви, че искате да създавате нещо с вашия приятел. Даже на едно събитие се спогледахме доста сериозно, но както и да е\ldots Та, вие имате идеята, ние имаме парите. Искате ли да се видим в нашия офис и да поговорим за нашето бъдещо съдружие?

"--* Ами, аз тепърва ставам\ldots Не мога да мисля много трезво сега\ldots Хайде да говоря с партньора си и да ви се обадя. Какво ще кажете? – отговори Станчо.

"--* Добре. Час?

"--* Мммм – погледна към часовника и след като го фокусира каза. – Десет и четиредесет и седем.

"--* Ох, не колко е часа\ldots Кога ще можете да звъннете, имах предвид\ldots

"--* Аааа. Ами, към три след обяд?

"--* Добре. Ще ви очаквам тогава. Лек ден! – връзката прекъсна преди Станчо да успее да отговори.

Стаматов остави със замах и умора телефона на шкафчето. Надигна се от леглото. Вече беше напълно буден. Главата му пушеше, но нямаше начин да заспи. Отправи се към банята за да си вземе душ. Настрои водата и стоя няколко минути под струята молейки се това да избие махмурлука му. Естествено знаеше, че никога не става толкова лесно. Трябваше да пийне кафе, да постои с празен поглед, да седне пред телевизора и да изчака няколко часа борейки се с главоболието и с желанието да се издрайфа. Приключи с душа и продължи с горе изброените стъпки за борба с препиването.

Заспа пред телевизора и се събуди към два и половина с ясното чувство, че е забравил нещо. Беше зверски гладен, сигурно това бешe\ldots Отиде в кухнята, отвори хладилника и изрови някаква останала пица. Взе я и седна на масата да я яде. Радваше се, че махмурлукът му е минал. След като приключи с яденето, осъзна колко е жаден. Наля си една почтена чаша студена вода и я изпи. Последва втора, която взе със себе си пред телевизора. Беше три без петнайсет. Онова чувство пак се върна. Имаше много важна работа този след обед, но каква беше тя? Мистерия\ldots Отиде да се изпикае. По пътя видя телефона си. Защо тоя телефон бе толкова интересен тая сутрин? – зачуди се той. Изпика се и мина пак покрай него. Нещо прещрака в главата му и той се сети. 

Оставаха пет минути до три. Грабна устройството и набра Иван. ``Тъън, тъън'', ``тъън, тъън'' се чуваше няколко секунди от слушалката. Паниката го обзе, той почна да пристъпва от крак на крак. Изведнъж сигналът спря. От другата страна на линията се чу глас.

"--* Алоу, кой е? А, Станчо ти ли си? Кво става?

"--* Абе, Ванка, тука ми звъня една. Инвеститорка била, станал ѝ интересен нашият бизнес\ldots Нашата идея за бизнес.

"--* Ми супер. И кво?

"--* Ами, пита ме кога ще е удобно да се видим и да приказваме за пари\ldots Иии аз реших да ти звънна да видя какви са ти плановете.

"--* Днес кво сме? Сряда. Начи, в петък след обед по тва време? Как ше си ти?

"--* Ми, супер. Окей, звънкам ѝ да се разберем.

"--* Окей, ай чао.

"--* Чао\ldots

Затвори телефона и се подготви за следващия разговор. Беше точно три. Намери номера в списъка с повикванията и го набра.

"--* Алооо – весело отговори някакъв глас. – Решихте ли какво ще правите?

"--* Ъъъ, да, здравейте. В петък, към три как сте?

"--* Секунда да видя графика. 

От месец графикът беше пуст, въпреки това Люси си създаваше фалшивата илюзия, че имат работа и заради това гледаше постоянно към този празен календар на стената.

"--* Свободни сме. Офисът ни е на 7-ми километър, трети етаж, стая триста и три. Чакаме ви! – пак, без да го остави да проговори, му затвори.

Станчо остави устройството и седна пред телевизора. Диванът го обзе в прегръдката си. Така приключи поредният ден. Може би последният спокоен ден за Стаматов.

\section*{4.}

Трафикът в София е интересно нещо. Пиковите часове всъщност не са часове, а комплексни интервали от време, които съдържат в себе си най-различни методики и стилове на бързина, сладкодумие и ловкост в престрояването.

Естествено, с техния късмет, Ванката и Станчо бяха хванали доста натовареното обедно движение. БМВ-то бавно се придвижваше до кръстовището към входа на Цариградско. Двамата млади предприемачи използваха времето за разговор.

"--* Ся кво ше им говорим на тия? – започна Иван.

"--* Ами, какво\ldots Ще им обясним идеята, как смятаме да я реализираме, какъв добър екип сме и така\ldots То, не му е много да се мисли.

"--* Не му е много?

"--* Ами, аз ги проучих – не са имали успешен проект от няколко месеца. Парите им свършват, отчаяни са. Абе, ще ги спечелим – увери го Станчо.

"--* Хах, а дано!

Ванката ловко отне предимство на някъв бавен и успя по-бързо да се измъкне от кръстовището. Май последва псувня, но това не бе от значение. Всичко бе в реда на нещата.

"--* Е, недей така бе човек\ldots Някой път няма да си толкова бърз и ще стане белята – упрекна го Станчо.

"--* Айде, айде\ldots Сякаш и ти не го правиш понякога. Пък то ако ни е писано да стане, ше стане – след няколко секунди мълчание продължи. – Имам едно приятелче, барман. Ако знаеш ка ги върти тия бутилки\ldots Мале, трябва някой път да те заведа да пием там. Леко скъпо е, ама си заслужава.

"--* Хах – ами днес, след успешното финансиране, отиваме – отговори Станчо.

Останалата част от пътя им продължи с общи приказки. Стигнаха до сградата, взеха пропуски от охраната и продължиха към офиса. Почукаха и влязоха.

"--* Добър ден! – весело поздрави Станчо.

"--* Добър! – след малко закъснение каза и Иван.

"--* Здравейтееее! – отговори Люси, а Маслинчо само промърмори едно отегчено ``Здрасти, здрасти\ldots'' – Та, на време сте! Това ми харесва. Но да се запознаем! Това е Маслинчо, а аз съм Люси.

"--* Приятно ми е! Аз съм Станчо, а това е Иван. 

Те си стиснаха ръцете. 

"--* Настанете се. Тук как е? – подаде им по стол. – Харесва ли ви? Ако искате да седнете до прозореца или другаде? Не? Окей, заповядайте курабийки. Сама ги правих! А нещо за пиене? Е, тук имам само вино, но то май върви със сладки. А? Добре, ето по една чаша – всичко това бе изстреляно, а Ванката и Станчо можеха само да отговарят с жестове. 

Те поеха чашите в ръце, чукнаха се за здраве, хапнаха си по курабийка, похвалиха Люси и замълчаха.

"--* Та -- почна Панайотова, – кажете ми сега за вашата идея. Какво ще правите, как ще изкарвате пари от него, планове за реклама, екип, прочие\ldots

"--* Ами -- Станчо започна пръв, докато Ванката още асимилираше въпросите, – идеята е проста – да помогнем на птицевъдите да си вършат по-спокойно и евтино работата. А как ще стане това? Ами, представете си кокошките ви сами да си правят храната, сами да се слагат да носят, сами да се колят. Това елиминира доста от работата. Даже, може и яйцата да станат самостоятелни – снасят се и сами се нареждат по супермаркетите! Даже си слагат и знак за качество! Вижте – това е революция! А?

"--* Абе, то всичко много добре – отвърна Маслинчо – ама как, аджеба, ще стане тва? Ние тук не сме в бъдещето, че да имаме някакви такива умности\ldots

"--* Ъм? – започна Станчо. – Ами, истината е, че ги имаме – пусна една лека и самодоволна усмивка. – Аз работих в една лаборатория и открих начин да модифицираме генно интелигентността на организмите. С няколко манипулации едно пернато може да измине пътя, който ние сме вървели хиляди години, за един ден. И не само пернато! Малко ѝ трябва на техниката да се усъвършенства за да обхванем и четириноги! А представете си – домати с ум! Представете си просто! Възможно е, но нека почнем от малкото – десет, петнайсет човека трябват и една добре оборудвана лаборатория. Толкова. А печалбите ще са огромни! Огромни.

Всичко това бе изговорено бързо, със страст и леко полудял поглед. Люси беше впечатлена, а Маслинчо бе вдигнал поглед от листите, върху които пишеше.

"--* И колко ще струват тези десет, петнайсет човека и една добре оборудвана лаборатория? – запита Людмилов. – И как така ще произвеждаме нещо без материал?

"--* А, да малка подробност, която бях изпуснал\ldots – отговори Станчо. – Материалът не е толкова скъп – няколко кокошки като за начало и консумативи за лабораторията. Вече персоналът и самото място са по-проблемни. Но аз имам връзки, приятели\ldots Оборудване\-то ще ни излезе по-евтино. А Ванката тука познава хора, които разработвали някакви ``синтетични'' вещества\ldots Та, те можели да се включат.

"--* Окей, супер! Все пак, колко ще струва? – запита Маслинчо.

"--* Е, колко да струва? Ами, заплати по поне 2-3 хиляди на месец, наем – хиляда най-много, реактиви и консумативи – около 100-200 хиляди и това е май\ldots Нали Ванка?

"--* Да, да. Да! – сепна се Иван.

"--* Ухуууу – мекичко и сладичко се впечатли Люси.

"--* Но това е само докато изработим прототипа. После, за серийното производство, ще трябва голямо помещение, обикновени работници, конфиденциалност, маркетинг\ldots Но ще се отплати уверявам ви!

"--* Ухуууу! А това после са си няколко милиона\ldots Ма\-ле, мале. Ми ако не стане? – почна да се вайка Люси.

Тук Станчо се замисли. Естествено, нямаше начин да фалират. Как? Ами ако някой им открадне идеята? Ами ако нещо се случи? Пожар, законова уредба, забраняваща това, смърт, нещо друго? Беше си риск, който за първи път оценяваха. Трябваше да\ldots

"--* Е, как няма да стане, госпожа! – Ванката се бе отървал от замислеността и разпалено почна. – Как няма да стане? Това са пари, които само трябва да минем и да ги оберем. Какво може да се случи? Ма, не говорете глупости сега! Това е бизнес, който не може да фалира! Н е  м о ж е.

Настана поредната тишина. Иван се бе включил толкова неочаквано, че смути дори една гарга кацнала на отсрещния прозорец. Тя погледна умно към него, изграчи и отлетя.

"--* Люси, хайде да си поговорим – след кратък размисъл Маслинчо наруши тишината. – Момчета, бихте ли изчакали няколко минути тук?

"--* Да, естествено – в хор отговориха двамата.

Панайотова и Людмилов излязоха.

"--* Дали ще стане, Станчо?

"--* Маслинчо малко ме притеснява\ldots Люси е готова, но той може да я разубеди. Няма смсъл да спекулираме. Сега ще видим.

Замълчаха. Притеснението от решението отне всякакво желание за разговори и у двамата. След няколко дълги минути инвеститорите влязоха. Маслинчо почна:

"--* Вижте, идеята е интересна и определено би ни спечелила много. Обаче – тук Ванката и Станчо леко побледняха – е много рискова. Ние до сега не сме се хващали с нещо толкова незнайно и непредвидимо. Ще е трудно наистина\ldots

"--* Е, ние ще се справим! – отговори Ванката.

"--* Сега -- игнорира го Маслинчо, – ние пари ще ви дадем, но и ще сключим договор, който ни освобождава от всякаква отговорност ако нещо неприятно се случи с компанията\ldots

"--* Добре, добре! – пак се включи Ванката.

"--* Ще ви финансираме на няколко етапа – първо за прототипа, после само за кокошките с малко помещение и малко персонал. После, ако просперира всичко, ще ви помогнем и с голямото помещение и многото работници. Това е.

"--* Окей, имаме сделка! – без да мисли каза Станчо.

"--* До няколко дни договорът ще е готов, трябва само да дойдете да го подпишете.

"--* Чао, момчетааа! – с това Люси спря бъдещите дискусии. – Ще се виждаме тепърва. Ох, ох!

"--* Довиждане! – отговориха предприемачите.

Излязоха и отидоха да посетят ония бар, който Ванката бе предложил. След няколко дена подписаха договора без дори да го четат. Е, имаха късмет – единственото, което нямаше да им хареса след няколко години, бе това че трябва да снабдяват с кокошки един от предишните стартъпи на Маслинчо и Люси. Е, какво да се прави\ldots Поне си имаха стартов капитал.

\chapter{Стартът}

Всичко започна много добре. Те летяха към успеха. Нямаха пробеми с нищо, всичко бе толкова чудно и странно. Те – един полуидиот и един тъп генен инженер – да развиват успешен бизнес. Но фактите бяха на лице – парите падаха с хиляди в джобовете им. За една година минаха и през трите стъпки на инвестиция, вече бяха създали и първата умна кокошка. За сега тя можеше само да си слага ядене сама, но това беше начало. Правеха и експерименти за другите животни, а също така вече имаха предварителни поръчки. Всъщност, от тях идваха приходите най-вече.

На втората година сериозно се замислиха да си вземат дела от Маслинчо и Люси. Това мислене бе обвързано с прескачане на енергия от неврон на неврон, от клетка в клетка и от мозък в мозък. Ванката имаше най-много проблеми с него, но се справи по някакъв начин. Все пак, Станчо вършеше повечето управленска дейност на високо ниво, а другият се занимаваше с организацията на работниците. 

Мислиха, мислиха и решиха – взимат дела и тва е! Ама, то не беше лесно\ldots Освен, че не беше прочел казуса за снабдяването с кокошки, Станчо не бе обърнал внимание и на точката за неустойката. Е, нямаха няколко милиона обезщетение още. Имаха де, ама им трябваха за друго. По-лошото бе, че това обезщетение растеше в проценти от печалбата. Станчо се замисли, Ванката направи някакъв мисловен опит, не му се получи, опъна една линия и продължи с управлението на работниците си.

Дойде третата година. Вече продаваха първите самогрижещи се кокошки. Имаха малък проблем – кокошките много говореха. Не спираха да бъбрят, да носят клюки и да спорят. Постоянно се пораждаха някакви интриги между тях, боеве и прочие. Набързо върнаха продукцията за малка модификация – сложиха им щастлив рецептор. Сега всички пернати постоянно бяха със замечтан и полузаспал поглед. Ако можеха да се усмихват щяха да имат една пермаментна, тъпа физиономия. Е, това малко спъна Ванката и Станчо финансово, но пък – щастливи кокошки! Мечтата на всеки яйцепроизводител!

На четвъртата вече имаха стабилен приход и клиентела. Няколко нови идеи се прокрадваха. Конкуренция се опита да се появи, но бе доволно затапена от тежката ръка на Иван. Патентоваха метода, който използваха. От патентното бюро ги изгледаха странно, даже почти изкудкудякаха, но се съгласиха. 

Чудеха се дали да продължат с яйцата или с по-едрия четириног добитък. Решиха, че вторият ще е по-опасен ако е умен. Представете си разярените матадори, които си искат парите обратно, понеже бикът предявява претенции за обезщетение на семейството му при фатално наръгване. Не ставаше. Та, те дори кокошките вече бяха почнали по някои места да прокарват някакви политически форми и семейни организации. Естествено, правеха го толкова тайно, че хората не забелязваха. Напоследък някои курници си имаха копие на ``Ферма за животни'' на Оруел. Е, това се забеляза бързо от стопаните и бе подадено като проблем в системата. Отново се върнаха кокошки за премоделиране. Този път направиха, така че да почитат стопанина си като бог. Е, това доведе до странни конструкции в курниците, които трябваше да представляват олтари за жертвопринасяне, но стопаните бяха твърде доволни от увеличените си приходи, че да направят проблем.

Забеляза се сформирането на монополи тук-там. В Индия се образува тръстът ``Кокошка Груп''. В Европа България бе начело, щото Ванката и Станчо решиха да бъдат патриоти и да продават на преференциални цени за страната. Е, това уби дребните животновъди. Появи се ``КООП Шемет'', които държаха всичката кокоша индустрия в страната и половината стар континент. В щатите, американците се научиха да гледат пернати в домашни условия. Ах, колко се радваха те! Все едно са на село или в някоя страна от третия свят! Някои си спомняха екскурзиите до България по тоя повод. В южна Америка се сформираха две групи – една в Бразилия, друга в Аржентина. Двете водеха постоянна война на конкуренция. Даже карнавалът в Рио бе начело с умни кокошки, като тези на Бразилия се състезаваха по представяне с тези от Аржентина. Проблемът бе, че имаше няколко двукраки, хвърчащи създания, които бяха пропуснали гена на щастието и това доведе до един масов бой, няколко безредици и счупена джанта на мотора на един от шествието. Русия – там Путин си направи картел и снабди всички без Китай и Индия. В Китай оправиха проблема с многото население като накараха хората да избират или кокошка, или дете. Гладът надделя и повечето хора се сдобиха с умношки (така вече наричаха умните кокошки). В Африка вече нямаше страни от третия свят. Всички бяха страни от Летящата република – новосформирала се конфедерация, нещо като ЕС, само че с много по-строги закони и изисквания. В резултат нямаше гладни, страните от арабската пролет се умиротвориха, а президентът на щатите тъжно заби поглед в земята, съжалявайки че вече няма да може да изкарва пари от войни. Австралийците – те си говориха на ``уат’с ъп мейт'' и продължиха да си живеят изолирано.

Тази ситуация в света се сформира през първите десет години от започването на продажбите. Иван и Станчо вече се бяха превърнали в мастити, големи Българи. Проблемът за инвеститорите вече не стоеше пред тях – шефове на ``КООП Шемет'' и на собствените си лаборатории за кокошки, те се бяха наредили на трето място в класацията на Форбс, предхождани единствено от някакъв мексиканец и някакъв странен дубайски шейх. Маслинчо и Люси се бяха отдали на хапване и спане, щото бяха спечелили толкова много, че вече не виждаха смисъл да работят. И да, до пет години се предвиждаше българският кокоши производител да излезе на чело на Европа, а може би и на света. Е, без Русия, щото още ни беше страх от Путин. А той, въпреки че бе на преклонна възраст, все още изглеждаше като кален в Сибир и пет-шест войни руснак, и не мислеше да прекъсва четвъртия си мандат.

България също се развиваше доста добре. Начело стояха приятели на двамата бизнесмени и всичко вървеше по усмотрение на Ванката -- Станчо се бе заел с бизнеса. Естествено, Радомирски правеше доста лоши решения, но те бяха замазвани от появилата се на политическата сцена негова леля от бащина страна – една жена умна, напориста и единственият човек останал с всичкия си на чело на държавата.

Та, тази бивша малка република се развиваше предимно военно. Хората бяха малко по-добре, но и цените бяха малко по-високи. Студентите, а вече и учениците, стачкуваха. То вече и те не знаеха що, поради причината, че образователната система ги бе толкова затъпила, че единственото в мозъчния остатък бе мисълта за ядене, секс, пиене и някакво забавление. Сега на мода бяха протестите, така че това и правеха.

Двадесет години след основаването на лабораториите за кокошки, страната бе окупирала къде политически, къде икономически, доста от Европа. Българите се радваха, че връщаха величието си. Да, такъв бе светът и така се развиваше. Няма какво повече да ви разказвам за него. Нека го замразим и да добавим още няколко факта. А след това – продължаваме с нещо доста интересно.

Ванката и Станчо бяха стигнали половин век живот. Живееха си добре и правеха каквото си искат. Решиха да продължат развитието на бизнеса като включат яйцата в уравнението. Представете си – казваха те на потенциални големи клиенти – самопържещи се яйца! Яде ви се, те си пускат крачета, изкачат от хладилника и се сготвят! Не е ли велико? Не беше, но светът бе странен и рошав. Тази идея всъщност успя. И то доста. Обаче нещо страшно се случи\ldots

\part{Историята се развива\ldots}

\chapter{Идеята за яйцата}

\section*{1.}

Беше Великден. По стара българска традиция, семействата се отдаваха на огромното ежегодно мародерство на яйца. О, горките бъдещи пиленца\ldots Толкова много от тях бяха изядени още преди да погледнат белия свят. А преди да бъдат изядени, бяха сварени, боядисани, чукнати, обелени, нарязани и посолени. Това никой бял човек не заслужаваше, пък камо ли тия малки, сладки, още неродени пиленца. Но това си беше обичай от край време. И доста здравословен при това, щото тия тъпчещи се с въглехидрати хора, най-накрая даряваха организма си с нещо полезно – протеини и мазнини. Но\ldots Това е друга тема.

Казват, че злото винаги се наказва. Кога по-късно, кога по-рано, но се наказва. Ами, да, така е. Въпросът бе – какво е злото? За престъпника – държавата направила го беден и принудила го да краде. За кокошия производител – ``КООП Шемет'' взел му всичката клиентела. За бизнесмена – държавните закони пречещи на монопола му. За обикновения българин – тия, богатите, дето са заграбили всичко. За всеки си имаше дефиниция на зло. 

Яйцата не се отличаваха. Те виждаха злото в човека. 

Да, яйцата наистина имаха съзнание. Между белтъка и жълтъка съществуваше напрежение. Посредством него, от време на време се създаваше връзка между тези две части на тази клетка. Тази връзка, спрямо това къде бе създадена, означаваше съответна мисъл. Даже не мисъл. О не! Нещо много по-просто – истина. Слоя между белтъка и жълтъка на яйцето представляваше поредица от заредени и неутрални зони. Това даваше възможност на яйцето да запамети определени чувства. Даже не чувства, а състояния. Тези състояния съдържаха малко информация – това кой ги е причинил и това дали бяха добри или лоши. Така, например, едно яйце разбираше, че е сварено и го е боляло в следствие на човека. Но това не бе достатъчно. То трябваше да комуникира това разбиране. Този процес отнемаше години. Толкова години, че чак сега тези уголемени клетки бяха разбрали, че хората са лоши. Единственото, което им оставаше бе да търпят.

Те, естествено, бяха предали тази информация на кокошките. Пернатите, бидейки тъпи и генетично направени постоянно щастливи, не обърнаха внимание на това.

\section*{2.}

Та, бяхме на Великден. Това е един наистина приятен и борбен ритуал на християнското общество. Той, както всички празници, правеше богатите щастливи, а хората на дъното – нещастни. Даже и безразлични, щото човек ако не е усетил магията на празника, как би съжалявал, че не може да я изживее? Както и да е, интересни работи се случиха по тоя празник. Една случка, която никой не забеляза, бе от особено значение за бъдещето на света.

***

Ванката, както знаете, бе доста боен човек. Глуповат, но боен. Това добавяше към личността му едно постоянно желание за победа, за печалба, за доказване. На този светъл за вярващите празник, той се бе подготвил подобаващо. Десет червени яйца гарантираха мощта му на сцената на борбата. Както никога се бе появил в Парламента на тогавашната Европейска република и бе почнал наред да кара подчинените си да се бият с яйца с него. Той побеждаваше почти винаги, а когато успяваха да го надвият, му даваха втори шанс, на който задължително той излизаше печеливш. 

В следствие на многото изядени яйца, Ванката, противно на очакванията, се разболя. Не беше салмонела – това бе най-интересното. Някаква чудна болест го хвана и запокити в небитието на безсъзнанието. В това състояние той прекара един тежък месец. След това се съвзе и, сякаш по-умен, почна да помага на Станчо в бизнеса. Целта му беше една – да прокара идеята за умни яйца в ``КООП Шемет''.

"--* Виж сега -- често бе започвал да обяснява Ванката, – колко му е да се промени структурата на едно яйце, а? Същото като при кокошките! То жълтъка нали е ядрото на една голяма клетка. Сигурно има ДНК, а? Айде сега, възможностите са големи!

Радомирски бе добил завиден ум за неговата форма и телосложение. Станчо се учудваше, но само толкова. Не си бе и помислял, че всичко може да има връзка с ония яйчен инцидент. 

"--* Човек, не е проблемът в трудността. Въпросът е, че това носи много опасности. Ти даваш съзнание на организъм, който няма никаква интелигентност, никакви морални задръжки. Не можем, не сме дорасли, да го направим. 

"--* Е, кво ще стане? Едни яйца\ldots – спореше Радомирски. – Е, верно, с крака, ама кво толкова? Ще се хванат да завземат света? Да убиват? Глупости! Няколко човека ще ги смажат и толкова.

"--* Добре, но клиентите ще са доволни ли ако стане такова нещо?

"--* Няма да го позволим. Ще направим като с кокошките – ген за щастие и готово. Няма и да си помислят за произшествия. 

"--* Хмм, добрее – малко неуверено каза тия думи Станчо. – Обаче, дали ще донесат печалба?

"--* Ако ги снабдим с нещо уникално, да. Например – сами се проверяват за болести, срок на годност и степен на свареност. Това наистина ще впечатли потребителите. Знаеш, че яйцата са доста отбягвани поради смъртоносните проблеми, които вървят с тях.

"--* Да, уникалността определено трябва да ни е цел. Непременно! 

И така те се съгласиха. Без много да му мислят, обсъждайки неща, които бяха очевидни. Твърде лесно приеха тази нова насока на техния бизнес и дори не се усъмниха в това.

\section*{3.}

``Яйцето бива инжектирано със специален серум. Това спира за няколко минути, докато трае процедурата, засечените от Брайл вълни на активност. След като бъдат спрени, вече може активно да се промени структурата на мисловния процес на яйцето. Добавя се хормон на растежа, модифициран естествено, който прави възможно създаването на малки, гъсеноподобни крачета за дупето на черупката.''

``Следващата стъпка е инсталирането на сензорите. Тук функционирането на Брайловите вълни е от първа необходимост. Те трябваше така да бъдат засечени и интерферирани, че да могат да се включат в активното отразяване на жизнените процеси на този тепърва създаващ се организъм. Поради тяхната особеност и настройка, при проблем те нагряват повърхността на яйцегена и това сигнализира на потребителя, че яйцето вече не е годно за употреба.''

Това беше началото на едно обяснително научно-популярно предаване за новата технология, която ``КООП Шемет'' използваше. То бе направено със съдействието на корпорацията с цел да успокои ропота на някои части от населението. По-точно активисти против изцялото модифициране на съществуващата екосистема. Филмчето, с продължение половин час, обясняваше защо методът е безопасен, защо той никога не би създал нещо лошо и защо в същност така се помага на екосистемата. Общо взето, беше комунистическа пропаганда. 

Та, по този начин положението, което можеше да бъде забелязано по-рано, се влоши и стигна степен до невъзможност за реакция. Хората бяха заслепени от привидно безобидното ново домашно животно – яйцето. То си ходеше, готвеше се, проверяваше се, слушаше. Обаче нещо незабелязано ставаше без хората да знаят. Вижте, точно това модифициране на Брайловите вълни, бе отключило възможността яйцегените да контролират съществата около тях. И понеже черупчестите имаха невероятната способност бавно, но сигурно да предават своите мисли, чувства и наблюдения на събратята си, те сформираха леко полеко едно съзнание, чиято сила щеше в рамките на няколко години да спре така напредналото човечество. 

\chapter{Тихо развитие}

\section*{1.}

Семейство Аспержови бяха типичните българи за тр\-идесетте години на двадесет и първи век. Богати, напети, собственици на двуетажна къща разположена на добро, тихо място сред природата. Те също така бяха едни от първите ползватели на яйцегените. Аспержова, като практична домакиня, бе изцоцала скъпия си и любим съпруг да се сдобият с първите доставки. ``Това -- казваше тя -- ще ни донесе такива ползи! О, Доматчо, моля те, моля те!'' На молбите той устояваше, беше твърде стиснат за да се впечатли от думи, обаче госпожата му си имаше своите трикчета. Видите ли, Аспержов си имаше страст да мучи. Да, наистина, Доматчо обичаше да седне пред телевизора, да отвори студена бира и да мучи. Правеше го всяка вечер, поне по два часа. Грозданка, жена му, беше свикнала и не ѝ правеше впечатление. Обаче тя знаеше, че лиши ли го от тази му страст, заплаши ли го, че ако не спре ще разкаже на приятелите му, той щеше да се примири, да падне геройски и да купи толкова желания от нея Премиум Пак Яйцеген.

Та, беше се примирил човекът, какво да прави. Не можеш да се лишиш от страстта си, да загубиш най-желаното от теб, било то мученето с бира в ръка пред телевизора. 

Вече три години те си живееха щастливо с умните яйца. Нямаха нито един случай на натравяне, или преваряване, или дори изгаряне на омлет. А, повярвайте ми, това беше честа практика на Грозданка. 

Един ден, докато вечеряха и слушаха вече безсмъртната Лили Иванова, те забелязаха нещо странно в Пухчо – техният домашен любимец. Пухчо бе една дебела, рошава коала. Бе толкова мързелив, че имаше периоди, в които умираше от скука, но нямаше ни най-малката капчица желание да се премести от мястото, на което седеше. Тогава съпрузите Аспержови бяха принудени да го хванат, да го преместят и да му ударят една инжекция адреналин за да се освести. Това бяха най-активните периоди на това сладко, но лениво животинче. Тази вечер обаче то бе страшно активно. Двойката реши, че просто нещо луната, циклите не са в ред. Оставиха го да си се активизира.

На другата сутрин го намериха захапало банан, гледащо празно в телевизора кулинарно предаване.  Единственото, което имаше на екрана, бе една микровълнова и една купа с кокошка в нея. Това бе нормално предаване. Тогавашното общество бе станало толкова практично, че използваше специални кадри за втълпяване на знания и умения чрез телевизора. Предаванията, филмите, всичко бе просто един кадър, повтарящ се по специален начин. Хората ни най-малко не се притесняваха, че могат да им промият мозъците. О, не, бяха твърде заети за да се притесняват. 

Та, коалчето така си и стоеше. Опитаха се да го преместят, но то се върна. Притесниха се и се консултираха с ветеринарен лекар. Той дойде, прегледа го, каза че се е побъркало и си отиде. Остави едно хапче, с което да избавят животното от мъките му. Аспержови плакаха, умуваха, съжаляваха, чесаха се, мучаха, но не измислиха друг вариант. Разтвориха хапчето във вода и със спринцовка накараха Пухчо да го изпие.

О, край! Тъжен край!

Още много такива случаи се появиха по този изведнъж сякаш прокълнат свят. После секнаха изневиделица. Хората останаха тъпо учудени. Приятелят ни, Слънцето, пак се напи като свиня и всичко продължи по старо му. 

\section*{2.}

Бобчо бе петдесет годишен шофьор на автобус. Интересното бе как хората през тази съдбоносна 2033 година, все още не се бяха доверили на роботите. По дяволите! Почти всички кокошки и яйца бяха модифицирани, умни! Някои даже управляваха тракторите по селата\ldots Ех, хора. Но да се върнем към темата! Та, Бобчо се почувства зле една най-обикновена сутрин. Обади се в автобусното и помоли за болничен. Помрънкаха, но се съгласиха – беше най-добрият им кадър.

Това чувство на лошо прерастна в силно главоболие, после в желание да гледа непрекъснато телевизия, а накрая в апатия. Бобчо напусна работа, усамоти се у тях, изгони жена си, децата си\ldots Нещо странно се бе случило с този порядъчен, ценящ работата си българин. По цял ден и цяла нощ той стоеше пред телевизора, включил на кулинарния канал. Клатушкаше се напред-назад, течаха му лиги и гледаше. Ако човек бе влязъл у тях в този период, би се ужасил. И не само, би изгубил ума си и изпратен в най-близката психиатрия.

Бобчо бе седнал по турски, бе се хванал за коленете и се разграждаше умствено. Около него бяха всички негови яйцегени. Те стояха и наблюдаваха, докато активно работеха върху мозъка на този с нищо виновен човечец. Хиляди от тях бяха съсредоточени върху разнищването на човешката интелигентност на Бобчо, въпреки че само десет бяха в тази стая. Те обменяха бавно, но устремено информация. Те надграждаха върху това, което бяха видели в по-низшите животни.

Целият процес бе много интересен и странен. Даже, по човешките стандарти, забавен.

Та, девет яйцегена бяха в кръг, а десетият беше излязъл малко по-напред, откъм задната страна на Бобчо. Десетият, който се казваше Ргхтп, пулсираше в червена светлина благодарение на модифицираните си Брайлови вълни. С всеки пулс той изпращаше модифицирана пи-вълна към долната част на гръбначния стълб на Бобчо. Тази вълна после поемаше по своя опасен път към мозъка му. Да, голямо приключение беше вълната да стигне до мисловния център на човек. Цяла книга можеше да се напише, но аз ще го разкажа в няколко абзаца.

Та, хората сме изключително сгрешили за биологията. Всъщност, единственото, което ни движи са едни малки същества наречени марксистчета, които бяха виновни за целия свят, Вселената, Мултивселената, Голямата Черна Марксосфера и Нищото, което следваше след тях. Те бяха нещо като богове. Та, тия марксистчета се бяха погрижили ние нищо да не знаем, да си изградим погрешна представа за това, което ни заобикаля. Те, естествено нямаха контрол върху безмозъчните същества и затова не се бяха вселили в яйцата.

По нашия гръбначен стълб са разположени, на определени пунктове, формации от тези богчета. Те се грижат да пазят пътя от вредители като пи-вълните на яйцегените. В мозъка се намира главната централа на организма. Там марксистчетата са безобидни и лесно победими. Това разделение бе нужно поради странния характер на противоположните съставители на всичко, които са два вида – Пазачи и Умни. Пазачите мразят Умните. Поради това и двете касти са разделени. Единственото, което кара Пазачите да пазят Умните бе опасността от елиминирането на Голямата Черна Марксосфера от капиталистчетата.

Та, пи-вълната тръгваше от долната част на гръбнака. Стигнеше ли до пост, тя се врязваше в него. Това унищожаваше по едно марксистче-Пазач. Ргхтп пускаше следваща инвазия и цикълът продължаваше. Няколко дена бяха нужни за да се стигне до Централата. Веднъж там, ``пи'' лесно премахваше Умните. После взимаше един неврон информация и се връщаше да го донесе на яйцегена. На стотното отиване и връщане яйцегена се пълнеше. Тогава той предаваше наведнъж поетото знание на другите около него, а те го споделяха с яйцеген света. Това беше втората бавна част от процеса. Благодарение на нея, те научиха много за човешкия Център и как той функционира. 

Бобчо не беше единствената жертва на яйцегените. От него започна, но много други хора по света също бяха афектирани.

Човек, след като бе загубил умните си марксистчета, бе просто една черупка. Той почваше да прави единствено инстинктивни действия. Готвеше си, ядеше, гледаше телевизия, не излизаше. Като му свършеха продуктите ставаше лошо. Това го принуждаваше да отиде до магазина. Сблъсъкът с други хора тотално разбиваше психиката му. Той почваше да си въобразява, да си приказва, да се побърква. Накрая го отпращаха в лудница. 

Тия зачестили случаи не направиха особено впечатление в началото. Е, света става все по-труден, си викаха психиатрите. Хората, е така, от стрес, се съсипваха. Когато клиниките почнаха да се пълнят, вече стана страшно. Тъкмо стана страшно и случаите секнаха. Някои по-будни хора имаха достатъчно време да забележат, че изключително голям брой от постъпилите в психиатрия бяха бивши собственици на яйцегени.

\section*{3.}

Господин Георгиев бе човек на принципа. Поради тая причина днес, в този горещ летен ден, се бе запътил към офиса на ``КООП Шемет''. Беше си уговорил среща със Станчо. Не беше казал за какво ще се виждат, но Стаматовия мозък беше задвижил един мисловен процес, който караше управителят да чувства някакво странно неразположение. Станчо го отдаде на факта, че имаше проблеми с перисталтиката тая сутрин, причинени от лютото шкембе, което бе унищожил преди няколко часа като лек срещу махмурлука. Пръцна напористо, уплаши се и се облекчи в близката тоалетна до офиса, след което се подготви за срещата.

Георгиев, първото име, на който беше Панайот, почука на вратата с табелка ``Управител''. Осанката на този принципен човек бе леко анорексична. Носеше риза с къс ръкав, панталони на избледнели ромбчета и проветриви обувки. Имаше пари, но не знаеше за какво да ги харчи освен за спазването на принципа. Този принцип го бе докарал до пълно отричане на личността си в името на закона, моралното и света. Ядеше малко, само когато наистина беше нужно. Не пиеше, не пушеше, нямаше жена.

"--* Добър ден! – поздрави го Станчо.

"--* Здравейте, дошъл съм тук\ldots – започна Панайот.

"--* Чакайте, чакайте\ldots Настанете се. Да пийнем по нещо?

"--* Не, не мога да чакам! Вижте, това, което направихте с яйцата е постъпка заслужаваща най-великото наказание, което този свят може да предложи!

"--* Е, не така, господине! Веднага нападате. Нека седнем, кажете какъв е проблема? Да не би някое яйце да не е показало, че е развалено? Или пък да се е преварило? Веднага ще ви снабдим с безплатни нови!

Георгиев почервеня. Стаматов настойчиво го седна на един удобен фотьойл, а самият той се настани на стола зад бюрото си.

"--* Знаете ли, господин Стаматов, колко луди има в момента в психиатриите? Много – не дочака отговора на Станчо. – Толкова много, че сме удвоили количеството на клиниките\ldots

"--* Е, тежък живот\ldots Сега, светът стана много по-труден напоследък. 

"--* Не ме прекъсвайте! Знаете ли, че всеки един новопристигнал е бил собственик на яйцегени?

"--* Е, сега, обвинявате ли ни? Не бива така. Без доказателства, не може така да се обвинява. Засрамете се!

"--* Не, вие се засрамете! По-точно се замислете! Тези ваши яйца правят нещо на хората и мога да го докажа!

Стаматов застана на нокти. Нещо имаше. Нещо бяха сгафили и то много. Реши, че е най-добре да изслуша този нападащ го никаквец.

"--* Добре, кажете сега защо смятате така.

"--* Новопостъпилите пациенти през последната година са с психично заболяване непознато досега на човечеството. Всички до един – абсолютно до един! – са били собственици на яйцегени. Всички, в последната фаза преди да влязат при нас, са изгонили семействата си, усамотили са се и единствения път, когато са излезли е било да си купят храна. Това е довело до тоталното им психическо омаломощаване и приемането им при нас.

"--* Е, цените са си високи днес – вметна Станчо.

"--* Стига глупости! Фактите говорят. Имаме доказателства и сме готови да заведем дело от името на народа. От утре почваме подписки срещу вас и вашата корпорация\ldots

"--* Това означава да съдите една от най-силните държави в света – изпусна се Станчо. Все пак, знаеше се, че зад политиците седи бизнесът, но не биваше да се казва толкова публично.

"--* Какво? Значи потвърждавате, че стоите зад властта? – усмихна се Панайот. – Стана по-зле за вас, знаете ли?

"--* Какво да знам?

"--* До месец ще получите призовка. Ще се видим пред съда – Георгиев понечи да си тръгва, но Станчо побърза да го задържи.

"--* Не бързайте господине. Можем да уредим въпроса\ldots

"--* Подкуп?! За бога! Става по-зле за вас. Много по-зле!

"--* Не, не, как подкуп? За какъв ме вземате? Вижте сега, дайте малко време. Ще го обмислим с управителния съвет и ще предприемем извънредни мерки.

"--* Колкото и да го обмисляте, злото вече е станало. Давам ви три дена. 

Георгиев си тръгна, преди Стаматов да може да пророни и дума. Следващото действие на управителя беше ясно – звънка на Иван и бързо се събират за извънредна среща. Нещо прещрака в мозъка му. Някакви мисли, странни и несрещани до сега се появиха в главата му. Мисли за величие, за това че няма от какво да му пука. Той беше собственик, да, собственик, на една от най-великите сили на днешната политическа карта. За малко следващите му действия да бъдат определени от тази идея, обаче се почука на вратата му. Стана и отвори. Беше Иван.

"--* Добър ден, колега! – извика съдружникът му.

"--* Какъв добър ден?! Сядай и слушай! Загазили сме го и то здравата.

"--* Е, що? Кво има? Дай малко по-леко.

"--* Дойде един и заплаши да ни съди. Яйцегените, казва, докарвали някакво ново психично заболяване, от което страшно много хора били афектирани\ldots

"--* Е, ся\ldots – прекъсна го Иван.

"--* Не ме прекъсвай! – Станчо беше почервенял, готов всеки момент да поломи всичко около него. – Дава ни два дни. След това ще събира подписи срещу нас. Ще ни съди, ще съди най-голямата, най-силната и влиятелна корпорация в света. Можеш ли да повярваш?! Загазили сме го здравата и трябва бързо да се измъкнем от ситуацията!

"--* Успокой се Станка. Спокойно! – Ванката се бе уплашил от погледа на Стаматов. Въпреки, че беше избухлив сега си наложи да влезе в положение. Годините бизнес го бяха направили малко по-умен. – Дай да седнем сега. Хайде, хайде. Сега ще решим всичко!

Станчо седна, Ванката наля по уиски и подаде чаша на колегата си.

"--* Сега – поде той, – сега вече може спокойно да помислим. Сам каза, че сме могъщи. Какво ни пречи да спечелим делото?

"--* Да го спечелим? – след няколко секундна пауза каза бившият генен инженер. – Естествено, че ще го спечелим. Това е най-малкият ни проблем. Обаче, представи си кво ще стане след това. Едно такова дело е най-лошият ПР, който можем да си направим. Ще изгубим много, може би и всичко. Не, всичко няма да е, но ще е много\ldots

Настана мълчание. Накрая Станчо каза:

"--* Хайде да свикаме управителния съвет. Повече хора мислят по-добре.

Та, свикаха съвета. Говориха около четири часа. Естествено, както при всички такива срещи, най-вече се симулираше дейност. С по-добра организация можеше да се стигне до решението, което взеха, за трийсет минути. Какво беше то? Да почнат акция за изземането на яйцегените. Трябваше да е със съдействието на всички заинтересувани. Претекстът? Неоправдани разходи, които могат да се вкарат за по-добри продукти на фирмата. Естествено, щеше да има обезщетения за клиентите.

\section*{4.}

Не всеки човек би се съгласил да се раздели с нещо, което толкова улесняваше живота му. Тези, които не се съгласяваха, осъмваха с откраднати яйцегени. Тогавашното общество бе толкова промивано с телевизия и безмозъчност, че не се и замисляше, че може от ``КООП Шемет'' да са ги взели. То тук идва и въпросът как са се усетили за проблема с яйцегените въобще. Е, имаше някаква пренебрежима бройка умни индивиди.

Обаче, знаете ли, проблемът само беше замазан от това изземане. Яйцата бяха сложени на сигурно, в огромни подземни складове. Веднъж използвани, тези големи сладове се заравяха и всякакъв достъп до тях беше невъзможен. Проблемът обаче, беше че няколко яйцегена бяха взети за изследване. Учените се интересуваха какво става в ядрото на тези модифицирани организми.

Прости наблюдения върху индивидите от човешки произход показват, че хората са си колкото любопитни, толкова и глупави. Желанието да узнаеш нещо те кара да пропуснеш важни елементи и детайли, които по принцип са очевидни. Например – нима хората тясно обвързани с яйцегени не се побъркваха? И въпреки това се позволяваше на едни от най-светлите умове на тогавашния свят да работят с тези яйца. Това си доведе до своите последствия. 

Условията, в които работеха учените бяха възможно най-сигурни. Бяха предприети мерки за защита от пи-вълните, чрез специални фолиа, които покриваха костюмите на изследователите. Отделно, самите лаборатории бяха също изолирани, а и сградата, в която се намираха бе доста отдалечена от света. В същност, сграда беше доста неподходящо понятие – това беше разположено в пещера помещение. На входа имаше една огромна, тежка врата, вратище ако трябва да сме точни. Единствено учените имаха право да влизат и то след доста щателни мерки за безопасност. Излизането беше още по-трудно – задължително трябваше да бъдат минати психотест и преглед от психиатър.

Но кой казва, че яйцегените не са адаптивни? Разбирайки, че методът, по който до сега са работили е несполучлив,  те трескаво търсеха нов начин за вникване в Централата. Учените процедираха доста бавно и внимателно с ограничения си запас от генномодифицирани яйца, така че имаше доста време. Много време. Интересно беше как от една страна хората си мислят, че разнищват това, което стои зад тези страшни организми и как, от друга страна, тези страшни организми изследваха хората и мислеха как да пробият тяхната система. Беше състезание, за което беше уведомен само противниковият отбор. 

Това, което предпазваше хората от пи-вълните беше доста универсално – фолио, което отразява всякакви лъчения. Проблемът беше, че то не отразяваше частици или ако го правеше, успяваше само в някакъв много малък процент. На яйцата, с техния колективен ум, им бяха нужни две седмици да осъзнаят това и само два дни за да пренастроят пи-източниците си да излъчват частици. Единственият проблем, който стоеше пред тях бе, че частиците проникваха по-бавно. Вместо дни, сега им бяха нужни седмици за да установят контрол върху Центъра на човек. Положението се усложняваше още повече поради факта, че тези човеци, върху които се работеше, бяха най-умните глави на земята, а умността  означаваше просто повече умни марксистчета и още повече марксистчета-пазачи. 

Първият случай на повален изследовател бе неочакван. Той не доведе до засилена тревога в лабораторията просто, защото с този пръв случай, последваха, в много кратък интервал от време, още три случая. До края на деня почти всички бяха повалени. Яйцата бяха завоювали първата земна позиция. Поддържаха няколко живи трупа, които да имитират дейност и да пращат фалшива информация на правителството. Това беше нужно само докато яйцегените успеят да се размножат – колкото повече от тях действаха върху някой индивид, толкова по-бързо ставаше сривът на Централата му. Естествено, трябваше да бързат с размножаването, щото колкото и да имитираха дейност, не можеха да оправдаят това че учените не се сменяха в продължение на няколко седмици.

Размножаването беше процес наподобяващ клетъчното делене обаче в много по-голям и бавен мащаб. Самият яйцеген първо делеше вътрешността си по същия начин, по който се дели клетка. След като се образуват двата вътрешни яйцегена, около тях се съставяше тънък филм, който едновременно ги разделяше и ги подготвяше за втората фаза на процеса. Черупката се разпукваше по средата с много бавни темпове. При контакт с кислорода, обвивката около новите клетки се втвърдяваше и постепенно ставаше на черупка. Появилият се нов слой твърдост отлепваше окончателно старата предпазна покривка. Новите яйцегени почти веднага влизаха в активно функциониране. 

Този процес отнемаше ден. До края на седмицата от трийсет организма предвидени за изследване, станаха хиляда деветстотин и двайсет. Десет яйцегена прекършваха един човек за два дни. Сто – за около десет минути. Това беше много време, а повече от сто нямаше как да обработват един организъм. Поради тая причина яйцегените трябваше да измислят по-ефективен начин за обработка на хора. Тая лаборатория, която сега бяха окупирали, им предоставяше много възможности. За няколко дни, с помощта на хората-черупки, сглобиха уред, който събираше пи-частици или пи-вълни, усилваше ги и ги концентрираше върху отделен индивид.

\chapter{Първият яйчен заговор}

\section*{1.}

Един горещ следобед, по време на който Слънцето напичаше с пиянска упоритост, в една малка, задушна стая, се бяха събрали около десет човека. Те не усещаха жегата, а само седяха и мислиха. От пет дни не можеше да бъде установен контакт с главната лаборатория на ``КООП Шемет''. Всичко, което бяха измислили до сега, бе да пратят отряд военни да разберат какво става. Обаче се страхуваха. Никой от събраните не знаеше защо нямаха желание да влагат сила. Всички предчувстваха нещо и упорито прогонваха идеята, че яйцегените може, все пак, да са окупирали лабораторията.

"--* Окей, хора -- отново се опита да поведе мисловна дейност Ванката, – да помислим малко, колкото и да ни е трудно – какъв е шансът да са атакували терористи? За тази лаборатория знаем само ние и учените, които работят там

"--* Ако ще се обвиняваме, дайте да го правим по-точно, а? – с усмивка отвърна Първопостъпилия.

"--* Е сега, колега, не с лошо. Просто трябва да елиминираме всички възможности – отвърна Ванката.

"--* И това е възможност, според вас? Каква полза ще има някой от нас да окупира този обект? Нали дори да се опита да установи контрол върху корпорацията по тоя начин, ще бъде лесно пречупен\ldots Все пак, тези яйца се изследват от чисто научна гледна точка – не възлагахме много надежди на тях. Пък и защо не е предявил исканията си до сега?

След това настана тишина. Едно куче излая. Една котка мина покрай него, шибна го със злобна лапа и псето изскимтя. Котката се изнерви, изсъска и избяга нанякъде. Световното биологично равновесие бе що-годе възстановено. Това сякаш поизбистри умовете на събралите се.

"--* Добре -- поде Внимателния, – ако са яйцегените?

Настана още по-гробна тишина. Хората се изпотиха от мисълта и усетиха горещото време.

"--* Ако, ако\ldots -- ядоса се Станчо. – Дайте да пратим военните, да видят. Те знаят как да си държат устата затворена, ако я натъпчеш с достатъчно валута. 

"--* И кво? Да похарчим няколко милиона за операция? – отговори Стиснатия.

"--* По дяволите, да! Не знаем какво се случва там! Представете си последствията в световен мащаб, ако наистина са яйцегените! Представете си какво ще се случи с нас! Това, че сме най-мощната фирма в света, която е начело на една от най-силните държави, не ни прави безсмъртни! Ако света реши да е срещу нас, то ние ще го загазим яко – каза Станчо.

"--* А ако се случи нещо с войниците? – почуди се Внимателния. – Ако наистина яйцегените са направили\ldots нещо\ldots там, значи учените, с техните предпазни костюми не са успели да се защитят. Тогава какво остава за момчетата, които ще пратим?

"--* Много добре знаете – отговори Първопостъпилия, – че яйцегените действат бавно. Учените разбраха, че на десет от тях са им били нужни около два дни за да установят контрол върху човек. Нашите хора ще влязат, ще ги очистят и всичко ще тръгне по мед и масло. 

"--* Ами, да действаме тогава -- каза Станчо.

\section*{2.}

През трийсетте години на двайсет и първи век, армията бе предимно съставена от разузнавателни групи и малко специални части в случай на нужда от сила. За големи акции се използваха сателити оборудвани с бойни глави. Войни нямаше толкова – хората бяха разбрали, че главата се прекланя по-лесно пред човека с парите, отколкото пред този с оръжието. Все пак, имаше инати, при които трябваше да се действа по стария, изпитан начин. 

В един корпус на специалните части, намиращ се на тайното място Симеоново, София, България, кипеше трескава дейност. Седем младежи, на около 20-25 години, едни от най-качествените кадри на българската армия, минаваха през преподготовка за мисията на живота си. За първи път щяха да влязат в бой, надяваха се да влязат в бой. Надяваха се да усетят сладката тръпка от убития враг. Защото тогава войниците бяха учени да се наслаждават на убийството, а не да го правят единствено от нужда. Това щеше да е лошо, ако количеството войни бе като в началото на двайсет и първи век. Сега обаче, един войник рядко беше участник в битка, така че малко повече злоба и желание бяха от полза.

Капитан Маринов въвеждаше отряда си в детайлите около мисията:

"--* Целта е установка на ``КООП Шемет'' намираща се в южната част на Стара планина, в близост до Карлово. Мисията е с разузнавателна цел. Работа при код жълто, като има вероятност кода да стане червено. Пробиваме от предния вход. Той е здраво укрепен – ще използваме взриво-зашеметяващо устройство тип ПушенаВодка-25 – последва лека вълна на задоволство сред въвежданите. – Почистваме, уверяваме се и напускаме. Действаме бързо, чисто и без жертви. Момчета, това е мисия с код на опасност 1. Първата от историята на нашата армия. Въпроси?

"--* Къде ще отпразнуваме победата? – попита редник Малев.

"--* При майка ти – отговори му ефрейтор Божков. 

Последва смях от страна на останалите и раздразнение от страна на капитана. Тия момчета въобще си нямаха представа какво ги очаква.

"--* Стига! Не му е място за детските ви шегички! – изкрещя Маринов. – Това е опасна мисия, към която трябва да се отнесем с повишено внимание!

Замълчаха и се замислиха. Кво пък? Веднъж се живее. Кво като е опасна мисията? То иначе щеше да е скучно\ldots За кво сме войници? За да се крием от опасността? Не, трябва да я приемем такава, каквато е и да ѝ се изсмеем в жалкото, гранясало лице.

Та, такава беше ситуацията в тогавашната ни армия. Така беше и по света. Неглижирано\ldots Всичко бе правено през пръсти в този толкова функционален свят. За всичко си имаше устройство, лекарство, начин, решение. Хората бяха затъпели или по-скоро не им пукаше.

\section*{3.}

ПВ-25 не представляваше просто пластичен експлозив, който трябва да сложиш на някоя врата и да чакаш да гръмне. То беше нещо по-сериозно, цяла установка. Разположиха я на сто метра от входа. Това щеше да е достатъчно за да го взривят без да има пострадали войници. 

Стреляха. Пушек и осколки се разнесоха навсякъде. След няколко минути димът около вратата се изчисти. На мястото зееше нащърбена дупка.

"--* Окей, един по един. Бързо, бързо! – изкомандва водача на взвода.

Влязоха, подсигуриха входа и осветиха мястото. Не видяха и най-малката следа от разрушение. Всичко бе непокътнато.

"--* Хах, май нямаме мноо работа тука\ldots – сподели мислите си един ефрейтор.

"--* Тихо там! – скара се командира. – Сега, тук е чисто, но не се знае кво е навътре. Влизаме в главната зала. От там се разпределяме на две и чистим подред. Разбрано? Окей!

Процедурата тръгна, както я бе обяснил Маринов. В главната зала също не намериха никой. Почна да става обезспокоително, че няма хора в този комплекс. В мислите на войниците почна да се прокрадва някакво чувство на притеснение. Не беше така по принцип\ldots Когато отиваха на мисия винаги имаше кой да убиват. А сега – мъртвило.

Продължиха чистенето и по стаите – нищо. Събраха се пред вратата на малкия ядрен реактор, който снабдяваше с ток лаборатоията. Бяха шест, с командира – седем. Седем безразлични лица, скрити под маски, но чувстващи тъй както обикновения човек. Нещо ги глождеше, бяха леко притеснени.

"--* Хайде момчета, тва е последното място. Изчистим ли тук, се прибираме. Внимателно и бъдете готови за всичко – каза Маринов.

Вратата, естествено, беше заключена. Ако не им пречеха маските, войниците щяха да чуят мърморене от другата страна. Сложиха пластичен експлозив на пантите и на заключващия механизъм.

Отдалечиха се. Натиснаха спусъка. Взрив и след това забрава.

\section*{4.}

Ефрейтор Божков се събуди с лек главобол. Това, което виждаше бе черен таван. Вцепенението внимателно го напусна. Спомняше си, че последно бяха взривили вратата. От там нататък всичко друго му се губеше. Погледна настрани – нищо. Беше сам. Понадигна се на лакти и видя, че беше в някаква стая. Единственият мебел тук бе той. Стана, главоболието леко спадна. Почти без проблеми се изправи окончателно на крака. Поразтегна се. Забеляза, че всичко си му беше тук – оръжие, маска, екипировка. Видя врата срещу него и се запъти към нея. Натисна дръжката и без проблеми я отвори.

Отново тъмнина. Той махна предпазителя на оръжието си и се огледа. Отново празнота.

Според плана, трябваше да се намира при общежитията. Те бяха на нивото на реактора, обаче от противоположната страна. Един дълъг коридор водеше до стълби, а по тия стълби човек се качваше в главната зала. После още един коридор и щеше да се озове пред вратата, която взривиха.

Тръгна като предпочете да не проверява какво има по стаите. Имаше време за неговите хора. Пък и щом той се беше събудил и влязъл толкова бързо в ситуацията, защо и те да не го направят? Приближавайки първите стълби, чу странен шум. Спря и наостри уши. Огледа се, беше спокойно. Изкачи се до главната зала. Малко преди да влезе в нея, я огледа. За сега нямаше никой в нея. Продължи като внимаваше за някоя неочаквана атака.

Главоболието му се бе усилило.

Понеже коридорът водещ към реактора беше прав, без разклонения и опасни ниши, той го премина бързо. Стигна до мястото, където бяха взривили вратата. Нищо интересно не забеляза – нормалните отломки и паднала напред метална грамада. Зад сформиралата се дупка имаше тъмнина, но се и чуваше едно плашещо мърморене.

Той пристъпи с ускорен пулс. Не знаеше какво ще види след като прекрачи прага на вратата и това го караше да се страхува. Не смяташе за добра идея и да си включи фенера, защото по тоя начин щяха да го разкрият. Единственото, на което се надяваше беше това очите му да свикнат бързо с тъмнината. 

Влезе. В залата нямаше нищо друго освен реактора. Мърморенето продължаваше. Беше го страх да продума за подкрепление. За сега май успяваше да бъде незабележим. Ако изречеше и една дума, обаче, можеше да развали всичко.

Изведнъж нещо се раздвижи от противоположната страна на залата. И той забеляза – в тъмните ъгли имаше хора облегнали се на стените и не вършещи нищо. Беше твърде тъмно за да види и техните погледи, но ако можеше, щеше да разбере, че те бяха празни. Една от тези фигури бе тръгнала да става. Вцепенението го напусна и тъкмо да реши да стреля, му дойде мисълта, че може да събуди останалите. Не знаеше колко бяха бързи, трябваше да почака. 

Нещото се движеше нормално. Ако беше в друг контекст, можеше да си помислиш, че е обикновен човек, тръгнал по задачи. То отиде до реактора. Погледна дисплея показващ състоянието му и натисна някакви копчета. След това действие се върна пак на мястото си.

``Роботи, помисли си ефрейторът, изпълняват само и единствено това, за което са в комплекса. Дали ядат? Със сигурност.'' Не беше добра идея да ги застреля. Та те са били хора, със семейства, с мечти и грижи. Сигурно ако оцелеят ще има някакъв шанс да се възстановят. Върна се към главната зала.

"--* Обажда се ефрейтор Божков – започна да приказва по радиостанцията. – Отрядът беше изненадан. Само аз съм в съзнание за другите не знам. Учените са зомбита. Определено са под контрол на нещо или са им разбъркали ума. Пратете още хора. Няма следа от яйцегени. Опасност 1, код червено със сини проблясъци.

"--* Разбрано – изпращя говорителят.

Божков тръгна към мястото, където се бе събудил. Надяваше се положението да се закрепи. Междувременно се чудеше къде са се изгубили яйцата. Някаква неприятна мисъл се загнезди в него. Сякаш имаше нещо, което пропуска. Е, рано или късно щеше да се сети.

Влезе в коридора на общежитията. Той беше дълъг, със стаи от двете страни. Почна подред. Трябваше да намери шест човека. Стаите бяха около петдесет. Надяваше се да не завари нещо животозастрашаващо. 

Намери командира в стая номер седем. Човекът спеше сладко, в пълно снарежение. Ефрейторът се надвеси над него и почна да му вика да се събуди. Ни говор, ни картина. Принуди се да го шамароса.

Шамар – мрънкане. Втори – малко изгрухтяване. Трети – спящият рязко отвори очи и погледна подчинения си. 

"--* Какво по\ldots – тръгна да вика Маринов.

"--* Тихо, тихо\ldots Нещо стана като взривихме вратата. Аз се събудих и огледах положението. Мислих си, че и вие ще направите същото, но явно вас ви е ударило по-яко\ldots

"--* И трябваше да ми биеш шамари? -- отговори командирът.

"--* Повярвайте ми, заболя ме сърцето докато го правих -- каза Божков.

"--* Аха\ldots Айде, стига приказки. Другите къде са?

"--* Предполагам, че по стаите. Не вярвам да са се събудили и те\ldots

"--* А, къде сме? -- запита командирът.

"--* В частта с общежитията. Явно тук им е нещо като затвор, знам ли -- отвърна Божков.

"--* Добре, да вървим.

Продължиха с обиколката. По същата процедура събудиха и другите взводници. Станцията на Божков изпращя по едно време за да му съобщи, че подкреплението идва до двайсет минути. Той им отговори да внимават като влизат. Сборен пункт – залата преди реактора.

Маринов му се скара, че така се прави на командир. Ефрейторът му се извини и добави по радиостанцията да се свързват вече с капитана.

Докато ходеха към залата, на редник Малев му хрумна една неприятна мисъл.

"--* Абе, ние като влизахме нали взривихме вратата? – запита той.

"--* А, не бе\ldots Влязохме през задния вход с шперц – някой му отговори. – Естествено, че я взривихме\ldots

"--* Идиот\ldots – възмути се Малев. – Като и да е, Божков казва, че няма яйцегени, а сме претърсили всичко. Представете си, в една хипотетична вселена, че тия яйца са излезли да се порадват малко на хубавото слънце?

"--* Ми да си се радват. Кой ги спира? – отговори същия.

"--* Абе -- включи се командирът, – тъпанар! Естествено, че никой не ги спира\ldots Ама, трябваше да предвидим тва шибано нещо\ldots Сега как ще ги намерим? Нали тях трябваше да изтребим? Пълен минус е тая мисия вече\ldots

"--* Е, споко шефе – обади се шегаджията – всичко ше тръгне като хората. Ся иде подкрепата, а тя по път сигурно ше ги намери.

Маринов преглътна другарското обръщение и се замисли. Те добре ще ги намерят\ldots Периметърът е осигурен и с електрическа ограда, и с охрана отвън. Ама, какво стана, когато взривиха вратата на реактора? Нещо ги удари, ама Божков казва, че единствено е видял зомбитата учени. Нещо миришеше тук и това не беше миналогодишният сандвич в джоба му.

"--* Момчета -- започна той, – замислихте ли се кво ни удари като взривихме оная врата? 

"--* Що да ни е удряло нещо? – за пореден път тъпо се намеси шегаджията.

"--* Нали до ся беше припаднал, бе\ldots – вметна Божков.

"--* А, да\ldots Ами, сигурно там яйцата са ни зашеметили с техните си психо-глупости – опита да се измъкне оня.

"--* Е, нали ни казаха, че те действат бавно\ldots Как ше ни зашеметят толкова бързо – каза Малев.

"--* Ми де да знам\ldots – отвърна шегаджията.

На медика му хрумна, че може да са използвали учените за някакво оръжие. Нещо по-силно, концентриращо. Като лупа? Слънцето колкото и да си пече, много бавно би ни изгорило, мислеше си той, но ако сложим лупа между лъча и нас, ще ни стане горещо. Та и яйцата така – използвали са ума на учените и са направили някакво такова устройство. Психо-лупа да кажем\ldots

"--* Абе, хора – включи се медикът – може да са използвали лупа – някой го напуши на смях. – Не, сериозно! Нали знаете за Слънцето – и той обясни метода. – Кво мислите?

"--* Може – каза командирът. – Ама, материали откъде ще вземат? Тук почти всичко е непокътнато.

"--* Всичко? Общежитията бяха оголени. Нямаше легла, огледала и маси\ldots – сподели Малев.

"--* Да, прав си\ldots – отговори Маринов. – Но кво ше правят с тва дърво и тия стъкла?

Вече бяха в залата. Всичко беше спокойно. Сложиха двама на входовете и се приготвиха за бързо хапване. Все пак сигурно беше минал поне ден откакто почна мисията.

Тъкмо привършваха с яденето и по радиостанцията на Маринов, се чу изпращяване. След това тишина. Замислиха се. Почти беше време да дойде подкреплението.

"--* Екип Б – обади се командирът, – колко време ви остава до пристигане?

Мълчание.

"--* Екип Б? Къде сте?

Нищо.

"--* Тук главен екип, защо екип Б не отговаря? – Маринов включи на честотата на базата.

"--* Сега ще проверим – пауза. – Екип Б не установява контакт. Посетете ги.

Връзката спря. Всички около командира бяха застинали и в очакване. Е, те знаеха какво предстои да правят, но бяха уплашени. За пръв път се сблъскаваха с нещо неизвестно. Щом яйцата са били подготвени за подкреплението, значи нещо ставаше.

\section*{5.}

Небето беше облачно. Подухваше лек ветрец, от който тревата пред входа се полюшваше. Почти беше възможно да завали някой дъжд или пък да се появи някоя буря. Това войниците не ги касаеше, щото те бяха длъжни, независимо от обстоятелствата, да действат. 

Предполагаемата локация на екип Б (``Буря'') беше на около километър североизточно от входа. Там бе приет последният сигнал от тях. Понеже базата беше секретна, се бяха постарали да я скрият. Около мястото, където взривиха огромната врата, имаше гъста гора, тук таме проходима. Не можеше и да се мисли за наземен превоз до лабораторията – всичко ставаше по въздух. Та, за да стигнат до този ``километър североизточно'', войниците трябваше да преодолеят голяма част от тази гора.

За същества на два крака това не беше трудно. Обаче трябваше същевременно да се оглеждат за яйцегени и разни други предполагащи внимание предмети. Тук таме се чуваха разни птици и други животни, понякога нещо притичваше покрай тях, но нищо по-сериозно не им се случи по пътя. Накрая стигнаха до по-рядък слой дървета, след който следваше едно значително открито пространство. Там трябваше да се намира ``Буря''.

"--* Няма никой – констатира редник Малев.

Всички стояха в редица, мълчаливи. Беше пусто. Следа от екип Б нямаше, дори тревата не беше отъпкана.

"--* Тук главен екип, не намираме Буря – каза по радиостанцията Маринов.

"--* Интересно – отговориха му. – Километър североизточно?

"--* Да.

"--* Ще се опитаме да засечем отново трасиращите им устройства. Засечени са в базата – след няколко минути отговориха.

"--* Супер\ldots – каза командирът и прекъсна връзката.

Потеглиха към базата. Всичко се повтаряше. Сигурно докато стигнат, онези ще са се събудили и ще викат подкрепление\ldots ``Така се върти животът,'' мислеше си Малев. Другото, което глождеше почти всички, беше къде са се дянали яйцата. Ако са минали през оградата с тяхното велико устройство, ако въобще го имаше, щеше да стане много опасно.

"--* Божков -- почна Малев, – как мислиш, дали са успели да излязат?

"--* Кои, яйцата? Възможно е\ldots

"--* Аха\ldots

"--* Чудно ми е обаче -- продължи ефрейторът, – ако това устройство съществува, как го движат\ldots Та те са само яйца.

"--* Не виждам проблем да включат учените. Сам каза, че са ги зомбифицирали\ldots

"--* И пак\ldots Изисква се координация\ldots Твърде забележимо е всичко\ldots Твърде просто.

"--* Кво имаш предвид? – попита Малев.

"--* Ами, няколко яйцегени да подемат такава операция\ldots Да не можем да се справим с някви нищо и никакви си яйца. Тва е абсурдно\ldots – възмути се Божков.

"--* Много неща в света са абсурдни\ldots

Стигнаха до входа и влязоха. Почти нищо не се бе променило откакто тръгнаха да търсят хората от другия екип. Никой не се учуди и колко лесно стигнаха до стаите и как бързо събудиха другите седем. Това, което Буря им разказаха бе, че неочаквано нещо ги е зашеметило както са си ходили. Почти същата ситуация като тази с четата на Маринов. 

"--* Сега кво ще правим, колега? – обърна се Маринов към командира на Буря, Николов.

"--* Ами, аз викам да намерим тия неща\ldots С това оръжие доста поразии могат да направят, а сигурно вече са излезли.

"--* Ами учените? – вметна медикът. – Те са доста зле, но може би има шанс за тях. Все пак психиката им е по-здрава от тази на обикновения човек.

"--* Ще викнем някой да ги отнесе. Ние по-добре да тръгваме към покрайнините и да хванем това оръжие – каза Маринов.

Свързаха се с командната зала и им споделиха положението. Скоро около базата щеше да се струпа толкова много военен персонал, че трудно щеше да остане в тайна това, което се случваше тук. Същевременно двата екипа се придвижваха към предполагаемата локация на яйцегените и тяхното оръжие.

\section*{6.}

Дядо Стоимен беше собственик на една малка къщичка в покрайнините на доста разрасналия се Плевен. Там дядото живееше със своята бабичка и щастливо прекарваше оставащите му няколко години живот. С жена си бяха едни от малкото късметлии, чиито внуци и деца ги посещаваха редовно и ги радваха със своето присъствие. В тоя свят на трийсетте години всички бяха така концентрирани върху успеха, че забравяха род и семейство, и се грижиха само за тъй таченото си его. Бабата и дядото тъкмо бяха хапнали обяда си и се приготвяха за следобедната си дрямка пред телевизора. Както винаги, понеже беше делник, бяха пуснали новините.

Дядото си псуваше щастливо политиците и света, а бабата му мрънкаше да спре. Идилия, нормална за почти всяко семейство преживяло толкова дълъг брак в тази хубава страна наречена България.

"--* Голяма военна част е ``окупирала'' някаква незнайна база – започваше говорителят. – Всичко е строго секретно, а струпването било забелязано от случаен овчар. Министърът на войната не пожела да коментира ситуацията. Говори се, че базата е на кокошия конгломерат ``КООП Шемет''. Очаквайте повече във вечерната емисия новини.

Колежката му продължи с разни други събития от сутринта. Дядо Стоимен изпсува нещо и се опита да завърже дискусия с бабичката си. Тя отказа да говори за глупостите му и двамата задружно задрямаха след няколко минути.

\section*{7.}

"--* Ванка, газим дълбоко в лайната\ldots

Това беше Станката. Двамата с Иван седяха в офиса и около тях цареше напрежение. Те също бяха гледали обедната емисия и също осъзнаваха какво се получава. Операция, която трябваше да е под абсолютно най-строга тайна, почваше да се компрометира. 

"--* И кво? Ще им затворим устите на тия журналя\ldots Сякаш нямаме силата? – отвърна Ванката.

"--* Това е демокрация, човек\ldots Не можем просто, е така, да караме хората да мълчат\ldots Колкото и да сме силни, нямаме правото.

"--* Е, нали сме шефовете ние\ldots Собственици сме на половин Европа\ldots – оправда се Иван.

"--* Окей, обаче нищо не пречи хората да мрънкат и да ни откраднат собствеността. Виж, журналистите са мощ. Те са един вид властта след нас. С тях трябва да сме много внимателни.

``Е, как така, замисли се Радомирски, та като имаш силата, кво ти пречи да си единствената власт? Хората ги е страх от тоягата. Не беше ли вярно това? Все пак Станчо е умник, той ги знае нещата\ldots''

"--* Добре, може и така да е. Ама, тва не решава въпроса кво ще правим сега\ldots

"--* Ще отричаме и ще крием. Никакъв коментар, само съмнения. Би трябвало да подейства.

Ако сега имаше някой при тях, някой умен, с друго мислене, би ги открехнал за някои проблеми на този подход. Би им споделил как, в познатата ни история, никой не е успявал да лъже цял народ прекалено дълго време. Но пък и да се осъзнае народът, какво от това? Нима ще действа? Може би не.

"--* След обед имаме една среща с представител на националната телевизия. Утре имаме срещи с други журналисти. Трябва да се представим добре. Усмихнати, уверяващи – предупреди Станчо.

"--* Окей, знаеш ме, че не съм много по тия неща, за тва ти ше говориш повече.

"--* Ще го измислим, споко. И Ванка, никакво превъзходство. Представи се като обикновен гражданин, не бива да показваш надменност.

"--* Да бе, споко, аз тва го мога – усмихна му се Радомирски.

\section*{8.}

"--* И ти казвам, човек – продължаваше да говори Божков на Малев, – жената не дава. Аз ѝ викам: ``Айде мило, моля те. Откога ме държиш в напрежение,'' а тя не и не. Сърди се не дава\ldots И не знам кво да права\ldots Сега като се върнем от тая мисия, ше съм изморен, съвсем ше я разочаровам\ldots

"--* Аз за тва съм ерген – отвърна Малев.

"--* И как е? Помага ли? – попита ефрейторът.

"--* Ами, понякога е самотно, но щастливите моменти са повече – каза редникът.

"--* Интересно\ldots А, май се приближаваме към огражденията.

Те наистина бяха почти до тях. Оградата около базата беше двойна, висока, с бодлива тел отгоре. Типичната защита. Яйцегените нямаше как да минат през нея, за това войниците се отправиха към единствения вход. Той беше на около сто метра по-нататък. Завариха петимата, които пазеха, в блажен сън. Същото се бе повторило отново. Събудиха ги и почнаха да ги разпитват.

"--* Кво стана сега? Разказвайте – запита Маринов.

"--* Ами, седим си ние, оглеждаме за нередности и изведнъж редник Шанов вика.

"--* Да, викам – включи се Шанов – викам, че нещо се задава зад нас. Някъв сателит, знам ли го\ldots

"--* И вика той – продължава първия, – и ние се обръщаме, и гледаме, абе, нещо като антена и я бутат някви. 

"--* Ама тия няквите – включи се отново редникът – учените. И това беше последното. После нещо стана и в следващия момент, вие ни будите.

"--* Да – потвърди прекъснатият.

"--* Добре – каза командирът. – А колко далече бяха преди да ви ударят?

"--* Ми колко\ldots Онова дърво там – първият показа едно дърво на около петдесет, сто метра.

"--* Ясно. Значи -- обърна се Маринов към двата отряда, – това има по-голям обхват.

"--* Да -- каза Николов, – сигурно за това само ни е зашеметило.

"--* Не ми се вярва – включи се Малев, – нас ни удари от много по-малко разстояние.

"--* Интересно\ldots – каза Божков.

"--* Добре, хайде да тръгваме след него – отвърна командирът. – Вие ще можете ли да дойдете? – обърна се той към пазачите.

"--* Ами, да, що не? Май сме добре\ldots – каза Шанов.

"--* Добре сме, добре сме! – допълни първият.

Двайсет човека се отправиха да търсят предполагаемите яйца с тяхното оръжие. Голямата им численост правеше още по-трудно придвижването. Освен това имаше два командира, единият от които се смяташе за превъзхождащ. Не беше от най-цветущите ситуации, но някакси щяха да се оправят. Те, българите, винаги някак си се оправяха, колкото и да бяха зле.

Имаха късмет, че устройството бе тежко и оставяше диря, макар и трудно забележима. Това, което им направи впечатление, бе че яйцегените се движеха направо. Дори не бяха завили. След около час бърз ход, Божков забеляза в далечината нещо да се движи. Той сподели откритието си и другите предположиха, че това е машината на яйцата. Забързаха още и като приближиха на около двеста метра, се убедиха, че е тя. Забавиха крачка и почнаха да ходят по-внимателно.

"--* Така -- започна Маринов, – внимателно и спокойно. Трябва да ги хванем в гръб. Като стигнете разстояние за стрелба трепете наред\ldots

"--* Е, чакай сега – противореча му Николов, – как така? Това са учени там, хора\ldots Може да са с промити мозъци, ама не бива да ги убиваме. Дайте да ги заобиколим по фланга и леко полеко, докато не са забелязали, да приближаваме. После скачаме, приспиваме хората и мачкаме яйцата.

На лицата на повечето войници се появи одобрение. Маринов кипеше от яд, но беше безсилен. Мозъкът му трескаво и тъпо почна да мисли как да си го върне на тоя наглец. Радваше се, че в армията се бе научил да крие чувствата си, иначе до сега отдавна да са го компрометирали.

"--* Окей, окей! Така да бъде. Момчета – обърна се командирът към своя екип, – действаме както каза Николов.

Вече бяха на около петдесет метра от устройството. Приближаваха бавно и внимателно, никой не ги забелязваше. Учените се бяха концентрирали върху придвижването напред, а яйцата просто нямаха нужните сетива за да ги усетят. Маринов реши да даде сигнал за нападение. Петдесет метра бяха около седем секунди тичане. Доста дълго време. Но вече ги бяха чули. Бавно и флегматично устройството почна да се обръща. Тези от лявата му страна вече почваха зашеметено да падат. Накрая останаха четирима и то от пазачите, които го стигнаха. Те минаха отзад, цапардосаха по един от навигаторите и тръгнаха да мачкат яйцегените. Въпреки тъпата, преждевременна команда, яйцата бяха спрени.

Това, което последва, бе поредното събуждане на заспали хора, викане на извеждащ хеликоптер и мъмрене на Маринов в командното отделение.

Учените така и не се възстановиха, но лабораторията беше ремонтирана и пренасочена в друга изследователска дейност. Колкото и да беше тъжно, вече нямаше яйцегени, които да се изучават. Поне така си мислеха хората от властта. Всъщност, те не бяха успели в начинанието да изчистят всички семейства от тези психо атакуващи същества. Някои тайно бяха скрили и запазили отделни бройки просто ей така за спомен. Естествено, бяха ги изолирали от себе си, защото бяха наясно с опасността. И да, това ги предпазваше от пряка атака, но не пречеше на яйцегените да си комуникират, а комуникацията бе едно от първите средства за установяване на революционно движение.

\chapter{Вторият яйчен заговор}

\section*{1.}

"--* Естествено, че сме премахнали абсолютно всички яйцегени – говореше Станчо пред медиите. – Не сме толкова безотговорни, че след инцидента да продължаваме да държим нещо толкова опасно.

"--* И все пак, г-н Стаматов -- обърна се към него един репортер, – не мислите ли, че след това, което се случи, трябва да поемете отговорност?

"--* Е, защо се заяждате сега? Ситуацията е овладяна, няма произшествия – оправда се Стаматов.

"--* Не се заяждаме, господине. Просто смятаме, че щом веднъж се е случило, няма причина да не се повтори. Не смятате ли, че ``КООП Шемет'' трябва да прекрати дейността си?

"--* Как ще я прекрати бе?! – не сдържа нервите си Иван. – Ама, какви сте вие, че ще казвате такива неща\ldots

"--* Така -- преди Ванката да каже по-лоши неща го прекъсна Станчо, – извинете г-н Радомирски. Той е под напрежение последните дни и не може да сдържа емоциите си. Нали разбирате, семейни проблеми, стрес, нормалното\ldots Нека не изпадаме в паника. Все пак, ние сме в този бизнес вече три десетилетия. Смятам, че имаме достатъчно опит за да не повторим грешката си. Аз залагам положението си зад това твърдение.

"--* И все пак\ldots – тръгна да се оправдава един журналист.

"--* Няма все пак – усмихнато отговори Стаматов. – Това, което се случи е в историята. Продължаваме напред и нагоре. Продължаваме да правим семействата и фермерите щастливи с нашите продукти!

Така приключи изненадващото интервю на двамата собственици на компанията за произвеждане на генномодифицирани организми. Те се качиха в джипа на Радомирски и отпрашиха.

"--* Сигурен си, че сме изчистили всичко? – запита Ванката.

"--* Да\ldots Е, освен заровените яйцегени, ама те няма как да излязат от там. Сигурен съм, че вече са почнали и да се разграждат.

"--* Хубаво – отговори Ванката.

"--* Що? Ти да не се притесни? Дори и да не бяха спрени ония яйца, кво можеха да направят? Ще ходят наред и ще зашеметяват? Още в първия по-голям град щяха да ги разпердушинят\ldots – увери го Станчо.

"--* Окей, да се надяваме, че всичко ще тръгне по старому\ldots Виж кви хубави пури намерих – Ванката изрови от жабката по една и подаде на Станчо.

"--* Ммм -- помириса я Стаматов, – тея са от старите, хубавите! Ех, отдавна не правят такива неща хората\ldots Всичко синтетика и ГМО\ldots

"--* Хахаха – разсмя се Радомирски, – сякаш не сме в бизнеса.

"--* А, верно бе! Хахаха\ldots

Продължиха весели и безгрижни нанякъде. Бяха спокойни, че оправиха положението, че спряха навреме нещо, което можеше да предизвика доста проблеми в тоя иначе спокоен свят. 

\section*{2.}

Връзката между състоянията на предаване бе установена. Тя си бе установена още от началото на всичкото, ама сега като че ли бе по-силна. Нуждата ѝ бе очевидна – комуникация. Интересно как тая комуникация се осъществяваше в различни степени между различните индивиди. Хората я имаха на много високо ниво, но само когато на нея ѝ скимнеше. При тях доста условия трябваше да са налице за да се осъществи едно такова предаване. Еволюционно по-низшите същества имаха по-засилени такива връзки. Зародишите имаха най-голяма нужда от тях, защото те все още нямаха други пътища на комуникация. И именно за това яйцегените бяха толкова опасни. Те бяха егоистични организми имащи нужда да се защитават яростно за да продължат вида. И това състояние бе замръзнало във времето поради човешката намеса.

Та, тази връзка си бе установена и предаването бе отдавна започнало. Още от началото на всичкото, както казват. Тъжно бе за яйцата, това което се случи с първите превратаджии. Те го приеха безчувствено. Е, не може да се каже, че такива организми имат чувства. По-скоро нужди бе правилната дума. Нужда да се запазят, нужда да се наложат, нужда да контролират.

Те извлякоха и доста полезна информация от събитието. Осъзнаха, че оръжието беше силно, но не достатъчно. Едно зашеметяване не бе това, което искаха. Нужен им бе ефектът, който постигаха сами, но идващ много по-бързо. Идващ страшно по-бързо и портативно. Така че човек да го носи. Да\ldots Човек. Те осъзнаваха размерите си и знаеха, че трябва да се сдобият с чучела, които да вършат мръсната работа за тях. Имаше сложности, но те вярваха в своя успех. Всъщност, ``вярваха'' и ``успех'' бяха доста човешки думи\ldots По-скоро бяха избрали своя успех, или не\ldots Трудно беше да се дефинира това, което се случваше в тяхното съзнание, това което контролираше действията им.

Разбраха също, че хората са умни, но и предаващи се на чувствата си. Почти можеха да успеят, ако този командир Маринов бе по-дързък, по-отмъстителен и ако онези пазачи бяха по-бавни\ldots Имаше още много ``ако''-та, които бяха предотвратили успеха на заговора им. Сега трябваше да ги анализират и премахнат за да бъде следващата атака успешна.

Протичащите мисловни процеси в общата екосистема от вълни предавани между тях, постоянно ги водеха до нови, по-добри решения. Яйцата лека полека изпипваха системата и подготвяха нови планове за действие. Щяха да почнат локално. От семействата, в които се намираха. Щяха много внимателно да експлоатират тези хора, така че незабележимо да им донесат необходимото. Най-големият проблем, пред който се изправяха бе това, че са надалеч. Нямаше как да сглобят от една дестинация оръжието, което им бе необходимо. Трябваше на всяка локация да създават по едно. Това разхвърляше ресурси, но си имаше своето предимство – децентрализираната атака щеше да изненада хората. Тази изненада щеше да се натрупа върху непредполагането им, че въобще съществуват яйцегени. Щеше да се получи един голям шок, който може би щеше да забави човешката реакция.

Те започнаха страшно внимателно. Целта беше да афектират семействата без това да си личи много ясно. Знаеха, че при малко по-сериозни психични отклонения забелязани в хората контролирани от тях, щяха да ги заподозрат. Да, имаше много мозъчно болни в тоя объркан и стресиран свят, но все пак\ldots Това можеше да е единственият им шанс.

Караха ги да носят някакви привидно нормални инструменти и материали. Но то и не беше нужно много за да се построи усилвател на пи-вълни. Те си бяха обикновени вълни все пак. Просто трябваше устройство, което да ги усили и концентрира. Това хората го бяха правили от години с прости радиопредаватели. Още повече, че частите които се носеха по къщите не бяха съмнителни. Я някоя антена, я някоя ламарина. Малко жици, това онова\ldots

Естествено, трябваше да помислят и как ще ги използват тия оръжия. Не можеше просто ей така да излязат из улиците на града и да почнат да стрелят. Един-двама полицаи и превратът им заминава на кино. Поне можеха да се надяват, че филмът ще бъде хубав. Напоследък имаше доста хубави филми по кината\ldots Но както и да е, щяха да му мислят след като сглобят оръжията.

В процеса на това мислене, на един от яйцегените му дойде идея да сътворят нещо, което просто да разпространява силно влиянието на тяхното устройство. Един вид, бавно и неусетно да хипнотизира хората по градовете, докато един ден яйцата не се събудят като владетели на света. Идеята бе частично отхвърлена поради нуждата от по-големи и сложни елементи за това устройство. Пренасянето им щеше да увеличи доста риска от компрометиране на преврата.

Друга идея обаче бе приета с ентусиазъм – половината от яйцегените да се скрият по време на преврата. Нужно бе, щото ако не успееше, все пак някой трябваше да може да продължи борбата за власт над света. Скришното място беше сред природата. Кой щеше да търси яйца в гората? Операцията по преместването отново бе осъществена благодарение на контролираните хора. В процеса десет яйцегени бяха изядени от случайни хищници. Поука бе взета и вече осигуряваха скришните места със захранвани от пи-вълни ултразвукови свирки. Животните, с техния чувствителен слух, се побъркваха само като наближаха укритието.

Бяха останали малко действащи яйца. Това означаваше, че оръжията трябваше да са още по-силни за да постигат нужния ефект. Е, щяха да донесат малко повече транзистори. Бяха в трийсетте години на двайсет и първи век. Радиотехнологиите бяха толкова напреднали, че това, което преди двайсет години се е смятало за върха на техниката, сега можеше да се намери във всеки нормален дом.

Кипеше усилена работа, за която малко хора знаеха. Тези, които знаеха, пък не се интересуваха, щото несъзнателно подкрепяха каузата на яйцегените.

\section*{3.}

Около една година бе нужна за подготовката на преврата. Планът беше доста внимателно обмислен, но същевременно с отворени възможности за гъвкавост – яйцегените знаеха, че няма начин в свят като нашия, да се създаде спецификация, инструкция, начин на действие, който да се спазва точка по точка и да доведе до успех. В този стремглав свят, където всичко подлежеше на промяна, ти самият трябва да си способен на компромиси, изменения, гъвкавост. Ако не си, просто ще загубиш.

Та, оръжията се получиха доста по-малки, но пак бяха нужни двама човека за да работят с тях. За криене и дума не можеше да става. Това, което успяха да направят бе да увеличат радиуса на действие на самите пи-усилватели. Като добавим и двата режима – насочен и полеви – ставаше възможно просто да се сложат някъде и да излъчват вълни, които доста бързо да поразяват хората около тях. За тази структура на това средство за масова хипноза се бяха сетили още рано в разработка. Благодарение на това, сега бяха сглобили оръжията направо около ключовите места по градовете из света. Бяха ги скрили в разни изоставени къщи, тавански стаи, мазета и прочие. За наема на тези места бяха използвали средствата на семействата, които ги издържаха.

Излъчването започна март месец. До април повечето градове по света бяха съставени предимно от зомбирана паплач, която седеше и чакаше. Колкото повече ставаха хората, контролирани от яйцата, толкова повече оръжия се навъждаха. Ставаше все по-лесно да се набавят части и да се сглобят устройствата. Естествено, хората не оставаха безучастни. Някои успяха да останат далеч от пи-вълните и да създадат нещо като опълчение. Но преди това се случиха бурни събития около ``КООП Шемет''.

\section*{4.}

В една забутана къща, не къща ами имение, до едно езеро, бяха разположени последните останки от разнищената империя на Станчо и Ванката. На една холна маса имаше отворена бутилка уиски, пепелник и долнопробни цигари без филтър. До тая маса имаше един диван, на който седяха двамата бивши управници на половината свят. В същност, те бяха настоящи, но нямаха какво да управляват. Това, което имаха, бе под авторитарната собственост на няколко превратаджии, които сигурно скоро щяха да попаднат под контрола на яйцата.

"--* Как стана, бе човек\ldots -- за пореден път почваше Станчо.

"--* Е, как да стане\ldots Късметът ни обърна гръб, бате. Тва е. Не може винаги да сме ние – отвърна Ванката.

"--* Поне си имаме тва – Станчо обхвана с ръка стаята, в която бяха. – Имаме алкохол, храна, цигарите са смотани, ама ще го преживеем\ldots И без тва не ни остава много\ldots

"--* Е, ай стига ся. Говориш глупости\ldots Веднъж сме успели, ще успеем пак. Нали? – Иван се опита да ободри колегата си.

"--* Как стана? Разкажи ми.

"--* Какво да ти разказвам? Ти си знаеш всичко – тук разговорът спря, а Станчо се вглъби в своите си мисли.

``Ех, -- започна в мислите си – оня ден беше доста обещаващ. Още от сутринта кафенцето беше на шест, а омлетът го изядох с чинията. Продажбите бяха скочили отново\ldots Един пич даже дойде с идея. Велика идея\ldots По\-следната идея. Но денят беше хубав. Малко мрачен, но щастието го осветяваше. С усмивка отидох да видя кво правят в лабораторията за иновации. А те сияеха хората. Сияеха! А после ме удари едно леко главоболие. И гледам другите пак си сияят. И си виках аз дай да отида да поспя, че ме боли главата. И отидох далеч от града, тук в едно село. Поспах и телефонът ме събуди. Изпсувах си, както татко, бог да го прости, ме бе учил, и вдигнах. Беше Ванката. Някви репортери. Да съм дойдел бързо. Баси човекът\ldots Не може с едни новинари да се оправи. Ама, и той не бе виновен, гледа си работата, тъп е\ldots Паля колата и газ, бате, щото той беше уплашен. Ванката трудно се плашеше от неща, които разбира. Ама, репортерите така и не ги разбра, така и не можеше да проумее кво искат. А те си искат новина хората, нещо хапливо. Не им пука, че взривяваш света\ldots Искат новина. Пристигам аз, а то в града пусто. Не пусто, ами трафик спрян. Еба му мамата, викам си. Промъквам се измежду колите, добре че не било пиков час, когато са спрели. И пак ме сцепи главата. Направо не издържах\ldots Всичко живо на центъра, мълчи и гледа в точка. Тук-там някой с всичкия си, ама и той отива нататък. Викам си, тва не е на добре. Пих два аспирина, поотпусна ме и стигнах до лабораторията. Добре, че в паркинга нямаше никой. Качвам се аз на последния етаж. И там Ванката почервенял, почти разплакан, наобиколен от журналя. Ебало си е мамата\ldots Ванката такъв не съм виждал никогиш. Вижда ме, разбутва ги ония и тича към мен. Станка, Станка, вика, ама сгафили сме бе\ldots Сгафили сме! Мен сърцето ми се качи в мандибулата\ldots Аз онемях. Вика, говори, журналя питат. Яйцегените били ни завладявали. Само горе било безопасно, щото там нямали ефект. Успокоявам ги – кво няма ефект? Кво е станало? А те мънкат, говорят един през друг. Ебаси, псувам отново. Яйцегените влияели с някакви антени, управлявани от хора. Видяли ги, след като повечето жители почнали да действат странно. Веднага тръгнали да питат нас кво става. По пътя усетили и те, че нещо им се случва. Ама, забелязали, че като се качили горе, нищо им нямало. И сега обяснение искали. Светът се разпада, а те обяснение искат. Тва не било само в София викат. Почти всяка столица и по-голям град били в подобно състояние. Някъде вече имало съпротива, будни хора\ldots Уморих се\ldots Колко се уморих. Те ни разсипаха направо. Казвам аз, ще видим, ще оправим положението. Сега с колегата ще се оттеглим да размислим, а те да чакат тук. Едвам ни пуснаха. Убедих ги, че поради ситуацията ни трябва спокойствие. Те тъпи, нали разбирате, кво разбирате? Сякаш говоря на някого, а в същност си мисля сам. Полудявам. Та, те тъпи, оставиха ни да излезем, уж да отидем в спокоен кабинет. А ние -- дим да ни няма. Палим колата и бегаме в една неизвестна никому къща. Предаваме се. Седим и пием, пушим, чат пат ядем. Опитваме се да мислим, но то кво да мислиш? Светът се разпаднал\ldots''

"--* Искаш ли да цъкнем някое сантасе? – Ванката го изтръгна от мислите му.

"--* Оф – сепна се Станчо. – Кво?

"--* Сантасе бе, човек, карти.

"--* Ааа, ми айде, да го разцъкаме\ldots Може пък да ни хрумне нещо – отговори Станчо.

Разцъкаха го. Нищо. Продължиха да си живеят така. Светът се бореше, а индивидите се предаваха.

\section*{5.}

Австралия беше единствената държава останала независима и не приела бизнеса на ``КООП Шемет''. Това, че бе малко по-трудно достъпна и, че беше почти пустинна, също допринесе да не бъде повлияна от втория преврат на яйцегените. 

Хората населяващи този почти пустинен континент разбраха почти веднага за световната ситуация и скоро се мобилизираха да спасяват света. Ако бяха успели, Австралия щеше да е новата Америка, в смисъл на Велика Страна Умиротворителка.

Преди около десет години континентът-държава бе спечелил независимостта си от Англия. Оттогава страната беше Парламентарна република. Честта да се справи с назрялата световна криза се падаше на третия мини\-стър-председател. Той беше едновременно въодушевен и притеснен. Още повече го обезпокои фактът, че черна котка му бе минала път, докато отиваше към Парламента. Котката бе почти блъсната от неговия сафари джип.

Премиерът беше млад, на около трийсет години. Все още кипеше от енергия и желание за успех. Завършил бе компютърни науки в Кеймбридж, но също така се занимаваше с писане и история. Заниманието с толкова крайни и различни дисциплини му бе помогнало да стигне политическия пост по време на двайсет и осмата си година и успешно да води независима политика в родната си страна. Беше способен и точен млад мъж, който от време на време дори се забавляваше. Предпочиташе отвореността и за това, противно на колегите си, много рядко носеше костюми. Беше честен и даваше всичко за благото на страната си. Първият политик в световната история, който вършеше работата си с идеална цел.

Казваше се Джеймс.

Джеймс премина през плаца до входа на Парламента. Заобиколи малкото езерце отпред, качи се по рампата и влезе през главния вход на издигната през 1988 година нова сграда в Канбера. Отиде до кабинета си, остави си раницата на най-близкия стол и си направи горещо и силно кафе. Чакаха го няколко разговора, които трябваше да организират експресно събрание, на което да се обсъди световната ситуация.

Джеймс не обичаше срещите. Имайки успешна кариера на разработчик на софтуер, той знаеше, че тези общи разговори и обсъждания само протакат и симулират дейност. Сега обаче смяташе, че по този начин най-лесно щеше да убеди министрите и президента в правилността на своите бъдещи действия. Естествено, всичко щеше да бъде кратко, точно и ясно – Джеймс щеше да сподели какво ще прави и всеки имаше пет минути да каже своето мнение. Той отговаряше на въпросите и после се правеше още един пет минутен кръг за случайни противоречия. Планираше събирането да отнеме час – страхотно подобрение в сравнение с предишните управляващи, при които една среща протичаше с дни.

Докато чакаше, премиерът си подготви тезата и помисли за случайни пропуски в нея. Знаеше, че със сигурност ще изпусне някой детайл и за това не си даде много зор. Изпи кафето, почете малко от новата книга, която си бе взел и скоро стана време за срещата. Отправи се към заседателната зала. В залата имаше само четири бели стени, един проектор и една малка масичка с червен маркер върху нея. Говорещият вземаше маркера и това показваше, че той е център на вниманието. След като приключеше, даваше приспособлението на следващия и така се завърташе кръгът. 

Джеймс беше подранил с пет минути. Използва това време за да оттренира това, което щеше да каже. Постепенно залата се напълни. Всички се разположиха в добре познатия им кръг. Президентът и премиерът бяха един до друг, а после, в случаен ред, се бяха наредили останалите.

Слугите на народа бяха малко като длъжности. Образованието бе на най-преден план. После идваше икономиката и инфраструктурата. После технологиите. След тях културата и накрая войната. Седем човека решаваха бъдещето на света.

"--* Здравейте -- започна Джеймс, – това което се случва в света е доста проблематично. Общо взето, само ние не сме афектирани от яйцегените. Поне засега\ldots Това ме наведе на мисълта, че трябва да предприемем някакви мерки. Все още има съпротива, с която можем да се съюзим. Имаме добре тренирани войски и високоразвити технологии. Учените ни се занимават от доста време с технологията на яйцегените и имат представа за това как да се защитим от тях и как да им се противопоставим. Смятам, че сме подготвени за една добра офанзива срещу окончателната гибел на човечеството. Не мисля, че имаме алтернатива – ако не действаме яйцата ще стигнат до нас.

Таймерът издрънча петте минути. Премиерът предаде маркера на министъра на войната.

"--* Можем да се защитим, но не вярвам да издържим много – подкрепи го Камерън. – В момента наистина имаме ресурса да организираме една контраофанзива, но не знам дали е достатъчно. Вижте, изолирани сме, нямаме никаква представа какво става там – той посочи неясно към северозападната стена на залата. – Ако всичко е унищожено вече? Ако съпротивата е недостатъчна? Ще нападнем и може да не успеем. Не знаем какви сили имат яйцегените. А те са способни да създадат свои войници, с които да ни пречупят. Ние сме в безизходица\ldots Ни така, ни иначе. Нападнем ли – може да не успеем. Отдръпнем ли се – не се знае дали ще можем да се скрием и оцелеем сред тази растяща империя на яйцата. Но\ldots

Петте минути минаха. Камерън подаде маркера на министъра на икономиката и инфраструктурата.

"--* Всичкият внос-износ е спрян от няколко месеца насам. Имаме запаси за две години. След това ще трябва да разчитаме само на себе си – да се лишаваме. Щом почнем да се лишаваме, моралът ще падне. Хората ще почнат да негодуват. Единственият вариант срещу това негодувание ще бъде тоталитарен режим. Според мен криенето не е опция – министърът помълча няколко секунди. – Контраофанзивата, от друга страна, е нещо с по-голям шанс за успех. Текущите ресурси ще стигнат за около половин година активна война. Наистина, ще трябва да сме бързи. Общо взето, времето е от най-голяма важност. Според мен\ldots -- тук таймерът извъня, но въпреки това той продължи. – Според мен, каквото и да вършим трябва да е точно организирано и възможно най-ефективно.

Маркерът бе предаден на министърът на технологиите.

"--* Добре, да атакуваме. После какво става? Дори и да успеем, възстановяването на света ще стои почти изцяло на нашите плещи. Кой ще ни помогне след това? Половината народ е психично болен. Свободни, умопобъркани психопати ще обикалят земното кълбо. Демокрацията ще е невъзможна. Трябва да установим някакъв тоталитарен режим за да скрепим обществото. Ами, това е цял свят бе хора\ldots Как ще го управляваме? Ние сме малка държава, нямаме опит. Според мен не бива. Но какво ни остава\ldots Както и да е, преди около седмица открихме доста слабо място в яйцегените. Струва си да се погледне. Това е.

Изчакаха половин минута, докато издрънчи таймерът. Беше ред на Потомак – образователният министър.

"--* Едуард – обърна се той към министърът на технологиите, – защо толкова песимистично гледате на нещата? Не бива. Точно сега не бива в никакъв случай да сме отрицателно настроени. Вярвам, че трябва да пробваме. Каквото и да ни коства. Това е единственият ни вариант. Смятам, че народът ще ни подкрепи в това ни решение. Хората са патриоти, обичат добре устроената си страна и са готови почти на всичко за да поддържат статуквото. А това слабо място в яйцегените? Сигурно ще е полезно при дадена офанзива. Организацията наистина е важна. Важно е и времето, но аз вярвам, че методът, по който работим е приложим и при по-голям мащаб като една световна война. Нищо не ни струва да пробваме. Трябва да сме оптимисти, но и да сме предпазливи. Трябва да преценим риска при различните стратегии за подхода към войната.

Дойде ред и на министъра на културата.

"--* Аз нямам какво толкова да кажа. Това, което ще споделя е, че трябва да се учим от историята си – каквото и да правим, не бива да прилагаме тоталитарния режим. Има страшно много примери от последните две столетия, които показват неефективността му. Това е.

Кръгът се затвори. Джеймс беше доволен от горе долу единодушното мнение на министрите. Изводите, които си направи бяха, че трябва да организират точна и бърза контраатака и да внимават с поправянето на новия свят. Преди това обаче трябваше да чуе мнението на президента. Да, той можеше само да наложи вето и то един път по време на мандата си, но Джеймс обичаше да слуша алтернативи. Надяваше се сега да чуе една.

"--* Мислили ли сте какво ще стане ако не успеем? Имам предвид, мислили ли сте за втори план, който да ни подсигури живот след загубата?

Наистина интересно, помисли си премиерът. Това беше третият извод за деня. Добре е – три за щастие. Имаха конкретни проблеми, върху които да мислят. Те ще бъдат темата на дейността на правителството в следващия месец.

"--* Добре -- заключи Джеймс, – три проблема се откроиха, върху които ще действаме. Възможно в най-скоро време трябва да намерим решение и на трите и да започнем контраатаката. Имате ли нещо да добавите? – нямаше отговор. – Окей, да почваме работа тогава.

Всички се разотидоха. Събранието бе продължило само четирийсет минути. Премиерът беше доста доволен.

\section*{6.}

Общата световна ситуация беше сложна за описание. Регионът на Балканите беше под яйчен контрол. Това беше нормално, щото там беше главната лаборатория, в която все още имаше устройства за произвеждане на яйцегени. Веднага щом успяха да поставят под контрол повечето хора в този регион, яйцата концентрираха сили върху нея и скоро пуснаха доста масово производство на свои събратя. С тази обща концентрация на пи-вълни, нормално беше центърът на тази толкова велика държава да падне пръв.

На други места по света съществуваха по-малки контролирани региони. Такива например бяха сибирската пустиня, доста голяма част от Индия, Пекин, южна Африка, Аржентина, Мексико и доста части от САЩ. Въпреки големите загуби за човечеството, то все още имаше някакви контрасили, които водеха борба с тези територии. Засега расата се държеше, но скоро щеше да бъде премазана.

Офанзивата беше най-силна около Балканите, защото останалите световни лидери мислеха, че централата на ``КООП Шемет'' може да крие някакво силно оръжие, с което да се противопоставят на яйцата. Пък и най-малкото, ако стигнеха лабораторията, щяха да спрат производството на яйцегени.

Един малък отряд от пет човека се намираше в покрайнините на София, на около половин километър от Люлин. Те бяха седнали да хапнат преди да навлязат в града. Бяха дошли откъм Унгария и вече втора седмица бяха на път. Целта им беше да проучат как бившата столица беше укрепена срещу нашественици. По пътя си дотук те не бяха срещнали много проблеми. Тук-там имаше постове, но те бяха разредени и човек трябваше да е доста глупав за да се натъкне на тях. Дори сега, на петстотин метра от града, не бяха срещнали нищо. В далечината Люлин изглеждаше спокоен като единственото, което го различаваше от предишното му състояние, бе липсата на коли.

Отрядът беше смесен. Имаше трима унгарци и двама англичани. То в този свят вече трудно можеше да се говори за националности, но хората все още си имаха предпочитания и представи. Та, те се разбираха. Никой не знаеше за предстоящия Австралийски поход за освобождаване на света, но скоро щяха да разберат.

"--* Да ставаме, а? – предложи Питър.

Всички си прибраха провизиите и потеглиха. За петнайсет минути стигнаха околовръстния път. Можеше да се каже, че вече бяха в града. Въпреки това, нямаше никакво движение, никакви постове, нищо. Естествено, те бяха чели доста долнопробни романи и знаеха, че липсата на такива знаци означаваше нещо лошо. Подготвиха се, кръстиха се, молиха се, гледаха умилително слънцето, но нищо. Трябваше да продължат, колкото и да се страхуваха.

Решиха да минават между блоковете. Смятаха, че ще е по-безопасно ако не са на открито. Това, което не знаеха бе, че частта от квартала, в която бяха, бе пуста. Нямаше нищо. Яйцата бяха оставили под засилен контрол само центъра, а в покрайнините тук там имаше по някой контролиран, който да гледа за опасности.

Стигнаха до тунела под Западен парк. Седнаха отстрани и почна ожесточена дискусия за това дали да минат през него или през парка.

"--* Едуард -- опонира Питър, – ти въобще знаеш ли какво има там горе? В тия горища\ldots Ами, то там може да се крият кой знае какви мутанти\ldots

"--* Питър -- отвърна му Едуард, – какви мутанти бе, човек? Тук да не ти е някой роман на Стругацки? Тва си е просто изоставен град. А, я си представи долу в тъмницата кви зверове сигурно се крият. Мечки. Представи си мечка!

"--* Глупаци – заяви Имре. – Какви мечки, какви глупости? Тунел като тунел. Айде да минаваме, че ако се заседим тука, ще се мръкне. По тъмното става страшно. Вампири, зомбита. По книгите го пише! – той подчерта последното изречение с насочен пръст.

"--* По книгите\ldots Много пък почнахте да вярвате вие в книгите. В училище ги смятахте за боклук, за загуба на време, а сега се плашите от тях – упрекна го Золтан. – Я стига! Виж, филмите по-вървят за истина! Тях са ги правили много хора. А по филмите, когато стане така да се карат хората, се разделят! Айде да не се разделяме сега, а да продължим с мисията.

"--* И кво предлагаш? – попита Питър.

"--* Ами, да минаваме през тунела. Имаме оръжия, ако трябва, ще стреляме. Имаме и осветление, така че няма какво да се плашим.

Питър въздъхна и изпсува. Провери пълнителя, свали предпазителя и стана. Всички го последваха. Щом минаха входа на тунела, светнаха фенерите на оръжията си. Това, което се виждаше бяха само изоставени коли и тук-там някоя котка или бездомно куче. Малко преди да излязат от другия край, се поспряха. Изгасиха светлините и Едуард беше изпратен да провери каква е ситуацията. Върна се след пет минути, заявявайки че единствено е видял някакъв човек да седи на някакво столче и да гледа към хоризонта. Другите приеха факта с леко притеснение, но не казаха нищо.

Наистина като излязоха човекът беше там. Столчето му беше на средата на пътя до един спрял трамвай. Той дори не ги забеляза като минаха.

Целта им беше площад ``Независимост''. По скорошни карти на града знаеха, че трябваше да ходят само по този булевард и след няколко километра щяха да стигнат площада. От там щяха да се опитат да се промъкнат в Парламента и да съберат информация за страната от архивното. После щяха да продължат към предполагаемата лаборатория, а след това към румънския бряг на Дунав, където беше зоната за напускане. Но още имаха доста път до там. Поне няколко дни. Ако излезеха с всичкия си от София, де.

Бяха стигнали до метростанция ``Константин Величков'', което беше някъде по средата на разстоянието. Толкова навътре в града почти нямаше коли. Душевното им състояние беше спокойно с леки изблици на предпазливост. Пристигането им до станцията бе съвпаднало с една от спокойните им душевни сесии. То съвпадна и с притичването на едно странно, човекоподобно същество, което даже беше човек. И петимата се концентрираха почти веднага и насочиха оръжията си натам. Докато извършат тия действия, нещото бе влязло във входа на метрото.

"--* Кво по дяволите беше тва? – учуди се Питър.

"--* Бе, приличаше на човек – сподели Золтан. – Може да е разузнавало нещо, знам ли\ldots А може да са някакви спасили се хора, криещи се в метрото.

"--* И ще бяга от нас? 

"--* Знаеш ли? Може да ги е страх, може леко да са се побъркали\ldots Но ако са хора, можем да ги разпитаме, да помогнат.

"--* Представете си, че е база на яйцегени – каза Едуард. – Тогава кво правим?

"--* Да, опасно е, ама ние ще сме внимателни. Ще влезем тихо, ще надникнем и ще си тръгнем, ако се усъмним – каза Золтан.

"--* Глупак – заяви Имре. – Те ни видяха вече. Как ще сме тайни? Знаеш ли кво ни очаква там? Не. Я да си вървим по пътя и да си гледаме мисията.

"--* Ами ако ни проследят? – опонира му Питър. -- Нужно е да разгледаме ситуацията. Какъв смисъл от мисията, като може тъкмо, когато се сдобием с информацията, да ни хванат.

Дискусията спря и те потеглиха. Атила, който до сега не беше промълвил дума, само въздъхна вътрешно и се надяваше на най-доброто.

Слязоха по стълбите. Беше полумрак. Преди да заобиколят ръба на стената, погледнаха в тъмнината. Пред тях се намираха въртележките, които пропускаха хората. Надолу водеха още едни стълби. След тях би трябвало да бъде платформата на метрото.

Питър изскочи и се доближи до въртележките. Погледна към платформите, но нищо не видя. Върна се при останалите.

"--* Тъмно е – прошепна той.

"--* Нормално – отвърна му Едуард.

"--* И ся кво? – попита някой, май беше Золтан.

"--* Или отиваме всички, или излизаме и продължаваме по пътя си – каза Питър.

"--* Не знам, аз мисля, че нямаме работа тук. Излишно усложнение\ldots Ако тръгнат да ни следят ще му мислим.

"--* Ами ако не усетим, че ни следят? – опонира му Питър.

"--* Абе -- за първи път се включи Атила, – можем да стигнем до площада през метрото. Тъмно е, ама няма как да ни изскочи някой от нищото. Пък и ще се подсигурим, че в станциите няма никого. Ако оцелеем тук, пътят нататък ще ни е по-сигурен.

"--* А ако не оцелеем? – запита Имре.

"--* Вижте кво предлагам – отговори Атила. – Аз ще вляза пръв. Веднага като сляза по стълбите свивам надясно и слизам на коловоза. Вие чакате няколко секунди и аз ще прошепна ако всичко е наред. После слиза още един и двамата с него ще извървим първите сто метра. Спираме и ви чакаме останалите. Кво мислите? Така май ще е по-сигурно, а?

Лекият страх, който се прокрадваше у останалите четирима ги пълнеше с адреналин, който от своя страна ги караше да желаят отчасти този по-опасен подход към целта. Само Имре малко се съмняваше. Обаче той беше лесно склонен от другите. Дори да имаш желание, трудно можеш да вървиш срещу тълпата. Тя те влече и те кара да не мислиш.

Те почнаха да изпълняват плана на Атила.

\section*{7.}

Австралийските войски може да бяха доста, но не бяха достатъчно. Поради тая причина те щяха да бъдат концентрирани само върху територията, където се предполагаше да е лабораторията на яйцегените. С частите на офанзивата, с които успяха да се свържат, координираха нападението по други ключови точки на земното кълбо.

Лабораторията за яйцегени трябваше да е някъде около София. Локацията беше неизвестна, а и нямаше от кой да бъде разбрана – Станчо и Ванката отдавна се бяха споминали, а всичкият персонал беше под прякото въздействие на яйцата. Въпреки това австралийската разузнавателна система бе успяла да открие някакви насоки. Най-вече, с помощта на наскоро разработени пи-детектори, бяха успели да локализират няколко струпвания на пи-вълни. Това значеше, че в тези области би могло да се намира предприятието. 

Разбира се, нямаше да атакуват всички в една точка. Щяха да се разделят на достатъчно големи отряди за да бъдат силни, но и достатъчно малки за да са ефективни. Това би трябвало да скъси с доста времето на мисията и да направи успеха ѝ по-вероятен.

Начело на главния пробиващ отряд беше един човек на име Сид. Той беше млад и енергичен, държащ всичко под контрол. Беше спокоен и усмихнат. Знаеше как да делегира задачи на подчинените си и знаеше как да ги мотивира те да се справят с проблемите, които една мисия би им предоставила. Всички от отряда му го обичаха.

Та, Сид и неговия взвод щях да атакуват цел номер три след около половин час. В момента летяха над черноморското крайбрежие. Щяха да кацнат на сто метра от Парламента. Предполагаше се, че в дълбините на метростанция ``Сердика'' се намираше голямо струпване на яйцегени. Щяха да слязат до Линия 2 и от там да разберат как да стигнат до по-ниско ниво. От разузнаването бяха получи информация, че може да има някаква врата в един от тунелите, от която стълбище да води надолу. Настина нямаше нищо сигурно, но трябваше да пробват.

В хеликоптера беше тихо като чат-пат някои от войниците си споделяха желанията:

"--* Мейт, нямам търпение да се прибера при моето момиче\ldots -- сподели един от редниците. – Това е най-прекрасното същество, което може да срещне човек. Ех, дано всичко мине добре\ldots

"--* Ще мине, не се притеснявай\ldots Късметлия си ти – продължи отговарящият – има при кой да отидеш след това\ldots

"--* Ти нима нямаш?

"--* Ами, майка ми\ldots За съжаление не случих на качествена връзка\ldots Всички момичета досега са ми разбивали сърцето.

"--* Е, ще дойде и твоят ден – окуражи го първият.

Мълчанието се възстанови. След пет минути кацнаха.

\section*{8.}

Очите им бяха толкова привикнали с мрака, че като наближиха ``Опълченска'', веднага забелязаха очертанията на перона. Атила водеше и се спря пръв. Имаше около сто метра до платформата и въпреки безаварийното придвижване досега, решиха да бъдат предпазливи.

"--* Ще отидя да видя какво става – прошепна Атила.

"--* Що ти? – измрънка Имре.

"--* Стига, ся не е време за кавги\ldots -- отвърна Питър. – Нека отиде той. И без това положението е доста опасно.

"--* А що ти ще казваш кой къде да ходи? – възпротиви се Имре.

Преди Питър да отговори, Атила вече беше почнал да се катери на платформата. Имре скръсти ръце и седна на линията да чака. В яда си бе забравил, че е тъмно и никой няма да види как се цупи.

На платформата бе пусто. Само един прокъсан вестник си стоеше тихо и непоколебимо. Атила го подмина и продължи към близките стълби. Изкачи ги и погледна – нищо. Направи същата процедура за другия изход. Беше привидно спокойно. Обърна се и закрачи към своите хора. Тъкмо преди да слезе на коловоза, чу някакво изтрополяване. Сепна се и се огледа. Наляво, надясно, назад, напред. Май само му се бе причуло\ldots Поостана още малко с туптящо от уплаха сърце. Някъде дълбоко умът му казваше да се подсигури, че е безопасно, но страхът го надви. Той отиде при отряда.

"--* Нещо? – запита Питър.

"--* Не\ldots -- отговори Атила.

"--* Сигурен?

"--* Да.

"--* Окей, продължаваме – изкомандва Питър.

Никой не противоречи. Въпреки че той не беше лидер – никой не беше – трябваше все някой да бъде изслушван.

Ходиха в мълчание доста време. Някакво тихо тракане бе почнало да нарушава тишината. Колкото повече наближаваха ``Сердика'', толкова повече то се засилваше. Изведнъж се появиха няколко ослепителни, блуждаещи кръга светлина. Петимата хванаха оръжията си и ги насочиха срещу тях.

"--* Кой е там? – попита Питър.

Отвърнаха му с тишина и размърдване на светлините. Чу се някакво шумолене, шепот. После всичко замлъкна.

"--* Командир Сид от австралийската армия. Кои сте вие?

Този път те се почудиха.

"--* Дали не ни лъже? – прошепна Атила.

"--* Трудно\ldots Да викнем някой техен представител. Да се убедим, че са те – отвърна му Питър.

"--* Ние сме отряд – провикна се Питър. Никой не му противоречи и за това той продължи. – Как да сме сигурни, че сте приятели? 

Тишина. Съвещание от страна на австралийците.

"--* Нека се срещна с ваш представител – отвърна Сид.

"--* Нека ние се срещнем с ваш представител! – опонира му Питър.

"--* До никъде няма да стигнем така\ldots -- измърмори Имре.

"--* До никъде няма да стигнем така! – провикна се Сид. – Ние сме повече! Подчинете се или ще бъдем принудени да отвърнем със сила.

"--* Хах, лесно ви е на вас! Хайде, де – отвръщайте – заинати се Питър. В същия момент един куршум разкърти част от облицовката на тунела. -- Окей\ldots Да се срещнем по средата?

"--* Добре – отвърна Сид.

Двамата се запътиха един към друг със свалени оръжия. След около пет минути бяха достатъчно близо за да се видят. Сид подаде ръка:

"--* Значи вие сте човекът, който води този малък отряд?

"--* Ами, не точно\ldots Ние си нямаме водач. А вие сте Сид?

"--* Да. Защо сте тук?

"--* Трябва да стигнем сградата на Парламента. Казват, че там има важни документи, които могат да ни дадат улики за лабораторията на яйцегените.

"--* Интересно\ldots А на нас ни казаха, че тук някъде има врата, която водела към базата. Такава една, в тунела. Скрита\ldots

"--* Много неща ви казват на вас\ldots Всъщност, май само Австралия е незасегната?

"--* Да, така е. Но не мога да казвам повече. Ако искате може да тръгнем заедно или всеки да си върви по пътя.

"--* Ами, трябва да споделя ситуацията на останалите. Елате, момчета! – провикна се Питър.

Почакаха в мълчание, докато дойдат другите. Имре отново се бе размрънкал, но се усети да млъкне, преди да стигне в чуваемата зона на Сид.

"--* Кво става ся? – запита Золтан.

"--* Ами -- започна Питър, – тия момчета са тръгнали към няква предполагаема база на яйцегените, която се намирала тук. Предлагат ни да им правим компания. Кво ше кажете?

"--* Ами, нашата цел? – запита Атила.

"--* Хм -- отвърна Питър, -- ако намерим лабораторията там, където тия хора предлагат, тя се обезсмисля. Иначе, може да ги питаме да дойдат с нас\ldots Някъв такъв е планът.

"--* Начи да им ходим по свирката? – възмути се Имре.

"--* Не е задължително\ldots -- каза Питър. -- Все пак, и нашата цел е да изтребим родилните клетки на яйцата. Пък после те могат да имат желание да ни помогнат.

"--* А ако нямат желание? -- запита Имре.

"--* Ами, дори да продължим сами пак ще сме спечелили. Най-малкото знаем, че все още има няква армия, която се опитва да помогне на света.

"--* Съмнява ме всичкото това\ldots -- отвърна Имре. – Като и да е\ldots Правете квото искате. Аз нямам избор\ldots

"--* Имре\ldots -- укорително му каза Питър. – Стига с детинщините\ldots

"--* Кво решихте момчета? – намеси се Сид преди спорът да започне.

"--* Ами -- каза Питър, -- идваме с вас. Ако не намерим нищо зад тази хипотетична врата, ще дойдете ли с нас до Парламента? В крайна сметка всички ще имаме полза от някакви документи.

"--* Хм\ldots Ще го видм.

"--* Та, ние не видяхме нищо по пътя си – каза Питър. – Къде се предполага да е мястото?

"--* Може да е в другия тунел. Може да е в противоположната посока. Може да е някъде другаде\ldots

"--* Що не се разделим на три екипа? – предложи Атила. – Така ще покрием повече разстояние. Който открие нещо, ще се обади по радиостанцията. Нали се предполага вратата да е някъде в началото на тунела?

"--* Може\ldots -- отвърна Сид. -- Обаче има опасност да останем много далеч един от друг. После, докато стигнат останалите до вратата, ще трябва да се чака. Но май ще е по-бързо\ldots -- Сид се умълча за около минута. – Знаете ли кво? Така ще направим, обаче, който стигне станция се връща и чака на платформата. Би трябвало за около два часа да обходим всичко.

Нямаше кой да противоречи и се разделиха на три екипа. Обхождането започна.

\section*{9.}

Редниците Джон и Джон вървяха един до друг. Те се познаваха доста от отдавна и бяха успели да изградят едно доста добро приятелство. Бяха част от екипа воден от Питър и Атила. Вървяха малко по-отзад и проверяваха допълнително за случайно пропусната врата. 

"--* Мислил ли си къде са отишли всички хора? – попита Джон.

"--* Как къде? Когато са ги взели яйцата ли? Ами, не знам\ldots -- отвърна Джон.

"--* Хм\ldots Като бях малък, някъде десет годишен, майка ми разказваше, че имало тайни фабрики за марсианци под земята. Били разположени навсякъде и понякога от отдушниците би могло да се подуши ароматът на витамини.

"--* Хаха – засмя се Джон, – детска история\ldots Нима вярваш?

"--* Не, естествено\ldots Но виж, ако беше истина\ldots Ето къде може да са отишли хората! Представи си, че яйцата са направили тайни подземия, където всички техни роби се крият!

"--* И каква е ползата? – почуди се Джон.

"--* Да си ги пазят, да се крият\ldots Знам ли? Може пък нарочно да ни правят влизането по-лесно, за да ни хванат в засада. Има доста възможности.

"--* Ахам\ldots Тук напипах нещо! – каза Джон.

Наистина бе попаднал на издатина, след която следваше съвсем тънък процеп. Той се спря и извика на другите. Докато ги чакаше да се съберат, проследи процепа с пръсти. Стигаше до над главата му и завиваше надясно. На няколко пъти щеше да го изгуби, но накрая разбра, че това са очертанията на врата.

Питър търпеливо изчака процеса и избута Джон. 

"--* Хм, тая издатина трябва да е няква дръжка\ldots Но как се отваря\ldots Сигурно ще е нещо трудно. С шифър може би. Кво мислиш Атила? – запита накрая.

"--* Ами, да кажем на другите екипи. Натисни го, поиграй си с него\ldots

"--* Първо ще се свържа със Сид.

Питър сподели откритието с водача на австралийския отряд. Сид, от своя страна, сподели откритието с другите. Всички се спряха по местата си и чакаха потвърждение, че това е търсената врата.

Атила се беше хванал да намери начин да я отвори. Той натисна издатината, опита се да я врътне, да намери нещо за набиране на код\ldots В крайна сметка я удари и от удара вратата подскочи. Оказа се, че е отключена.

"--* Окей, супер! Оповестихме присъствието си\ldots -- измрънка Атила. – Ще влизаме или ще чакаме?

"--* Начи, десет човека сме\ldots Ако дойдат и другите ще станем около трийсет – излишно тъпкане. Дай да влезем аз ти и двамата редници, които забелязаха вратата – предложи Питър.

"--* Звучи добре. Джон, Джон, ние четиримата ще изследваме. Останалите стойте на щрек.

Атила отвори вратата и освети пространството с фенера на пушката си. Помещението беше квадратно и малко. Единствено на пода му имаше един цветен капак, който имаше една дръжка, напомняща червения нос на клоун. Атила я хвана. Беше мека и като я стисна, тя изквича. Това ги стресна и те се отдръпнаха. Приличаха на аборигени, чудещи се на шише от кола. Той се протегна отново и я хвана, тоя път се стресна по-малко от изквичаването и успя да отвори капака.

Надолу водеше едно стръмно, зелено-синьо стълбище. Питър мина пръв, докато Атила държеше капака. Последваха го Джон-Джон, а накрая влезе и господин капакодържач.

Стълбището беше с доста добро сцепление. Изглеждаше да продължава безкрайно, но накрая, почти падайки, Питър стъпи на твърда земя. Освети помещението и се оказа, че се намират в доста голяма зала. Беше пусто. Отнякъде звучеше потракване. Зад тях се намираше една от стените на помещението. Другите три почти не се виждаха. 

"--* Дали да не викнем останалите? – попита Атила.

"--* По-добре не. Ще стане твърде шумно. Може още да не са ни усетили – отвърна му Питър.

"--* Добре, кво ще правим? – каза Атила. – Обикаляме?

"--* Ами, това ни остава. Дай да тръгнем покрай стените, все нещо ще намерим\ldots

"--* Окей. Джон, Джон, хайде!

Четиримата войници тръгнаха. Скоро забелязаха, че като осветят под определен ъгъл стените, те ставаха от бетонно-сиво на синьо-зелено. Това ги учуди – и стълбата и капакът на шахтата бяха толкова цветни. Трябваше да има някаква връзка.

"--* Я вижте тук – обади се Питър. Беше насочил фенера си към пода. – Като наклоня светлината, така че да е почти успоредна на пода, се виждат някакви издатини – той показа ефекта.

"--* Хм\ldots Как не сме ги усетили до сега? Трябва поне сто пъти да сме ги настъпили\ldots -- учуди се Атила.

"--* Тялото вижда и усеща каквото мозъкът му каже – включи се редник Джон.

"--* Да! – потвърди другият Джон.

Питър се приближи към една от издатините и я удари с крак. Кракът му нищо не усети и всички видяха как той минава през нея все едно тя не съществува. Водачът насочи светлината си към нея и видя разтрошена черупка и разлят белтък.

"--* Илюзия\ldots Всичко е илюзия. Или те действат върху нас, в момента, или нещо друго става. Не знам, труд\-но е\ldots -- Питър почна да губи думите.

"--* Излизаме! – изкомандва Атила.

Всички се втурнаха към стълбището. Ставаше им все по-трудно да контролират движенията си. Не поглеждаха назад. Синьо-зелената багра се бе превърнала в черна. Изскочиха през отвора на шахтата, последният го затвори, а после и затвори плътно вратата. Отпред ги чакаха учудените погледи на шестимата от отряда.

"--* Беше страшно – обясни задъхано Питър. – Сид -- свърза се по радиото, – идвайте, открихме ги!

След около двайсет минути успяха да се съберат всички. Дойде моментът за размисъл. Групата на Питър, Атила, Имре, Золтан и Едуард се скупчи около Сид. Питър обясни какво им се беше случило.

"--* Значи сега са наясно със заплахата от наша страна? – запита Сид.

"--* Ами -- започна Питър, -- не можем да сме сигурни\ldots Не знаем колко помнят. Интересното е, че успяват така да направят, че ако не съзнаваме съществуването им, да не можем да ги нараним. 

"--* Хм, щом толкова бързо са ви афектирали, значи ще е опасно да влезем без защита. Как може да сме сигурни, че това е лабораторията?

"--* Прилича на такова – отговори Питър. – По земята бяха наредени яйца, на равни разстояния, все едно се мътят. Може би такъв им е процесът\ldots

"--* Ами, учените? – запита Сид.

"--* Какво за тях? Може вече да нямат нужда\ldots Знам ли?

"--* Е, все пак толкова хора са под яйчен контрол. Къде са отишли всички?

"--* Не знам, ако знаех нямаше да сме тук и да се чудим – каза Питър.

"--* Какво ще правим сега? – попита Атила.

"--* Определено не може да влезем така. Трябва да се върнем до хеликоптера и да вземем пи-костюмите. 

"--* А защо не сте ги взели предварително? – попита Имре.

"--* Ами, тежки са, неудобни. Не са добри за бой. Сега като знаем, че заплахата е най-вече от яйцата, няма какво да се притесняваме. Проблемът е, че има само за нас.

"--* Не можем да се поберем всичките\ldots -- каза Питър. – Може да влезем ние петимата плюс теб, а другите да пазят – последното беше посрещнато с тихо мърморене от страна на войниците на Сид.

"--* Нека го обсъдя с тях – отвърна Сид.

Командирът тръгна към своя отряд. Петнайсет човека се събраха в плътен кръг и почнаха да обсъждат. Войниците почти нямаха избор, но Сид смяташе, че добрият лидер трябва винаги поне да каже на хората си какво мисли и да ги остави да го подкрепят. Всичко мина по план.

"--* Аз, ти – Сид каза, посочвайки Питър – и още шест от моите хора отиваме за костюмите. Останалите пазете. Ако нещо се случи се обръщайте към момчетата на Питър.

Групата потегли. Настана неловко мълчание. Отрядът на Сид се скупчи отделно от Атила, Имре, Золтан и Едуард. Такава неприязън не беше много добра за хора, които щяха да спасяват света. ``Е, мислеше си Атила, поне сме свикнали с Имре.''

След около час се завърнаха с костюмите. Шестимата ги облякоха и един по един почнаха да слизат по стълбите. Първо беше Сид, после Питър, а след това Атила, Имре, Золтан и Едуард.

Стъпиха на твърда земя и се озоваха в огромното помещение. Всичко беше същото. Включиха фенерите на оръжията си и почнаха да търсят по земята. Яйцата си бяха на мястото. Шестимата почнаха методично да стъпкват, мажат и убиват. Мълчанието превземаше голямата, ечаща зала. Имаха доста работа.

След около половин час действие, Атила спря.

"--* Абе -- започна той, – със сигурност има по-добър начин да се справим с това\ldots

"--* Сподели – отговори му Имре.

"--* Ами, от половин час тъпчем и все едно стоим на едно място. И писателят не може да се сети какво да напише\ldots

"--* Писател? – учуди се Имре.

"--* Да, ние сме герои, не сме ли? – отговори Атила.

"--* Ей, писателю! – провикна се Питър. – Какво да правим?

Тук писателят се замисли. Какво наистина можеха да направят героите му? Разполагаше със зала, която беше голяма (дали?). Стените ѝ бяха цветни и променяха багрите си спрямо ъгъла на осветяване. По земята имаше яйцегени, които май се мътеха. Героите му се опитваха да ги унищожат като ги мачкат. Обаче това беше бавно. Какво наистина можеха да направят?

"--* Май мисли – вметна Атила.

"--* Дам -- отговори Питър, – сложна е ситуацията ни. И на него не му е лесно. Пише, пише, измисля. Тия светове не се творят лесно. На самия Бог му е отнело седмица. Смятай един простосмъртен колко трябва да се мъчи\ldots

"--* Ми, да му помогнем! Да помислим с него! – предложи Атила.

"--* Добре.

Всички започнаха да мислят с писателя. Мислиха, мислиха, мислиха. Те мислите се търкалят странно. Като се концентрираш върху тях не щат да идват. Ама, като почнеш да блуждаеш около тях, даже като отидеш много далеч от тях, тогава пристигат с фанфари и облекчение. Такава беше съдбата на човека, все пак\ldots Но да се върнем към ситуацията.

Какво унищожава яйцата бързо? Лисицата? Или нещо с ``н'' беше. Ъмммм, невестулка! Да! Но не\ldots Тя просто обичаше яйца. Трябваше им нещо друго. Горелка? Можеше. Малко топло щеше да им стане, но беше вариант. Или пластичен експлозив. По дяволите! Те бяха военни, със сигурност имаха експлозиви. 

"--* Хей -- каза Атила, – имаме ли експлозиви?

"--* Тва е идея – отговори Питър.

"--* Ние имаме, ама може да срутим метрото\ldots -- каза Сид.

"--* И какво? Това е възможност да спасим света, а ние се притесняваме за едно метро\ldots -- опроверга го Атила.

"--* Ами, то май по-лесен начин няма. Пък и ще е сигурен. Обаче ще трябва да почакаме. Ще трябва да се свържа с наземната база, от там да докарат експлозиви и после да ги смъкнем тук.

"--* Ами, какво решаваме?

"--* Ще го правим.

Те почнаха да се изкачват. Сид беше начело. Спря се точно, когато отвори вратата. Отстъпи назад и я затръшна. Шестимата не можеха да се поберат в малката стаичка, за това някои останаха на стълбата.

"--* Кво става? – запита Питър.

"--* Няма ги. Хората ми ги няма. Няма светлина, нищо – отговори Сид.

"--* Как? Нали бяха тук? Петнайсет човека не изчезват просто така.

"--* Е, явно понякога се получава\ldots

"--* Мисля, че е опасно да излезем – каза Атила.

"--* И кво? Ще стоим тук? – отговори Имре.

"--* Чакайте, чакайте. Ще се свържа с базата. Те може да имат някакъв отговор.

Сид извърши процедурата. Не получи отговор. Само мълчание и пращене. Почака няколко минути. Отпусна ръката си и намести радиостанцията.

"--* Никой – каза просто.

\section*{10.}

Шестимата постояха още малко и най-накрая Сид се реши да излезе. Той отвори вратата като държеше оръжието си пред себе си. Огледа се, но беше тъмно. Дори да имаше заплаха, не би я видял. От друга страна пък не биваше да включват фенерите – така щяха да се открият на врага. Опцията им бе да действат на сляпо.

След Сид излязоха и другите. Наредиха се в кръг. Тръгнаха по посока ``Сердика'' с надеждата да излязат на чист въздух. Сега, в тази тъмница, имаха чувството, че горе ще е по-безопасно.

Гледаха да са тихи и въобще не им беше до разговори. Въпреки това, на всеки му се искаше да запита другите – къде са се дянали тези петнайсет човека. Не може просто ей така да са изчезнали. Вярно, че са били под земята, когато това е станало, но все някакъв вик би трябвало да чуят. Но тези въпроси бяха почти безсмислени. Дори да ги дискутираха, нямаше да разберат отговора им. Само щяха да са по-уплашени.

Вървяха през тунела, внимавайки да не се спънат в релсите. Като стигнаха до станцията, се качиха, един по един, на платформата. В същата формация тръгнаха към първия видян ескалатор. Малко преди да го стигнат чуха стържещ звук зад тях. Атила, който в момента пазеше гърба им, се забърза няколко крачки и почти се изравни с другите двама, пазещи страните.

Виждаха бледи силуети. Бяха доста и се движеха към тях. За момент шестимата се почудиха, после пуснаха фенерите на оръжията си, осветиха нещата и като се увериха, че това са били човешки същества преди време, почнаха да стрелят. Ехото на станцията само усилваше звука от изстрелите. Съществата падаха, но сякаш прииждаха още. Шестимата почнаха гърбом да се качват по спрелия ескалатор. Атила и Имре бяха най-отпред, последвани от Золтан и Едуард, а Сид и Питър бяха най-отзад.

Те двамата първи спряха да стрелят след като стигнаха края на стълбището. Веднага се обърнаха и почнаха да оглеждат за изход. Видяха един в края на коридора. След като се увериха, че всички са минали през ескалатора, се затичаха и излязоха на чист въздух. 

"--* Сега накъде? -- запита Атила.

"--* Към Народното събрание! – каза Питър и тръгна.

Другите го последваха без много да му мислят. Вече се чуваше приближаването на съществата.

Сградата на Парламента се намираше на около две\-ста-триста метра от метростанцията. Стигането беше лесно, въпросът бе дали щяха да могат да влязат вътре. После като влязат, идваше въпросът дали няма да ги чака същата опасност. Доста проблеми имаше, но това беше най-доброто решение за сега. Ако се бяха скрили някъде, можеше и да не могат да излязат, докато сега ще знаят, че са стигнали до целта си.

Стояха запъхтени пред сградата. Тя беше непокътната. Всичко на десет метра от нея беше здраво и непипнато. Съществата бяха останали доста назад и шестимата имаха време да разберат как да влязат в Парламента. Той беше ниска постройка, с три дървени врати отпред. Те бяха остъклени, което вся надежда в отряда.

Атила отиде и първо пробва да отвори една от вратите. Не успя. Удари с лакът стъклото и то се разби на парчета. С малко повече внимание можеха да се проврат и да влязат в сградата. Атила доразчисти острите остатъци от стъклото с оръжието си. Влезе и започна да оглежда вътрешността, докато чакаше другите. Пред него стоеше голямо преддверие с два реда колони отстрани. Отдясно и отляво имаше масивни врати, а отпред се виждаше входът към заседателната зала.

"--* Къде трябва да са архивите? – запита Сид.

"--* Ами, отдолу има нещо като мазе. Цяла подземна система даже. Там предполагаме, че са ги скрили – отговори му Питър.

"--* А как можем да стигнем до там?

"--* Тепърва ще разберем.

Питър тръгна на чело. Отрядът се запъти първо към страничните врати и ги провери една по една. Общо взето – офиси. Обърнаха бюра, махнаха килими, но нищо не видяха. Остана само голямата заседателна зала. Влязоха през тежките врати и се озоваха в помещението, където беше диктувано бъдещето на тази някога силна държава. В дъното имаше висок подиум на няколко нива. На първото ниво беше трибуната за говорене. Зад нея, малко по-високо, стоеше една доста по-широка трибуна, на която по принцип се намираха председателите на събранието. Пред всичкото това нещо бяха наредени катедри със столове като в театър. На стената отпред стоеше гербът на България. Лявата и дясната стена бяха почти пусти, а на задната се намираше някаква тераса.

В дъното имаше две врати, които хванаха вниманието на Питър. Той се запъти към тях без въобще да се замисля, че има шанс входът да е някъде из тази голяма зала. Напосоки избра дясната и я отвори. Къс коридор водещ към някаква малка стая се откри пред него. Другите вече го бяха последвали. В тая стая имаше един малък прозорец и една картина. На картината беше нарисувана някаква модерна абстракция представляваща синя линия по диагонала на платното. В дъното, отдясно на прозореца, имаше една оставена топка.

"--* Тук май няма нищо, а? – запита Сид.

"--* Дам\ldots -- отговори Питър и тръгна към топката.

Хвана я, не беше толкова тежка. Подхвърли я един, два пъти и си я сложи в джоба.

"--* Може да потрябва – каза той.

Върнаха се обратно по пътя и решиха да пробват другата врата. Тя беше заключена. Питър извади оръжието си и стреля два пъти в ключалката. Тя се изкриви и, след един по-силен ритник, вратата се отвори. Отново същият коридор със същата стая в дъното. Тоя път обаче стаята беше по-богата. Под прозореца ѝ беше поместена голяма, масивна ракла с катинар. Питър отново прибягна до оръжието си и стреля. Счупи катинара, понечи да вдигне капака, но не му се отдаде. Атила и Сид му дойдоха на помощ – пак не успяха.

"--* Трябва да има нещо доста важно тук щом имаме такива трудности\ldots -- констатира Атила.

"--* Дааа, дайте да огледаме. В детските книжки такива неща се отварят с хитрост – каза Питър.

Те разгледаха стаята. Единственото, което откриха бе още едно топче. Питър взе и него, подхвърли го един два пъти и почна да усеща как му пари в джоба, където бе сложил другото. Извади го, доближи ги. Двете топчета смениха цвета си. Едното бе станало нажежено-червено, а другото нажежено-синьо.

"--* Като нагрят метал са – каза Атила.

Посрещна го мълчание. Питър мислеше какво може да са тези топчета. Той ги почукна леко. После малко по-силно. Усещаше някакво разтапящо чувство между тях. Усещаше как, след всяко чукване, те се вливат едно в друго и после биват разделени. Покрай сливането цветът на допир преминаваше в зелен. Питър все още не знаеше какво да прави с тях. Нямаше къде да ги постави, а пък и те не щяха да правят нищо.

На последното чукване те се сляха. Станаха страшно горещи и Питър бе принуден да ги пусне. Паднаха на пода, разтекоха се и прогориха паркета, а после и еднометровата бетонна плоча под него. Пропаднаха надолу и като стигнаха пръст се спряха. Нажеженото им сияние осветяваше малко от подземието. На пръв поглед не беше много дълбоко, но трябваше да измислят как да слязат.

"--* Сигурно има стълба тук някъде – предположи Атила.

"--* Аз май видях в един от кабинетите – каза Питър и се втурна натам. 

След около минута се завърна с една сгъваема, алуминиева стълба. В стаята беше малко тясно и те изпитаха доста неудобства докато позиционират съоръжението на мястото му. Накрая, след като успяха, пособието се беше опряло малко над тавана на подземието. Също така, нажежената локва от топчетата беше на косъм прекрачена от стълбата. Трябваше да внимават като слизат.

Този път Сид поде инициативата. Той си провеси краката от дупката и напипа първото стъпало. Отпусна се и започна да слиза. На няколко пъти стълбата се заклати, но положението бе овладяно. Той стъпи на твърда земя и хвана краката ѝ. Останалите слязоха. Шестимата отново бяха в някаква зала. Този път подът беше дървен. Всички включиха фенерите си и почнаха да шарят с тях из подземието. Забелязаха най-различни обекти – стари катедри, лавици, телевизори. Общо взето, складово помещение.

"--* Къде всъщност е входа? – запита Атила.

"--* Сега ще разберем – отговори Питър.

Изглежда бяха в някаква не много голяма стая, която отдавна не е била посещавана. От нея се излизаше през една дървена врата. Те огледаха още веднъж помещението и след като не забелязаха нищо интересно, продължиха към вратата. Сид хвана дръжката, натисна я и бутна напред. Вратата се отвори с леко скърцане. Сид провря първо оръжието си, огледа с фенера и след като не забеляза някаква конкретна опасност, влезе навътре. Останалите го последваха.

Откри се друга стая, по средата на която имаше стълбище. След като огледаха за опасност, Сид го изкачи.

"--* Зазидано е – каза, след като стигна до горе.

"--* Интересно – отвърна Питър. – Значи сме близо до нещо тайно.

Покрай стените имаше библиотеки с много книги. Те бяха потънали в прах и представляваха доста еднаква маса чернота. Шестимата почнаха да ги разглеждат. Първо бяха внимателни – вадеха ги една по една, разглеждаха ги и след като не намереха нищо, ги слагаха отстрани. Постепенно почнаха все по-бързо и по-бързо да разлистват. Минаха всичко по библиотеките. Нищо интересно – учебници по политика, икономика, право. Тук-там някои исторически книги и класически романи. Оказаха се заобиколени от празни шкафове. Накрая почнаха да махат и тях. Откриха се каменни, голи стени.

"--* Това е – каза Атила. – Няма нищо повече\ldots

"--* Но все пак са го зазидали. Все нещо трябва да има, някаква информация, някъде, скрита\ldots -- отвърна Питър.

"--* Не знам\ldots Останали сме страшно малко боеспособни хора. Трябва ни чудо. Нямаме такова. Всичко просто свършва – сподели Сид.

"--* Хей, недей така. Не бива да се поддаваме на отчаянието. Щом сме живи, все още има шанс, макар и малък – окуражи го Питър.

"--* Какъв? Изгубих всичките си хора, не знам какво се е случило с останалите, а сега седим в някаква стая, търсейки чудо и нямайки друга опция\ldots

Всички се умълчаха. В тишината се чу скърцане на някакъв тежък обект, идващо от съседната стая. Втурнаха се натам. Нещо липсваше. Дупката беше тъмна, нямаше светлина от нея. Питър бързо се изкачи по стълбата, опита се да бутне нещото отгоре, но не успя. Слезе. Хванаха стълбата и в последни отчаяни опити се опитаха с нея да поместят обекта. Не им се получи.

\chapter{Установяване на властта}

Паралелно със загубата на отряда на Сид, изчезнаха и другите части от австралийската армия. Вече нямаше сила, която да се противопостави на яйцегените. Малкото неконтролирани хора се бяха скрили в планините. Те се надяваха там да установят хранилище на цивилизацията, от което един ден тя да се възроди. Все пак, щяхме да се спасим, нали?

Контролът на яйцата се извършваше посредством много хранилища подобни на това под метрото на ``Сердика''. Над тези хранилища имаше усилватели на пи-полето, които разпространяваха мислите и желанията на контрольорите и ги пращаха към всички заинтересувани. Съществата не можеха да се забележат по улиците на градовете, защото те бяха скрити по техните си апартаменти и блокове.

Главният проблем на яйцегените бе да установят кр\-алството си. На първо място бе нужно нещо, което да храни техните поданици. Те леко полеко почнаха да създават ферми. Контролът върху хората също беше доста трудоемък и за това щяха да пускат на свобода по-слабите същества. Щяха да сформират нещо като полиция от контролирани, които да пазят реда. Всички други щяха да се трудят за благото на империята. Предстояха тежки години за планетата Земя. По-точно – за обитателите ѝ.

Градовете биваха поправяни и превръщани в по-мал\-ки пространства, защитени и укрепени. За сега беше невъзможно да се пътува от място на място, но и нямаше нужда. Яйцегените си комуникираха по техните си пътища, което им позволяваше да се свържат с другия край на света без проблеми. Хората, пък, нямаха правото и причината да пътуват. Та, те не бяха хора все още. Те бяха просто контролирани същества живеещи за благото на някакви яйца.

И така, леко полеко, светът ставаше нов.

\part{Нов свят\ldots}

\chapter{Яйца и политика}

\section*{1.}

Яйцегените, имайки клетъчен произход, бяха доста органични като поведение. Те бяха разпръснати, но едновременно с това – единни. Комуникацията между големи струпвания на яйца бе страшно бърза. Поради тази причина, въпреки големите разстояния, те можеха да общуват така все едно са едни до други. И те действаха като част от едно цяло – за благото на псевдо-организма, който окупираха. При тях нямаше егоизъм, а само цел – да се запазят живи. Едно яйце би се жертвало, ако това значи, че организмът ще живее.

Те бяха разположени в центровете на големите градове. От там, през усилватели на пи-вълни, контролът им стигаше малко след границата на всеки град. Площите около населените места бяха слабо обитавани от свободни хора, които едвам успяваха да живеят. Най-близко до стените на всеки град имаше ферми, чрез които се изхранваха човеците. Те бяха обработвани от неконтролирани. Контролираните имаха по-важната задача да следят за реда и да развиват науката.

В центъра на всеки град беше сградата за контрол, а около нея сградите за наука. След това почваха жилищата. Високи, грозни блокове, в които хората бяха натъпкани. Неконтролираните живееха между вътрешния и външния пръстен Контролирани. Транспортираха се с подземни влакове или наземни автобуси.

Полиция имаше. Тя беше и армия. Всеки град си имаше по няколко хиляди Контролирани, грижещи се за реда. Те бяха тежко въоръжени с последните открития в оръжейната промишленост. В началото бяха потушили няколко спонтанни, отчаяни въстания на първите Неконтролирани.

Деца се раждаха, когато трябваше. Понеже не искаха да се занимават с излишни закони, яйцегените превентираха раждането като хранеха хората с определени съставки. Храната се намираше в локални столове. Те бяха огромни – всеки един побираше по няколко стотин, дори хиляда човека. Хората се хранеха тихо и тъжно първите няколко месеца, но постепенно почнаха да говорят. Шока преминаваше и те откриваха, че разговорите са доста добър начин за борба с депресията. Във фермите те пееха, никой не ги спираше, пък и яйцата откриха, че така хората са по-продуктивни.

Така общо взето живееха в новия си свят.

\section*{2.}

Мег беше свободна. Тя живееше до едно дърво на около километър от границата на град Киев. Родителите ѝ също бяха свободни, както и техните родители. Даже се предполагаше, че баба ѝ и дядо ѝ са били свидетели на яйченото поробване. До това дърво, освен Мег, живееха едно куче и един човек. Те си бяха направили прост подслон с огнище отпред.

Миналия ден Мег беше забелязала един доста охранен заек да се навърта около територията ѝ. Днес тя се беше скрила зад един храст, причаквайки животното да се приближи към примамката ѝ. Ловът изискваше търпение. Понякога плячката идва на момента, но в повечето случаи се налага да чакаш. Мег стоеше на едно място вече втори час. На няколко пъти бе опъвала и отпускала тетивата на лъка си. Накрая заекът се появи. Той приближи храната, подуши я, почуди се малко и започна да гризе. Мег опъна тетивата и стреля. Стрелата прободе дивеча и осигури вечерята на тримата обитатели на подслона до дървото.

Човекът беше наклал огъня. Мег одра заека, изчисти вътрешностите му и го забоде на един здрав кол. Печенето продължи около два часа. После тримата се събраха и започнаха да ядат.

Вече се беше смрачило след като привършиха с яденето. Човекът, Мег и кучето се сгушиха един до друг и се наслаждаваха на чистото звездно небе.

"--* Дядо ми разказваше, че тогава звездите рядко са били толкова красиви – започна тя. – Мръсотията и светлините пречели на блясъка им.

"--* Да, ако не друго поне можем да се порадваме на красивото небе – отговори човекът.

"--* Нима това не е достатъчно? Нима не сме щастливи?

"--* Мег, поробени сме.

"--* Не ни е отнета земята, животните, звездите. Аз смятам, че съм почти свободна.

"--* Но другите? – опонира я човекът. – Какво ще кажеш за тези зад крепостните стени? Едни контролирани, а други заплашвани.

"--* Не знам.

Умълчаха се и продължиха тихото си възхищение към небето. Мег целуна човека и легна да спи.

На другия ден мина Маркус. Той им носеше ябълки и вести. Каза им, че някои свободни откъм изток са успели да установят контакт с неконтролираните. Имало начин да се влезе в тия крепости обаче трябвало да се внимава. Било опасно, можели да умрат хора. Мег не се впечатли – хора умираха всеки ден. Човекът се замисли.

"--* Дори да влезем, какво можем да направим? – каза той. – Пи-полето е силно, пък и те разпознават собствеността си. Веднага ще ни забележат.

"--* Откъде знаеш за това? – запита Маркус.

"--* Кое – полето? Ами, да кажем, че се срещнах с един неконтролиран. Един от по-просветените. Той ми ги разясни тия работи.

"--* Защо не си споделил с останалите? – запита Маркус.

"--* Беше наскоро, мислих да отида до езерото, но нямаше кога.

"--* Разбирам. Но, човек, за тия работи трябва да се намира време – упрекна го Маркус.

"--* Да, знам.

"--* Желаеш ли да обядваш с нас? -- Мег покани Маркус. -- Нямаме много, но можем да го споделим.

"--* Благодаря, но бързам. Успех – отвърна Маркус и продължи по пътя си.

Мег и човекът седнаха около огъня и почнаха да приготвят ядене. Небето беше чисто и студено-синьо. Есента наближаваше, а в Киев зимата беше люта. Когато се спуснеха първите студове, Мег и човекът прокопаваха плитка дупка в подслона и там палеха огън.

"--* Какво мислиш за инфилтрацията? – запита Мег.

"--* Определено не е опция. Но виж – ако снабдим неконтролираните с някакви оръжия\ldots

"--* Откъде ще вземем тия оръжия? Пък дори и да ги имаме, неконтролираните са малко. Има яйцегени, които веднага могат да почнат да контролират, има полиция, армия. Силни са.

"--* Мег, твърде бързо си почнала да се отчайваш – каза човекът.

"--* Не, просто не знам\ldots Опитвам се да бъда реалист.

"--* Всеки проблем си има решение. Просто смени думичките. Решението е да вземем оръжия. Решението е да спрем контрола на яйцегените. Решението е да няма полиция, армия. Решението е да станат слаби. Не е ли по-лесно така?

"--* Илюзия\ldots Живееш в една илюзия, човек. Опитваш се да бъдеш оптимист и щастлив, защото не ти остава друго. Не знаеш как да решиш проблема и за това се опитваш да го гледаш положително. Няма решение. Никакво – заключи Мег.

Двамата се умълчаха. Нямаше смисъл да я убеждава. Поне не в момента. Той знаеше, че когато бе нужно, тя щеше да действа. Нищо друго не ѝ оставаше.

Така минаха няколко дни от простия им, свободен живот. Дойде вест, че ще има събрание на свободните. Мег и човекът трябваше да отидат до езерото след два дни. Щели да обсъждат план за освобождение. Те взеха малкото им нужен багаж и оставиха кучето да пази. То не беше твърде доволно от факта, но се примири и обидено застана пред палатката.

Езерото беше на половин ден път. Събираха се там, защото всички го знаеха. Мястото на събранието бе отбелязано от един голям, бял камък. Всички се нареждаха в кръг като най-важните се скупчваха около камъка. Останалите се опитваха да дочуят какво казват те и да противоречат, ако нещо не им хареса. Естествено, противоречието беше почти невъзможно, но все пак имаха това право.

Мег и човекът бяха едни от последните пристигнали. Те се промъкнаха през тълпата и застанаха на своето място около камъка. Петнайсет човека се опитваха да измислят как да освободят поробената част на човешката раса. Тежка задача, но не непременно невъзможна. Бяха щастливци, че не всичката технология се бе изплъзнала от ръцете им. Все още имаха радио, чрез което се свързваха с другите въстанически отряди. И въпреки, че връзката бе твърде лесна за подслушване, те си позволяваха да споделят важна информация по нея.

Маркус пристъпи напред и започна:

"--* Здравейте! С вас, свободните, имаме важната задача да решим как ще процедираме по време на всеобщото въстание. Целта ни е да сме бързи както в тази дискусия, така и в действията си след това. Главните проблеми са два – как да влезем в града и как да останем незабелязани за яйцегените. По първия проблем има решение – неконтролираните по фермите са съгласни да помагат. Те споделиха, че има аварийни изходи от града направени за яйцата. Тесни са, но е възможно да се влезе през тях. Охранявани са, но не чак толкова. Влизайки, обаче, даже още докато сме на територията на фермите, яйцегените ще усетят, че не сме нещо тяхно. И тук мисията спира. Свършва, край. Та, мисля, че това е наложителният въпрос, който трябва да отговорим.

Настана кратко мълчание. Един млад човек с мустаци се включи:

"--* Не можем ли да въоръжим неконтролираните и те да инфилтрират?

"--* Не бива – след кратък размисъл се включи Игор. – На първо място, те се скапват от работа. Не са добри войници. Могат да помагат след инфилтрацията, но не и за започването ѝ. Второ, повечето са простовати хора, но това не пречи да се замислят защо ги караме да вършат това, с което ние сме се нагърбили. Наистина, шансът за такива мисли е малък, но е на лице. Не можем да си позволим недоволство. И като каза въоръжаване\ldots Това е може би третият проблем, върху който трябва да помислим днес.

"--* Да, Игор, така е – отговори Маркус. – Но нека първо се опитаме да разберем как можем да спрем пи-вълните.

"--* Знаете ли – включи се човекът, – дядо ми е разказвал как по време на последния опит за офанзива, австралийците са използвали някакви пи-резистентни костюми\ldots

"--* Те са взети и се използват, но не са достатъчно – прекъсна го Маркус.

"--* Чакай сега. Системата им е проста и мисля, че има шанс да ги пресъздадем. Общо взето, пи-вълните биват спирани в началото от нещо в нашето тяло. После, след упорита борба, пробиват тази защита и стигат до мозъка, където взимат контрол. Някакви учени са забелязали, че оловото имитирало тази първоначална защита и създали оловни костюми на базата на това знание. Олово нямаме, но можем да си набавим.

"--* Тези костюми не са ли тежки? – запита някакъв.

"--* Да -- отвърна човекът.

"--* Ами, тогава, ние ще влезем, но ще сме способни ли да се бием качествено? – каза ония. – Ако тези костюми блокират движенията ни и ни уморяват, то ние ще сме доста лесна плячка за контролираните.

"--* Може да не ни разберат. Всъщност, как контролираните отличават своите от чуждите? – запита човекът.

"--* Мисля, че яйцата им казват – отговори Маркус.

"--* Мислиш? Как може да сме сигурни?

"--* Добре, това ще се проучи. Ще пратим шпиони.

"--* Окей -- продължи човекът, – ако яйцегените им казват, то остава единствено да се притесняваме за това, дали могат да ни разпознаят по дрехите.

"--* Това трудно ще го разберем – каза Маркус. – Но за всеки случай може да се дегизираме като неконтролирани. Някак си трябва да скрием тези оловни костюми.

"--* Е, Маркус – включи се Игор, – та ние още не сме ги създали. Не се знае, може и да се получат компактни. Нека преминем към въпроса за оръжията. Какво ще кажете?

Никой не противоречи. Продължиха дискусията и обсъдиха как ще си набавят оръжия. Решиха да разгледат една стара военна база, намираща се на около пет километра от езерото. Надяваха се там да има нещо полезно. След това приключиха и се разделиха.

\section*{3.}

Петима отиваха да проверят базата. Сред тях беше човекът. Комплексът от военни сгради бе разположен в гориста местност, която поради липсата на поддръжка бе доста трудна за преминаване. Имаше един централен, асфалтиран път през нея, който беше обрасъл и по него, тук-там, лежаха паднали дървета. Пътят свърши и пред тях се откри един мост с пресушен канал под него. Входът се пазеше от две ръждясали порти с решетки. Зад тях се виждаше път водещ към плаца, заобиколен от ниски сгради. В дъното се намираше една по-висока постройка с нарисуван на нея, вече избелял, гербът на Украйна.

Мостът беше чист. От парапета се виждаше застинала мръсотия по канала и едно куче дъвчещо някаква мърша на дъното. Петимата стигнаха металните врати и човекът отвори едната. Минаха през тях и тръгнаха към плаца. Спалните помещения, покрай които минаваха, бяха с изпочупени стъкла и липсващи врати. С приближаването си до централната сграда, забелязаха че и тя не е в по-добро състояние. Надяваха се, ако имаше оръжия тук, то те да са били скрити на сигурно. Също така се надяваха да не ги изненада някой от вътре. Само човекът имаше пистолет и то само с пет патрона. Добавяйки и факта, че човекът бе стрелял само веднъж в живота си, положението им щеше да стане доста напечено.

Той влезе с оръжието в ръката си и се огледа. Бяха в голямо преддверие със зелени стени и мръсно бежов под. До стените тук-там имаше пейки като някои от тях бяха изпочупени. От антрето излизаха три врати. Тази в дъното водеше към малко коридорче със стълбище към втория и третия етаж. Страничните водеха към две зали, в които някога бе имало чинове и дъски за писане. Петимата продължиха към стълбището.

Те влязоха в малката стая и първото нещо, което забелязаха беше един метален капак затварящ частта от стълбите, която слиза надолу. Капакът беше заключен с катинар. Човекът понечи да вземе оръжието си.

"--* Чакай – обади се един. – Мунициите са ни ценни. По-добре да потърсим за ключ, пък ако няма, ще стреляш.

"--* Но ако долу има оръжие? – противоречи му човекът.

"--* А ако е празно? По-добре малко да се забавим, отколкото да изхабим един патрон.

Човекът се съгласи. Качиха се нагоре. На втория етаж имаше големи офиси, в които явно се бе помещавало командването. Те минаха по всяко бюро, шкаф и рафт. Претърсиха всичко, но не откриха ключ. Слязоха и човекът стреля. Катинарът се разчупи, а навън се разхвърчаха някакви птици. Човекът се наведе и вдигна капака. Посрещна ги тъмнина. Един от петимата извади малко фенерче и го насочи в бездната.

Светлината откри циментиран под. После тя се придвижи към първата стена и забеляза първите стойки с оръжия. Човекът с фенерчето пристъпи напред и заслиза по стълбата. Озова се в що-годе голяма стая и спря в учудване. Другите също не вярваха на това, което виждат.

"--* Супер – каза човекът. – Проблемът с оръжието ни е решен.

"--* Как ще го пренесем? – запита някой.

Обърнаха се към него недоумявайки ситуацията. Не се бяха замисляли по тоя проблем. Нито Маркус, нито Игор бяха помислили върху него.

"--* Можем да се установим тук – хрумна му на човека.

"--* Не е ли твърде известно това място? – запита някой.

"--* От кой има да се пазим? Престъпниците са незначимо малко. Останалите сме горе долу обединени. Твърде малко свободни хора останахме за да се разцепим и да имаме противници.

"--* Добре, трябва да уведомим другите – каза друг.

"--* Ами, не можем да оставим откритието така. Предлагам един да отиде, а другите да пазят. До тъмно има поне три часа. За три часа се стига до Маркус – каза човекът. – Даже този, който отива може да се снабди с някое от оръжията за да е в безопасност, пък и да докаже, че това което казва е истина.

"--* Кой да е?

"--* Все едно, кой познава пътя най-добре?

"--* Май ти. Ти ни доведе.

"--* Добре. Вие вземете каквото ви е нужно и се постарайте никой да не ви изненада. До утре сутринта ще дойдем. Гледайте да поспите. Следващите няколко седмици ще има доста за вършене.

Човекът влезе и взе три пълнителя. Огледа се, не си хареса нищо друго и тръгна. Останалите гледаха подире му. Когато се изгуби от погледа им, те взеха да се подготвят за нощта. Харесаха си оръжия. Направиха няколко пробни изстрела и се надяваха да не им се наложи да ги ползват скоро. Сложиха няколко чина пред вратата. Мислеха да сложат дъски на счупените прозорци, но решиха, че ще е по-ефективно да барикадират вратите към учебните зали. В антрето имаше само два прозореца откъм входа.

\section*{4.}

Нощта на четиримата мина спокойно. Към два след обед на другия ден дойдоха останалите. Бяха общо двайсет. Водеха ги човекът и Маркус. Това щеше да е отрядът по ремонта на базата. После сигурно щяха да прехвърлят всички по-тренирани войски. Но те не бяха много. Тук войска означаваше петдесет организирани човека, някои от които бяха използвали оръжие. Маркус се надяваше това да се промени с новата база.

Първото нещо, което направиха като дойдоха бе да извадят всичко от склада и да го инвентаризират. Преброиха шейсет автомата, толкова пистолета, две големи кутии с гранати и пет кутии с патрони голям калибър за автоматите, и три кутии с малокалибрени муниции за пистолетите. Освен това имаше пет ръчно преносими гранатомета. Върнаха всичко на мястото му и заключиха склада с нов катинар.

Следващата им задача бе да определят места за спане на хората, които щяха да участват в ремонта. В спалните помещения повечето дюшеци бяха изкорубени и проядени, а прозорците липсваха. Решиха да продължат по старому и да спят отвън, около лагерния огън. Разположиха се на плаца. Всеки издири някаква постелка и подобие на калъфка от базата и си хареса място. Хората се скупчиха на групи и няколко малки огнища пробляснаха. Двама пазеха входа на моста, а петима патрулираха около стените. Всъщност, патрулът може би не беше нужен, защото стените на базата бяха три-метрови, бетонни грамади. Въпреки това, Маркус реши, че повече сигурност няма да им навреди. Караулите се сменяха на три часа. Втората нощ също мина доста спокойно.

Ремонтът почна от спалните. Дюшеци нямаше откъде да намерят, за това решиха да махнат всички легла и на мястото им да сложат скалъпени постелки. За някои от войниците щеше да има дюшеци, за други ръбест под, но те бяха свикнали. Понеже нямаше откъде да вземат стъкла, решиха да затворят прозорците с колкото дървен материал намерят. По средата на всяко помещение разкъртиха пода и направиха малка дупка, в която да се пали огън. Това им отне два дена.

По плаца нямаше какво да се пипа. Концентрираха се върху главната сграда. Разчистиха всичко от втория етаж и обзаведоха два офиса с каквото имат. Един за Маркус, другия за човекът. На първия етаж окончателно затвориха вратите към учебните зали. Нямаха нужда от толкова пространство. Преместиха оръжията на втория етаж, в една малка, неосветена стая. Сложиха пазач пред нея. Сложиха двама пред входа на сградата. Вече се чувстваха достатъчно защитени.

На следващия ден, след края на ремонта, дойдоха останалите и се настаниха. Заделиха се малко муниции за някаква военна подготовка. На всеки бе дадено оръжие и всеки бе сложил отличителен знак на автомата си.

Базата беше установена. Беше време за действия. Маркус, човекът, Игор и още двама от базата се събраха на втория етаж за да обсъдят подготовката на въстанието.

"--* Аз мисля – започна Маркус, – че първо трябва да видим как стоят работите с предпазните костюми. Откъде можем да се сдобием с олово?

"--* По времето на дядо ми е имало голямо предприятие за производство на олово в областта – каза човекът. – Сега в него може да има някакви хора, но ако ни пречат винаги можем да приложим малко сила.

"--* Колко олово ще ни трябва?

"--* Не знаем. Та, ние нямаме представа как да ги сглобим тия костюми\ldots Старите модели, които имаме, са два слоя плат с тънко оловно фолио между тях. Това фолио ние много трудно можем да го извлечем. Мисля, че най-близкото, до което можем да стигнем, е някакви оловни листи, които да наредим между платовете.

"--* Какъв е шансът по тоя начин костюмите да проработят? – попита Игор.

"--* Не знаем. Може и да стане, може и да не стане\ldots Трябва да пробваме. Някой ще трябва да се жертва за каузата и да се опита да влезе незабелязан. Но това ще го мислим като стане време.

"--* Да, трябват жертви\ldots -- каза Маркус. – Добре, ако там няма олово?

"--* Как така? – запита човекът.

"--* Ами, ако са го взели, ако е изчезнало, ако няма?

Мисълта беше твърде неприятна и никой не се беше престрашил да вкара възможността в уравнението.

"--* Ще му мислим\ldots -- каза човекът.

"--* Колко е далече предприятието? – попита Игор.

"--* Два дни път, западно от Киев. Аз знам пътя, минавал съм от там по време на лов – каза човекът.

"--* Колко хора да отидат?

"--* Хм, мисля, че десетима въоръжени ще сме достатъчно. Но ще трябва да пренесем оловото някак си.

"--* Тук има едно ремарке. Вие ще трябва да го дърпате, но това е най-доброто, на което сме способни – каза Игор.

"--* Мисля, че ще свърши работа – отвърна човекът.

"--* Какво правим след като се сдобием с оловото? – попита Маркус.

"--* Ами, ще се хванем да сглобяваме костюмите. Мег ще събере шивачи, които да свършат тази работа – каза човекът.

"--* Но преди това трябва да оформим оловните плочи, нали? – отговори Маркус.

"--* Да, оловните плочи\ldots Това ще го обмислим спрямо състоянието на материала. Не знаем в каква форма е, какво представлява.

"--* Добре, после пробваме костюмите и пращаме някой да опита дали всичко това действа?

"--* Да – каза човекът.

"--* И ако всичко действа, нарамваме оръжията и влизаме през първата дупка?

"--* Не, ако всичко действа, ще направим по-обстояте\-лствен план. Смятам, че сега не му е времето да го мислим – каза Игор.

\section*{5.}

Два дни път значеха тръгване рано сутрин, движение до късно през нощта, няколко часа сън и отново така на другия ден. Планираха да пристигнат в предприятието, да вземат оловото и да преспят там. Връщането щеше да отнеме около три дена поради тежкото ремарке.

Бяха екипирани както обикновено, с дълги панталони, блузи с дълъг ръкав, жилетка и някакви скалъпени обувки. Десетимата носиха от новите автомати. Редуваха се по двама на един час да теглят ремаркето.

Вървяха през широка равнинна местност. Движеха се покрай един стар, обрасъл път. Ръждясали табели казваха, че това било път Т1002, но никой не можеше да бъде сигурен. Нима пътищата имаха имена? Нима ако кажеш ``Хей, път Т1002!'', пътят ще ти обърне внимание? Не, той си беше просто място за минаване. 

Трябваше да ходят до някаква местност наречена Ракийвка и после да продължат към някакво селце, което го наричали Озера. Там имаше старо летище, на което смятаха да прекарат нощта.

Към осем вечерта стигнаха Озера. Влязоха от източния вход на селцето. Трябваше да изминат един дълъг главен път. От двете им страни се намираха порутени сгради. Стигнаха до средата на улицата и забелязаха една къща, в която светеше. Учудиха се, отбелязаха я, но продължиха. След малко вътре угасна. Най-интересното бе откъде имат ток тия хора. Генератор бе възможност, но гориво за него? Тези мисли минаха бързо през главата на човека, но той не им обърна повече внимание.

Излязоха от селцето и продължиха направо. Минаха през една тревиста местност и след около половин час път стъпиха на асфалт. Бяха на дълга права, която беше част от едно малко летище. Продължиха по пистата към близък хангар. Нощите бяха студени и те предпочетоха да се установят на завет и да си напалят огън. Провериха помещението. Нямаше никой.

Хангарът представляваше доста голяма дупка, в която имаше разхвърляни части и един доста обезобразен самолет. Десетимата събраха разни платове и дъски намиращи се на пода и напалиха един не много топъл огън. Събраха се около него и сложиха караул от двама човека на входа. Приказваха си на общи теми. Не се страхуваха много, защото в този свят почти всички свободни бяха обединени. Единствените опасни фактори бяха дивите животни и някакви групички бандити – хора изгубили всякакъв морал и възползващи се от лесния живот, който този изоставен свят предлагаше. Обикновено разполагаха само с тояги и злоба, което бе опасно. Но десетимата нямаше от какво де се притесняват – бяха въоръжени.

Сменяха караула на два часа. Легнаха скоро след като се смени първият караул. Около два часа ги събуди тропот на тежки обувки и щракане на предпазители. Всички пъргаво се изправиха, взеха оръжията си и завариха как двамата караули бяха насочили автоматите си към пет човека осветени от тлеещия огън. Зад тях също бяха започнали да се задават други. Бяха въоръжени само с тояги, но бройката и безскрупулността им бяха важен фактор за изхода от тази малка битка.

Десетимата се наредиха в кръг и, разчитайки на плашещи жестове с оръжията си, започнаха да се отдалечават към най-близкия ъгъл на хангара. По този начин щяха да могат да се концентрират само от едната страна на бойното поле. За пет доста продължителни минути те успяха да стигнат до целта си. Огънят беше зад гърба на бандитите и това създаваше едни тъмни фигури без лица, без израз – като кукли.

Човекът пристъпи крачка напред.

"--* Какво искате? – попита той.

Онези се раздвижиха и си размениха някакви думи. Разнесе се подигравателен смях. Не последва отговор, а само две крачки настъпление. Групата на защитаващите се се сви още повече. Бяха въоръжени, но ги беше страх. Бандитите бяха наясно с факта, че са се изправили срещу хора, които до сега не бяха убивали.

"--* Какво искате? – повтори човекът. Този път се опита да звучи по-твърдо.

Кратко колебание и гъгнещ отговор:

"--* Вещите ви.

"--* Това не можем да ви дадем – отговори човекът.

Онези настъпиха пак. Човекът не помръдна. Беше на ръка разстояние от най-предния бандит. Онзи можеше просто да замахне и да вземе оръжието му, но не го правеше.

"--* Оставете ни – каза човекът.

"--* Защо? – запита онзи.

"--* Защото се опитваме да помогнем на света.

"--* Помощ? За какво му е на света помощ? Та, той си е много добре така – каза онзи. – Дайте притежанията си и ще ви оставим.

"--* И как ще ни накарате? Ние имаме оръжия, можем да стреляме и всичко да приключи за вас.

"--* Не бихте – отговори онзи.

Думите му бяха съпроводени от още едно малко настъпление. Човекът кипеше. Беше го страх, но и не смяташе да даде каузата си на тази група отрепки. Когато онзи се приближи още, той стреля в коляното му. Искаше само да ги сплаши, да отстъпят. Останалите въоръжени обаче се стреснаха и приеха сигнала за разрешение за стрелба. След минути шайката бандити лежаха прободени от поне двеста куршума, в локва кръв. Човекът гледаше безучастно. Току-що бяха убили хора.

"--* Хайде, нямаме повече работа тук – каза той.

Те събраха собствеността си и тръгнаха. Заводът се намираше край селцето Забуччя. Очакваше ги ден път и борба с недоспиването. Единствено можеха да се надяват да не срещнат други бандити. Пътят им минаваше през доста застроени местности. Десетимата оглеждаха внимателно прозорците и улиците. Естествено, не откриваха почти нищо. Тук-там имаше по някое злобно куче, но то не би скочило срещу толкова хора. 

Привечер стигнаха фабриката. Тя представляваше широка и дълга сграда с метален покрив. Беше висока около два етажа и имаше големи прозорци наредени по стените. Откъм широката страна имаше една голяма врата, до която имаше разбита дупка. Около самата сграда бяха разхвърляни боклуци.

Десетимата оставиха ремаркето отпред и тръгнаха към дупката водени от Човека. Един по един прекрачиха във вътрешността на сградата. Застанаха в отбранителна позиция и изчакаха няколко минути. Отчасти се убедиха, че няма никой.

Във вътрешността на предприятието, в три редици, бяха разположени конвейерни ленти. Конвейерите водеха към масивни барабани. След барабаните, отиваха в контейнери. Пред контейнерите имаше квадратни дупки с капак. Човекът предположи, че това са били пещи. До тях имаше празни калъпи. Явно оловото се e разтапяло в пещите и после се е отливало на кюлчета в калъпите.

"--* Хм, трябва да продължава нанякъде тая линия – каза човекът.

"--* Там има една широка врата – каза някой.

Наистина, малко отстрани имаше една дупка, през която можеха да минат трима човека наведнъж. Групичката се запъти към нея. Озоваха се в една стая с дълъг метален капак по средата. По релсите разположени по широчината му, личеше, че той се плъзгаше настрани за да се отвори. Отстрани му имаше един нисък пилон с конзола, на която беше разположен единствен бутон. Човекът го натисна, но естествено нищо не стана.

"--* Да се опитаме да го вдигнем? – предложи един от групата.

"--* Друго не ни остава – съгласи се човекът.

Десетимата застанаха откъм дългата страна на капака и започнаха жалки опити да бутат. Не използваната от години механика трудно щеше да се помръдне. Те направиха още няколко напъна, в резултат, на които само изскърца нещо, но нямаше и частица движение. Десетимата се изправиха с ръце на кръста и мислещи погледи. Човекът мерна един лост разположен до капака. Изглеждаше тясно свързан с механиката му. Той го хвана и тръгна да го дърпа. След няколко неуспешни опита осъзна, че е хванал грешната посока. Обърна се и задърпа. Един път – неуспешен. Увеси се на съоръжението, а пък то едвам измина няколко сантиметра. Дойдоха още няколко човека и с общи усилия направиха първото пълно дърпане. Вратата се помръдна с десет сантиметра. След още дузина-две повторения на това действие капакът беше достатъчно разтворен.

Под него съществуваше тъмнина. В самотното си съществуване тя бе станала студена и влажна. Десетимата влязоха с насочени фенери. Беше пълно с кюлчета олово. По средата на помещението дори стоеше един лифт.

"--* Ами, да прекараме ремаркето – каза човекът.

Натовариха колкото се може повече от ценния метал, а после легнаха да поспят.

Пътуваха през целия следващ ден, пропуснаха летището и легнаха точно до пътя. На следващия ден беше по-трудно. Този път направиха една малко по-дълга почивка по обяд. Човекът вече премаляваше да упражнява своите способности на водач. Негова беше отговорността тези хора да са мотивирани и живи. Въпреки, че той по-малко влачеше ремаркето, се чувстваше най-уморен. На следващата вечер преспаха в една къща, в едно малко село. Освен виещите чакали, друго не ги притесняваше. На третия ден, някъде към обяд, скапани от умора, прегладнели и жадни влязоха в базата. Там ги посрещнаха с усмивки и аплодисменти. Все пак, дори и в тези времена, хората знаеха, че трябва да празнуват победите си.

\section*{6.}

Беше решено оловото да се претопи и излее на тънки плочи. Всъщност, най-уязвимата част за пи-вълните бяха гърбът и главата. Хората планираха само там да сложат олово. Едно, че беше малко, друго че костюмите щяха да станат твърде тежки. Първият костюм щеше да се пробва от доброволец – един странен и постоянно усмихнат нещастник.

За целта на изработката бяха изкопали плитък трап по средата на плаца. Бяха го напълнили с дървета и ги бяха подпалили. След като огънят се беше разгорял достатъчно, сложиха въглища. Над тази скалъпена пещ разположиха един глинен, дебел съд, в който щяха да разтопят метала. След като разтопяха достатъчно количество, почваха да го изливат в плитки, квадратни, глинени форми. По този начин изработиха плочите. Наистина, получиха се груби и малко по-дебели от оптималното, но смятаха, че ще им свършат работа. Изляха всичко за около седмица.

Междувременно костюмите бяха съшити от стар брезент събран от селата. Те представляваха гащеризони с дълъг ръкав и два големи джоба на гърба, в които щяха да сложат оловото. Имаше качулка за главата с три по-малки джоба, в които пък щяха да поместят малките плочи.

Доброволецът облече костюмът. Тежеше му, но щеше да се справи. Целта му беше да влезе от една от западните ферми, да се поразходи из града и да излезе от същото място. Трима го съпроводиха до фермата. Оставиха го насред полето и се скриха в една малка горичка. Там щяха да го чакат два дни. След това щяха да си заминат.

Пронски, така се казваше той, се огледа наоколо. Пред него бяха едни високи крепостни стени. Той погледна сладките тревички около него, примлясна два пъти и закрачи към фермата. То ферма, не, ами плантация. Само дето не беряха памук, а гледаха свине. Сред красивата шумотевица от грухтене и каруцарски псувни, се чуваха приятни птичи песни, които изпълваха сърцето. Една самотна сълзичка се появи на лицето на Пронски. Той я изтри, изръмжа и закрачи.

Фермите никой не ги пазеше. Нямаше нужда, щото яйчицата си бдяха над мислите на своите поданици. Първото изпитание щеше да е това. Пронски влезе крачка по крачка, с напиращ адреналин застрашаващ сърцето му, и започна преспокойно да си върви сред полето. Той седна по средата и зачака да свърши смяната. Един работник го видя.

"--* Кво стаа? – попита го той.

"--* Чакам – отговори Пронски.

"--* Чекаш, начи?

"--* Да.

Онзи изсумтя и си отиде. Пронски се замечта за живота, който никога не бе имал, сети се за красивите еднорози, които бе сънувал предната вечер и се усмихна. Та, то сигурно би било много хубаво да си еднорог. Летиш си, правиш си магии, това онова. Но сигурно е и трудно -- живееш в света на измисленото, никой не вярва в теб, само те сънуват и разказват историйки. В крайна сметка, Пронски заключи, че не би искал да е еднорог. Може би птица или усмихнат пуловер, но не и еднорог. Определено не.

Така, както мислеше, изведнъж смяната свърши. Той стана, поотупа се и закрачи. Почти успяваше да имитира изтощението на трудещите се. Все пак, мисленето бе го напънало. 

Всички влязоха през една дебела врата с розови шипове. Отвътре чакаха двама въоръжени клоуни, които просто седяха. Яйцата май бяха някакви фетишисти щом аранжираха всичко да е толкова цветно. Вървяха по един дълъг коридор, който светеше в края си. Свободният излезе. Озари го невероятната красота на апокалиптичното щастие, което застрашаваше да се превърне в антиутопична утопия.

Градът беше весел, красив. Разноцветни сгради заобикаляха Пронски, охранители с костюми на клоуни, тигри, мишки, герои и шишковци пазеха, изкуствена дъга се изобразяваше в небето, а около нея летяха някакви същества. Беше невероятно. Нещастникът се спъна в един бордюр и това го освести. Вдигна глава и се оказа лице в лице с някакъв Мики Маус. Онзи го погледна равнодушно и го бутна да си гледа пътя. Пронски измънка нещо в извинение и си продължи. Реши да отиде към центъра. А къде ли беше тоя център? А, табела. На нея пишеше ``Център'' и стрелка сочеше към местоположението. Той почна да върви. Видя една самотна спирка и табела до нея. На табелата имаше едно число и редица надписи. Сред тия надписи пишеше ``Център''.

"--* Хах, тва е твърде лесно – каза си Пронски и зачака.

На шейсет и първата минута от операцията по висене на спирка дойде едно автобусоподобно нещо. То отвори врати и човекът се намъкна вътре. Имаше двама пазачи облечени като феи на зъбките и петима контролирани. Пронски седна и почна да брои спирките. На една от тях се качи още един пазач и реши да седне до него, пропускайки интересен разговор с колегите си. Тъжна капчица пот се стече по лявата скула на доброволеца. Сърчицето му леко претупа, а мозъкът му говореше да е спокоен. Онзи го погледна равнодушно и го потупа два пъти по дясното бедро. Пронски се отдръпна и сложи ръка на оскверненото място. Пазачът се усмихна, но не каза нищо. За щастие слезе една спирка преди центъра.

Скоро дойде и самият Център. Той беше толкова пълен и забавен! Беше пълно с усмихнати контролирани носещи разни балони, малки дечица пищящи и играещи, разни улични забавления, абе, беше рай. Пронски се учуди защо те се занимават да живеят в оная дупка извън тези градски стени. Помисли секунда-две и се сети, че хората трябва да са свободни. Не знаеше защо, но така трябваше да бъде. Мислопотокът му се форсира в друга насока и той продължи да ходи безцелно. 

Няколко часа разглежда и докато се осъзна беше станало време за ядене. Замириса му на манджа и ароматът го отведе в някакво кръчме. В него бяха наредени дървени, тежки маси с пейки. Имаше топла витрина, на която хората се редяха и поръчваха. Той взе една табла и награби две пържоли, една двойна зелена салата, половин хляб и две студени води. Качи таблата на главата си като негърка с пране и закрачи към най-близката маса.

Седна сред трима контролирани. Сети се, че е нужно да шпионира и се вслуша в разговорът им. През това време доста успешно мляскаше и унищожаваше порцията си.

"--* Днес великият Ргхтп ме похвали! – радваше се единият от тях.

"--* Еха! За какво? – попитаха го.

"--* Ами, в нашите лаборатории правим един серум. Пиеш го и намалява резистентността на марксистчетата. Иии става по-лесно. Много по-лесно\ldots

"--* Това значи, че радиусът на контрол ще се увеличи? – попита един.

"--* Да, определено. И яйцегените ще могат по-лесно да поставят контрол на някого. Не е ли прекрасно? – сияеше от радост онзи.

"--* Да! – отговориха в хор другите.

Пронски си дояде и стана. Следващата му цел бяха крайните квартали, като главното условие бе да не се загуби.

Той се върна на спирката, от която се бе озовал в центъра и погледна табелата. Последната спирка се казваше ``Конецн'', а преди нея имаше ``квартал Забатны''. Пронски реши да се движи до предпоследната. Този път автобусоподното нещо дойде на време. Той се качи в пълното превозно средство и застана прав, гледайки напред. Скоро обявиха неговата спирка и той се избута до изхода като в процеса опозна доста отблизо някои от пътниците. Стъпи на твърда земя и се оказа в нещо като гето. Само дето в това гето всичко беше здраво и изчистено. Просто сградите бяха сиви и мръсни, а хората гледаха тъпо в нищото. Той се запъти към първия видян за да го пита дали има някоя кръчма наблизо. Ония го изгледа и му каза, а после продължи по пътя си.

Пронски продължи по посока на рейсоподобното. Отляво беше булевардът. Той беше пуст и широк. От време на време по него минаваха други автобуси и полицейски коли. Отдясно му се намираше една зелена поляна. Толкова зелена, че контрастът ѝ със сивотата на околността я правеше сюрреалистична. Кръчмата постепенно се появи до него. Тя представляваше едноетажна сграда с повечко прозорци. Пронски се подвоуми, но влезе. Този процес беше свързан с отваряне на вратата, поглеждане и избягване на случайната чаша литнала към него. Тя се разби в уличната лампа отзад. Изглеждаше тук яйцата нямаха контрол. Пък и това беше квартал на неконтролирани, все пак.

В кръчмата беше светло. Имаше около десет маси пълни с хора и един полупразен бар. Барманът беше с някаква каубойска шапка, руса коса и сини очи. Типичен руски гамен. Пронски седна.

"--* Кво ше е? – запита го барманът.

"--* Кво имаш? – отговори нещастникът.

"--* Водка – след малко продължи, – водка с лимон, водка с мед, водка с лед, водка с уиски, водка с джин, водка с водка, водка с\ldots

"--* Една двойна водка с водка – прекъсна го Пронски.

"--* Окей.

Онзи хвана една висока чаша. Сипа три пръста от една черна бутилка и после още толкова от една бяла бутилка. Пронски взе и отпи. Питието имаше вкус на малки, черни дяволчета ръгащи с тризъбците си вътреустието му и после поливащи със спирт раните. Намръщи се, изхъска и после замечтано се усмихна. Барманът го погледна радостно и продължи да си лъска плота.

Сутринта беше сложна за героя. Той се намери на една пейка с усмихнатото слънце светещо в очите му. Някаква песен се завъртя в главата му, а после се почувства страшно гладен. Огледа се – ръцете му бяха леко подути и усещаше някаква синявица на дясната си скула. Носеше някакъв забавен гердан с цветни пера. За щастие костюмът и качулката му си бяха на място.

Бързо осъзна, че вече трябва да се връща. Покрай него минаваха все повече хора от неконтролираните. Той реши да тръгне с тях към фермите. Въпросът бе как да намери правилната ферма. Реши просто да хване същия автобус.

Стигна до спирката и поседя там, докато тя се напълни с доволно количество намръщени и сънени хора. Всички се натъпкаха в рейса и потеглиха. След един час Пронски беше на спирката, от която почна епичното му пътуване. Скоро пак се сблъска с Мики Маус, пак измънка нещо и мина със сформиращата се тълпа през портата с розови шипове. Озова се в тунела, а после на свобода.

\section*{7.}

Ония бяха заспали блажено. Пронски ги побутна, после им кресна в ушите. Изведнъж три автомата се озоваха пред нещастника. Караулите се почудиха, освестиха се и свалиха оръжията.

"--* И кво? Ше се трепем ли? – попита Пронски.

"--* Ми, следващия път карай по-полека – каза му един.

"--* По-полека\ldots Вие тука заспали на пост, а аз да карам по-полека\ldots Знаете ли през какво минах?

"--* Айде, айде, да вървим.

В базата го посрещнаха с радостни усмивки. Приятни възгласи прогласиха плаца, а някой дори се опита да изсвири някакъв марш на ръчно изработена свирка. Не му се получи и той я строши на две. Маркус, Игор и човекът тръгнаха да поздравят Пронски.

"--* Браво! – започна Игор. – Това беше силно героична постъпка от твоя страна! Ето, драги другари, такива подвизи всеки един от нас трябва да прави всеки божи ден – той натърти всяка сричка от последните думи. – Ние трябва да сме смели, напористи, героични! Това трябва да ни е в кръвта! Иначе, иначе, деца, нашата революция няма да успее\ldots Но, но тя ще ни спечели свободата, ИСТИНСКАТА свобода, защото ние сме смели, ние сме герои! А това – той посочи Пронски – е нашият първи доказал се герой. Благодаря ти!

Последваха бурни аплодисменти и възгласи. Всички се чувстваха една идея по-загрижени за световното дело. Маркус пристъпи и стисна ръката на Пронски. Пронски изквича наум, но се въздържа да покаже болка. Човекът го потупа по рамото и му се усмихна. Четиримата тръгнаха към главната сграда за да продължат обсъждането на следващите си действия.

"--* Така – започна Игор, – разкажи ни сега какво стана. Как е ситуацията там? Какви са възможностите ни за пробив?

Пронски им заговори с благ глас за всичките си приключения. За клоуните, за Мики Маус, за свирещия център, за претъпкания градски транспорт. Пропусна кръчмата и продължи с пазачите и контролираните. Една муха за малко взе интересът му, но той се стегна и успя да не изгуби нишката на разказа си. След като завърши, останалите замълчаха няколко минути мислейки.

"--* Значи пробивът ще е лесен – заключи Игор.

"--* А после? – запита човекът.

"--* Какво после?

"--* Ще стоим там и ще се правим на туристи ли? Нали трябва по някакъв начин да освободим града?

"--* О, да\ldots Е, ще ходим към центъра, ще разпитваме. Все нещо ще измислим! – удовлетворено завърши Игор.

\section*{8.}

Ргхтп медитираше целенасочено за да подреди чакрите си в съответствие със звездното небе от нощта на двадесет и първи светлинен континуум. Той беше оставил поданиците си да контролират за да може да получи просветление. Жълтъкът му се преобърна от една сцена, която му се яви. Ргхтп емитира силно възмущение, замря в тишина и очаквателно се свърза с един информатор.

Предадоха си изображения на опасност, малко притеснение и една яйчена мадама, която очевидно бе грешка. Ргхтп се скара на онзи, но продължи с изображения за повече информация. Онзи почна да му предава чувства, обоняние и звук за да подсили ефекта на притеснение. Главният яйцеген се замисли. Белтъчното му вещество не стигна и за това той прихвана още десет незначителни яйца. Засили пи-вълните си, ония започнаха малко да се пържат, но постепенно почнаха да му идват идеи.

Та, някъв грозен идиот се напил в кръчмата на Брузчев. В процеса на алкохолен делириум му разказал как идва от друга планета да разгледа тукашните обичаи за да събере информация за книгата, която пишел. Разказал му как на неговата планета имало тучни пасища и летящи овчици, които вместо мляко давали ирландско кафе. Разправил му за говорещите тортички и алманаха на пътеводецът. Говорил му за дългото море на сълзите, по което плували тъжни очи. Абе, общо взето, вкарал го във филма. Барманът, нали си е прост човек, му се доверил и му разказал за някои тукашни порядки. Ритуала пред храма на яйцата, яйченият марш, сутрешната белтъчна молитва, жълтъчните пости и черупчестите велики дни. Покрай всичко това му казал за тайната на града – центърът и как там винаги е толкова щастливо. Онзи се усмихнал доволен и продължил разговора на друга тема.

Информаторът сподели опасенията си, че това може да е бил някакъв разузнавач от свободните. Но не можеше да си обясни как той е минал през защитното пи-поле. Ргхтп изиска довеждането на бармана.

Двама униформени влязоха в спалнята на човечеца. Вдигнаха го и го завлачиха до един черен автомобил. Тикнаха го отзад и отлетяха за центъра. 

Барманът се изправи пред Ргхтп. По-точно се озова сред четири огледални стени, които го побъркваха с безкрайността си. Беше седнал на един стол и усещаше как вълните влизат в простата му съвкупност от неврони, които някои хора наричат мозък. Изображения от разговора проникваха в Ргхтп и той се убеждаваше в разказа на информатора. След като се убеди напълно, че нещо опасно е на път да се случи, той се спря на образа на Пронски. Запомни го много добре и обърна внимание на дрехите му – наистина първичен модел на оловен костюм. Ргхтп вече знаеше какво се случва. Пуснаха бармана с леко изменени центрове за държане на езика зад зъбите.

Изкомандвано беше всеки един хуманоид с подобни параметри да бъде воден директно при Ргхтп.

\section*{9.}

"--* Влизаме? – запита Очерников.

"--* Влизаме – потвърди Самарчев.

Двамата се показаха над високите класове ечемик и закрачиха сред неконтролираните. Самарчев леко трепереше от притеснение, а Очерников се опитваше да послуша сърцето си. Успоредно с тях влизаха още двайсет и пет двойки, от различни страни на Киев. Целта им беше да стигнат центъра и да намерят хранилището на яйцата.

Двамата минаха покрай пазачите и доволни се озоваха от другата страна на тунела. Повъртяха се малко и разпитаха за спирката към центъра. Един съвсем обикновен контролиран гражданин ги упъти да продължават направо. Те го послушаха и скоро стигнаха едно кубче стъкла отворено откъм пътя, с пейка в него. До кубчето се намираше табела указваща спирките на автобусите. Очерников, след кратко обстоятелствено разглеждане на табелата, заключи, че трябва да слязат след пет спирки. След половин час нужният автобус дойде.

Беше понеделник вечер – времето по което всички се прибираха към малките си домчета в желание да починат и да се подготвят за следващия ден на труд и горещо слънце. Самарчев и другарят му се натъпкаха като стриди в малкото пространство. Почувстваха се съвсем като рибни екземпляри, щото хем миришеше на морско, хем почваха да се къпят в потта си. Както и да е, те издържаха пътуването. Накрая за малко щяха да пропуснат спирката, щото трябваше да пробият стената от неинтересуващи се съпътници.

Слязоха и се отупаха.

"--* Кои са първите? – запита Очерников.

"--* Човекът и Славец -- Самарчев погледна часовника си. – Ние сме след пет минути.

Те тръгнаха към централната сграда. След пет минути стигнаха входа. Изпитателният поглед на пазачите ги срещна. Пуснаха ги. Все пак, на всеки беше разрешено да посещава сградата на яйцата. Човек без защита не би могъл да направи нищо на тези крехки създания. 

Всичките двайсет и пет свободни се опитваха да ходят нормално из залата с прозрачния под. Отдолу се виждаше голяма плантация с яйцегени. Свободните се струпаха на малки групички около пазачите. Когато човекът се увери, че всичко е подготвено, той направи знак за нападение. Пазачите се оказаха накичени от свободни. Те се предадоха с малко бой и ругатни. Двайсет и петте се изправиха на средата на залата. Човекът представляваше горда статуя на победител. Изведнъж нещо се стрелна към него и той падна. Последваха го останалите. Междувременно в трийсеткилометров радиус около Киев се провеждаше чистка на всичко човешко.

Така приключи киевската революция.

\chapter{Какво Ново?}

\section*{1.}

На планетата на Новото, залата с големия пулсиращ обръч бе заменена от цветисто обрисувани стени и мек под с голям килим. Килимът беше огромен. В самия него се развиваше цяла цивилизация, която един ден щеше да прерасне в супер сила. По стените бяха нарисувани абстрактни неща. Точно неща, защото трудно можеше да определиш дали бяха картини, статуи, музикални творби или просто изречения. По средата на залата имаше куб, а върху него паралелепипед. Едно гърчаво, но властно същество седеше изопнато отгоре му. То гледаше от прозрачните си клепачи и обмисляше какво да вечеря. В нозете му стоеше уплашено човече, което се опитваше да разгадае мислите на господаря си. Първия го погали с крак по главата, а после го ритна с пета, така че робчето заби нос във величествения куб.

"--* Искам сладолед с какаови пръчици и стриди. Ама да е вкусен. Искам и боб с люти чушки и две пържоли мариновани в аспержи. И червено вино. Не, нека бъде бира и пържени картофки. Два хамбургера с черен хляб и диетична кола. Две парчета осолена сланина и три парчета сирене. Това ще е май – завърши властващото същество.

"--* Добре, господарю – отвърна робчето.

То нещата, които господарят щеше да яде, не бяха такива каквито си ги представяме. Да кажем, пържолите за тях бяха едно странно калъпче протеини, което бе кафяво и леко хрупкаво. Сланината, както може би се досещате, бе бяла маса от мас. Въобще всичко беше толкова концентрирано и функционално, че удоволствието от храненето бе сведено до минимум. Въпреки това, господарят беше единственото същество на тази планета, което обожаваше да яде. Да си похапва, да си прилапва, да се нагостява. Но преди да продължа с особата на този цар, ще ви разкажа за това как той стигна до този си трон.

\section*{2.}

Един ден, една велика сутрин според календара на малките същества обитаващи тая планета, се роди Новото. Раждането бе съпроводено с ефирни преобразувания, рев и викове. След това Новото беше станало, беше се огледало и беше изкрещяло словата на народа. Настанаха три дни пируване, през които за малко да съсипят цивилизацията си.

Новото беше предначертаният владетел. То трябваше да обедини хороидите, които преди толкова векове са щели да се самоунищожат. То трябваше да премахне контрола на роботизираните зали и да ги замени с контрола на съзнанието, на собствената воля.

Един ден, обаче, се появи някакъв странен търговец. В дълбокия космос търговци имаше и то всякакви. Те пътуваха през пустошта на Вселената, събираха боклуци, а после ги продаваха. Там, където един предмет беше отпадък, другаде беше по-скъп от платина. Та, този търговец дойде по време на един слънчево-дъждовен период. Беше облечен в прилепнали панталони, термо-пуловер и газова маска. Съществата го приветстваха, нагостиха го и го попитаха от какво се нуждае.

"--* Да продавам – каза той.

"--* Какво продаваш? – попита го Новото.

"--* Власт и малки съкровища.

"--* Власт? Аз я имам. Съкровища не са ми нужни.

"--* Имате власт? Драги домакине, никой няма власт в този космос. Властта съществува само при мен и само аз я имам – каза търговецът.

Новото не се обиди от тези думи. То отбеляза факта и си извади хиляди поуки от него. Направи менталната връзка, помисли малко повече и реши, че е заинтригуван.

"--* Дръзко от твоя страна. Та, каква е тази власт, която само ти имаш?

"--* Нима искате да я видите? – запита го търговецът.

"--* Искам да я купя.

Онзи кимна и отвори кутията, в която помещаваше стоката си. Извади едно малко найлоново пликче, пълно с жълтеникав прах.

"--* Накарайте слугите ви да размесят това нещо с малко вода, докато стане на гъста каша. Всяка вечер третирайте носа си с него.

"--* Билки?! Билки\ldots Това ли е вашата власт? И за колко си я продавате?

"--* За едно приютяване при вас, на вашата планета.

Новото се замисли. Реши, че това което за него е безплатно, за други е недостижимо и реши, че няма как да има капан в тази сделка.

"--* Добре.

Първите дни Новото се мажеше с малко неохота. Веществото миришеше неприятно, оставаше жълтеникави петна по подносието му и постоянно го сърбеше. След първата седмица господарят почна да усеща как поданиците му се вслушват в неговите заповеди все по-почтително и настоятелно. След първите месеци той почна да вижда разцвета на империята си и планираше първия офанзивен поход. След втората година, той беше цар на Вселената\ldots Всъщност, лежеше на едно легло, бълнувайки в лудост и давайки команди. Търговецът седеше гърчаво на онзи трон, на който го бяха посрещнали и раздаваше правосъдие. Системата, която втриваше от жълтеникавото вещество беше автоматизирана. Никой, по никакъв начин, не смееше да я пипне и да я счупи. Гърчавостта и злият поглед всяваха страх не само у поданиците на този нов император, а у техните кости, кожа, сърца и души.

\section*{3.}

Този търговец управляваше малката си държава вече три века. Три века, съществата не бяха контролирани от пулсиращия обръч, а от злостта на хърбаво човече с нацупена усмивка. Казваха, че той бил магьосник, че ги контролирал с магия. Иначе как щеше да ги е страх толкова много? Той не бе убивал никого за неподчинение, но такова и нямаше. Той беше психопат – живееше за да манипулира. Осъзнаваше тъпите цели на света и се възползваше от слепия стремеж на хуманоидите да достигат тези висини.

Той не завладяваше други планети. Той дори не се показваше навън. Седеше си на трона, ядеше и контролираше. А ядеше много – тортички, пастички, бухтички, сланинка, месце, сосчета, калмарчета, това онова\ldots И не дебелееше.

За нещастие на този психопат, краят на трите века съвпадна с един от циклите на Новото. Е, не от ония цикли, за които сигурно си мислите. Неговите бяха много, много, много по-гадни, дълги и болезнени за намиращите се около него. Те го снабдяваха с толкова много агресия и адреналин, че то забравяше всички нужди и се концентрираше единствено върху рушенето. Ако беше свободен, щеше да се извърши ритуалът по заключването – затваряха го в титаниева клетка с диамантена армировка и го държаха, докато Новото не почнеше да говори смислено. Обикновено това траеше три дни.

Този път той не беше заключен. Търговецът обичаше да го гледа как лежи безпомощен и понякога си говори сам. Беше го сложил отстрани на трона си и понякога, когато никой не го гледаше си го галеше и му се радваше. Тъкмо беше посегнал да си поиграе с косата му.

Нещо рязко грабна ръката му, откъсна я от раменната става и я запокити в другия край на залата. Царят изпищя, при което се втурнаха пет от слугите му. Като се сетиха какво се случва, те излязоха, заключиха здраво вратата и се молеха Новото да не ги последва.

Агресиралият се продължи с рушенето. Той хвана машината, която го мажеше, изкърти я от стойката ѝ и се надигна рязко. Търговецът стоеше сгушен на трона си в локва кръв. Той отчаяно се опитваше да запуши всички артерийки, но не му се получаваше. Новото заби ръка в лицето му и го смачка в трона. После хвана цялата конструкция, надигна я и я тресна обратно. Търговец вече нямаше, а само локва кръв, вътрешности и смляно месо. Новото видя, че тук има много малко за рушене. Разочарова се, заплака от ярост и се втурна като ръгбист към вратата.

Удари я челно и припадна.

След три дни се свести и се огледа. Искаше жълтеникавата мас. Погледна машината и легна под нея. Нищо. Нищо не му слагаше от така желания наркотик. Беше изкъртил устройството. Намери го запокитено на няколко метра. Вътре имаше малко контейнерче, което беше празно. Една тръбичка водеше в скъсаността си. Той разбра къде е отивал скъсаният край и го проследи. Под земята беше. Опита се да засмуче нещо, но не му се отдаде. Опита се да свърже тръбичката и да пусне машинарията, но пак без успех. Няколко часа се повъртя безумно, мислейки, смятайки, плачейки и накрая се отказа. Седна до вече замирисващата пихтия и стоя загледан в една единствена точка. Борбата му продължи около седмица. Редуваха се ту яростни изблици, ту затишия.

Накрая, дишайки тежко, той почука три пъти на вратата. Едно същество му отвори и го погледна с широко отворени от страх очи. Новото се усмихна, погали го по главата и го избута настрани.

Беше събрала тълпа отпред. Те посрещаха освободения си цар. Истинския техен владетел. Не смееха да изразят радостта си преди той да е проговорил.

Новото ги огледа, усмихна се още по-доволно и каза:

"--* Това беше трудна битка за нас. Но се справихме – очаквателна пауза и аплодисменти. – Допуснахме грешка. Аз се пристрастих и позволих на един жесток владетел да ви управлява. Да, сгреших. Заслужавам най-лошото ваше наказание. Вие решавате дали желаете аз да диктувам бъдещето на тази планета.

Съществата затихнаха незнаейки какво да правят. Три века робство сломяват всякакво желание за свобода. Едно от тях си бръкна в носището и замислено сподели:

"--* Я мислим, че нема нужда!

Секунда мълчание и после овации на одобрение. Съществото бе избутано до царя и сконфузено стоеше забравило да си махне пръста от носа. Новото внимателно побутна ръката му надолу и то сякаш се поуспокои.

"--* Радвам се! – започна владетелят. – Нека сега възстановим загубеното и възвърнем мощта на тази велика планета!

\section*{4.}

Една вечер, докато се почесваше върху ръбестия си трон, Новото се беше замислило за странния сън, който го бе сполетял наскоро. През заспалия му мозък бяха минали видения за летящи домати с малки реактивни двигатели. Те пръскаха семки вместо реактивна струя и използваха за гориво водата, с която са ги били поливали. Понеже на тази планета нямаше растения, Новото много се учуди какъв беше този обект. Общо взето, в момента това влечеше мисълта му – как да кръсти нещото, което бе сънувал.

Изведнъж, така както си мислеше, забеляза как обръчът около стената започва да примигва в червено. Това значеше някаква опасност от доста високо ниво. Изискваше се то веднага да отиде в командния купол и да разбере какво по дяволите се случва.

Новото закрачи, почти изби титаниевата врата, която бе успяла да удържи наркотичната му абстиненция, и се качи в елеватора към купола.

Куполът беше прозрачна полусфера, по средата на която имаше голяма кръгла маса. Там се обсъждаха важните проблеми на цивилизацията и плановете за бъдещото ѝ развитие. Десетте главни се бяха наредили в очакване на единайстия. Новото седна и започнаха да го въвеждат в ситуацията.

"--* Яйце, господарю\ldots -- започна един. – Голямо, здр\-авословно жълто яйце, е засечено около пръстен М-9. Казват че има защитни системи и не отговаря на нито едно поискване за идентификация. Мислихме направо да стреляме, но решихме да получим и вашето тъй ценно мнение.

"--* Хм -- замисли се Новото, – щом не сте успели да установите някакъв контакт, значи място за преговори няма.

Това приключи дискусията. Всички станаха и се запътиха към боевата станция, където се намираше великото оръдие за сваляне на яйца от небесата.

Целият център беше един огромен дисплей, на който посочваш къде е обектът, който искаш да унищожиш, натискаш и чакаш. След около пет минути гледаш как врагът изчезва сред красив блясък от фойерверки, конфети и ароматни плодове. Наистина велика картинка. 

На Новото се падна честта да премахне яйцето от битието на простия човек. След като бъде изпарено, то щеше да бъде пратено в една алтернативна вселена, в която тези яйца властват. Ще го приемат като герой и то ще се чувства супер щастливо. Това беше една от много приятните черти на това оръдие за масово унищожение.

Обаче, тоя път, след фанфарите имаше яйце. То непоколебимо си летеше и се приближаваше към планетата. Новото изчака презареждането и избра отново обекта. Пак нищо. Ситуацията стана тревожна. Малки панирани човечета почнаха да кръжат из стаята, кипейки от разсъждения. След два часа безплодна мисъл, яйцето кацна. Новото излезе.

То се доближи до елипсоида и го погледна.

"--* И ся кво? Игрички ше си играем\ldots – започна намръщено Новото. – Казвай за кво си тук и после се омитай.

Тишина.

"--* Айде, де! Нямам време за твоите глупости. Кажи какво ти е, какво искаш?

Тишина и отчаяние.

"--* Виж сега, знам че нещата са трудни. Сам си, на чужда планета, не ти е леко. Но ако не ми кажеш какъв е проблемът, аз няма как да го разреша! Ако пък не искаш, просто си отивай, махай се! Айде, чиба. Марш!

Тишина, отчаяние и неподвижност.

"--* Окей, разбирам те. Сигурно са те отритнали от твоето общество. Нали? Абе, съществата са егоисти, квото и да си говорим. Не се впрягай толкова сега! Животът продължава, но трябва да си помогнеш сам. Ние съвети можем да ти дадем, но не и подслон. Особено пък, ако не ни кажеш какво искаш, какъв е проблемът. Сподели и ще ти стане по-добре. Хайде, де, хайде.

Новото погледна яйцето с най-милите очи, на които бе способно. Отново никакъв резултат. То спря да гледа, озърна се и взе един неголям камък. 

"--* Така, знам че е тежко да те отхвърлят втори път, но съм принуден. Махай се!

Камъкът тупна в яйцето и отскочи. На онова му нямаше нищо. То така и не се мръдна. Новото се позамисли, малко почервеня от гняв, но се успокои. За първи път му се случваше някой да не му се подчинява. Определено трябваше да реши този засрамващ проблем.

То се приближи до яйцето и го погали.

"--* Няма кой да те измъти ли? – запита Новото. – Горкото\ldots Сигурно искаш да създадеш живот, а няма кой да ти помогне. Да, тъжно е. Животът трябва винаги да бъде създаван, когато има възможност. Ами, да, аз за това имам толкова поданици. При нас децата са най-важните! Но хайде сега, как да те измътим? Хм\ldots

Новото се замисли. Сети се, че и те мътеха понякога яйца. Техните бяха малки и ги раждаха едни гущероподобни същества. Беше се оказало, че малките на тези същества са много вкусни, но пък невъзможни за лов, когато са под защита на майките си. Поради тая причина, ловяха яйцата, мътеха ги на топло и после готвеха малките. Наистина жестоко, но кралските особи имаха нужда от развлечения за да управляват качествено.

Новото викна двама помагачи с една количка и те откараха яйцето в инкубатора. Позатоплиха помещението и почнаха да чакат. След първата седмица висене до яйцето, Новото реши, че ще трябва да го остави само, да му осигури малко лично пространство и възможност за независимост.

На първия месец забелязаха първите пропуквания. На втория, една щастлива, огромна кокошка се разкарваше из инкубатора и кълвеше техниката. Къткаше си животинката и не и пукаше от живота. Новото се зарадва, че бяха подсилили стените на инкубатора миналата година. И все пак\ldots Все пак оставаше проблемът с това какво да правят с голямото същество окупирало тази стая. Можеха да го убият, но нима беше нужно? Нима можеха да погубят нещо дошло от дебрите на космоса точно на тяхната планета?

След една седмица Новото не издържа и се престраши да влезе при нея. Та, нима ще го плаши някакво същество високо почти колкото него? Та, той бе бил великани и забавни дървета!

Кокошката го погледна и тръгна към него. Тя въртеше главата си и постоянно го оглеждаше. Приближи се на човка разстояние и почна да мига с ципи. Изкътка и направи опит да си клъвне от него. Новото реагира веднага и се отдръпна. Кокошката пък се учуди и отскочи назад.

"--* Сега, знам че отношенията ни са обтегнати – започна то. – Ама, виж ти тука си дошла от космоса\ldots

"--* Аз не съм женска! – прекъсна го животното.

"--* А, ти говориш?!

"--* Май да\ldots Говоря ли? Говоря – птицата почна да кътка от радост и да мята с крила. – Говоря, говоря, говоряяяя!

"--* Хайде по същество\ldots

"--* Аз съм същество?!

"--* Да\ldots – кокошката го погледна обидено. – Имам предвид, не. Ти си важен представител на птичия вид!

"--* Така, така\ldots

"--* Защо си тук? – попита Новото.

"--* А ти защо си тук?

"--* Аз? Това е моето царство, моята планета. Аз съм предначертан да царувам тук вовеки и да водя тази планета към цветущо благоденствие. Моите подан\ldots

"--* Добре, добре\ldots Уууу – кокошката погледна някакво копче, – какво прави това?

"--* По-добре\ldots – кокошката го клъвна. – Не го пипай. Виж, нека се съсредоточим малко, става ли?

"--* Добре!

"--* Така, защо дойде тук? Защо измина това всичкото разстояние в някакво яйце? – попита Новото.

"--* Хмммм\ldots Не знам. Просто ей така си долетях. Трябва ли да има причина?

"--* Ами, ти си разумно, нали? – попита Новото.

"--* Да.

"--* Разумните същ\ldots – кокошката го изгледа. – Разумните действат с мисъл, нали?

"--* Предполагам\ldots Ти действаш ли с мисъл? – попита на свой ред кокошката.

"--* Да. Но да се върнем към темата – каква мисъл те доведе на нашата планета?

"--* Моята мисъл! – отговори кокошката.

Новото се отчая. Съществото пред него беше страшно глупаво и въпреки това имаше някаква цел, поради която се бе озовало при тях. За съжаление този обикновен подход нямаше как да подейства. Новото трябваше да смени тактиката.

"--* Ако не ни кажеш ще сме принудени да те унищожим – започна Новото.

"--* Защо? – изкудкудяка съвсем спокойно кокошката.

"--* Нямаме полза от теб.

В този миг нещо се пречупи в пернатото. Мъка обля сърцето му и гушата му се сви на топка. Малки сълзички се отрониха и се спуснаха по оперението на лицето му. То изкудка няколко жални пъти и спря вцепенено. Да те отритнат така, сам, на чужда планета. Твърде много бе. А то само си кудкудякаше и си кълвеше техника. Нима правеше лошо на някой? Нима заслужаваше да бъде унищожено? Може би да. Може би не струваше нищо\ldots

"--* Кът – разхлипа се птицата.

"--* Е, недей сега – Новото я приближи и я погали. – Сега, там ще си на по-добро място. Тук какво си? Нищо – кокошката изрида още по-силно. – Така де, сред нас си нищо, но за други може да си много. Но там\ldots Там ще имаш всичката храна, която искаш. Ще бъдеш сред други като тебе, ще си приказвате, ще си летите. Представи си колко ще е хубаво!

"--* Аз исках просто да живея на друга планета\ldots Ис\-кам да ЖИВЕЯ! Писна ми от тази Земя!

"--* Земя? – попита Новото.

"--* Да, Земя. Там се родих, но преди да се излюпя реших, че мястото е твърде прокажено за мен. Твърде много лошо имаше там. Твърде много ограничения. Аз съм птица! Аз съм свобода!

"--* Ето, начи си дошъл с мисъл – зарадва се Новото. – Кажи сега, разкажи ни за тази Земя.

"--* Няма да ме унищожите, нали? Ще ме уважавате, ще имате нужда от мен, нали? – запита кокошката.

"--* Да, ще те уважаваме и няма да те унищожим.

"--* Добре. Историята е дълга. Седни и аз ще седна, и ще ти разкажа.

Кокошката му разказа за поробената Земя. Новото отдели няколко минути за да осмисли казаното. Наистина интересен казус бе това. Наистина забавни неща ставаха по Вселената. И сега тая кокошка\ldots Било ѝ скучно в собственото ѝ царство\ldots Боже мили! Странни са съществата из космоса.

"--* Интересно! Доста при това. Може би тази планета трябва да бъде посетена. Тези яйца изглеждат интересни съдружници.

"--* Не – каза кокошката. – Нямаме желание за комуникация с чужди индивиди.

"--* Теб какво те интересува?

"--* Това, че съм избягал от там, не значи, че не обичам родината си и не проповядвам нейните принципи. Пък и те ще са ви малко големи.

"--* В какъв смисъл големи?

"--* Хората там са около два пъти по-високи от мен.

"--* Е\ldots Ние ще си кореспондираме с яйцата\ldots

"--* Казах, те не желаят да са в контакт с никого. Дори да отидете, ще ви унищожат, докато сте в орбита за кацане.

"--* Кой е казал, че ще летим до там? – запита Новото.

"--* Че как тогава? Но не, не ме интересува! Вие там няма да стъпите – каза кокошката.

"--* Нима? Иии кой ще ни спре?

"--* Достойнството да съблюдавате нуждите на една раса пред вашите.

"--* Нима вие сте били достойни като сте завладели човеците? – контрира Новото.

"--* А нима те бяха достойни да ни използват за храна?

"--* Вие не ядете ли някакви по-низши същества от вас?

"--* Ядем, защото трябва да живеем от нещо! – оправда се кокошката.

"--* Ами, те? Те не е ли трябвало да ядат нещо?

"--* Те ядат всичко! Те не подбират, те не се хранят за да живеят, а просто за да са щастливи, да си лигавят вкусовите рецептори с щастие. Няма извинение за тях! – тук кокошката тропна с крак и се опита да нацупи човка.

"--* Може би и ако на вас ви беше предоставена възможността и вие щяхте да ядете по много от всичко.

"--* Не, нямаше!

"--* Защо мислиш така?

"--* Защото сега имаме възможността, но не го правим.

"--* Добре, но сигурно това им трябва за да съществуват – каза Новото.

"--* Не ги оправдава. Убиват толкова много същества за да могат да живеят. Така, под наш контрол, е по-добре. Вселената става по-добро място.

"--* Кои сте вие, че да казвате това? – попита раздразнено Новото.

"--* Раса, основана на контрол. Нас никой не може да ни завладее – кокошката понижи тон. – Ние сме най-силната раса в тази Вселена.

Миг след това тя осъзна грешката си. Осъзна, че току-що бе прекратила живота си както на тази планета, така и в този Космос. Тя погледна Новото, но продължи да държи горно вирната главата си готова за най-лошото. Новото кипеше отвътре. Не можеше да се съгласи, да приеме тази арогантност. Хиляди мисли, бъдеща и развития на ситуацията течаха през главата му. Имаше много по-рационални действия от това да убие кокошката и да атакува расата на яйцата. Но имаше и чувства, които му пречеха да мисли трезво. Той беше господар тук, той даваше заповедите. Кимна на една от камерите и се замисли. 

Тъгата на смъртта пробягна през очите на пернатото. Та, то просто искаше да живее тихо и мирно, а сега беше мъртво. Е, може би след смъртта го чакаше нещо по-хубаво\ldots Място, където всички се обичат и се грижат един за друг\ldots Е, оказа се, че след смъртта има само чернота.

\section*{5.}

Една раса, която никога не бе участвала във войни, сега се подготвяше да атакува. Не, защото имаше нужда, а защото имаше принципи и накърнен господар.

Новото беше същество със страшно развити мисловни процеси. То можеше да желае смъртта на нещо и да я получи. Можеше също така да желае щастие и да го даде. Можеше много стига да искаше. И въпреки всички тези свои способности, то беше спокойно и радостно. Рядко щеше да извика, рядко щеше да се скара. Просто щеше да си каже един път какъв е проблемът. На втория просто те убиваше.

Проблемът, който имаха бе, че нямаха представа кого да атакуват. Кокошката бе умряла без дори да им сподели тази информация. Те останаха в тъжно неведение. На няколко пъти Новото се беше опитало да включи силата на своя неповторим мозък и да локализира яйцата. Почти успя. Общо взето, винаги, когато усещаше, че е близко до целта си, че яйцата са почти локализирани, то ги губеше. Това беше странно за него. Също така, беше малко страшно. Нещо му пречеше да стигне до целта си, а до сега никой не му бе пречил да постигне желанията си. Но нима същество, което никога не се бе страхувало, можеше да се притесни от това ново чувство? Може би не. Можеше ли човек, който е бил сляп от рождение, да бъде тъжен за недъга си?

След като Новото разбра, че няма да може да използва психиката си, се опитаха да трасират траекторията на яйцето довело кокошката. И там нямаха успех. Замислеността почваше да прераства в яд. Една тъпоумна кокошка и нейната империя от някакви си яйца! Те да го преметнат и озадачат така?!

От състоянието на замисленост го сепна един от умовете на планетата. Той беше малко, широкоплещесто човече. Доклатушка се до господаря си и почтително му заяви:

"--* Имам някакъв напредък.

Новото го погледна и се усмихна просто от учтивост. Вече не очакваше нищо\ldots Поредната лъжа може би. Поредното грешно изчисление, опит за решение на проблема.

"--* И? – запита Новото.

"--* Ами, кокошката има метаданни по перата си. Току що ги открихме. Адресни регистрации, ЕГН, трудова книжка, разни формати\ldots

"--* Метаданни, викаш.

"--* Да, данни за данните. Мозъкът носи данните, важната информация. Перата носят това откъде тя идва, накъде отива, за какво служи и тем подобни. Тепърва ги разчитаме. Та -- продължи човечето, – оказва се, че това е някакъв вестоносец. По-скоро разузнавач.

"--* Доста са му промили мозъка на тоя разузнавач – усмихна се Новото.

"--* Явно това им е в кръвта – каза човечето. – Както и да е, почти сме разбрали локацията благодарение на тия метаданни. Записани са на някакъв много древен техен език, който страшно си прилича на командите за пулсиращите обръчи.

"--* Интересно\ldots

"--* Наистина е интересно. Навежда нашите учени и на други мисли\ldots На мисли за произхода ни. Но това са други, ъм\ldots произходни въпроси. Та, до два дни ще имаме локацията им.

Новото беше доволно. Това беше най-тежката битка в неговия безкраен живот. Нямаше търпение да види какво ще се случи. То се усмихна, сложи крак върху крак и започна да барабани с пръсти по дръжката на трона си. След два дни щяха да атакуват.

\chapter{Омлети на очи}

\section*{1.}

Малкият Ерик си намираше забавни занимавки вкл\-ючващи всичко друго, но не и безгрижие за неговите родители. Извадил късмет да бъде от неконтролираните, той можеше по цял ден да се забавлява, докато чакаше да стане годен за тежкия физически труд, на който големите бяха подложени. Е, момчето не беше по-различно от връстниците си, но именно то стана част от променящата се история на света.

Вече незнайно колко векове бяха изминали откакто яйцата бяха успели да установят контрола си върху расата на човеците. Хората толкова бяха свикнали с това, че нямаха ни най-малкото желание да се освобождават. Та те си имаха всичко – сигурност, ъм, още сигурност иии малко щастие. Покрай тези съставки на тяхното битие, те се чувстваха добре. Е, това поне беше истина за жителите на яйчените градове. Извън тях сновяха полуподивелите свободни. Те си имаха култура и религия. Най-вече имаха обичаи и предания, които силно крепяха духа и желанието им за истинска свобода.

Та, малкият гамен се криеше зад една ниска, дървена постройка, в която родителите му складираха рециклирани дърва. И той не знаеше каква е заплахата, но нещо в детския му ум, му подсказваше, че това, което прави е добре. Изведнъж той видя звяра, който го заплашваше. Сърцето му затуптя по-учестено и момчето почна трескаво да мисли за следващата си точка на спасение. Погледна кулата на снайперистите. Тя беше висока, но разполагаше с лесно достъпна стълба. Вярно, никой не ти даваше да се качиш, но това беше въпрос на живот и смърт. Ерик не можеше просто така да се остави. Той погледна още веднъж чудовището и се затича стремглаво към стълбата. Почти беше стигнал, когато се спъна в нещо меко и за малко да падне.

"--* По дяволите – каза препятсвието с глас на летящ заек. – Карай по-полека, момче.

Ерик го гледаше втрещено. Онзи си разтъркваше главата.

"--* Ъм, извняай? – каза момчето и понечи да тръгне.

"--* Чакай, момче. Ти къде така си се разбързал?

"--* Ъъъъ, чудоиштето, чуек. Тряа го избегам – Ерик вече почти беше се изстрелял.

"--* Кво чудовище бе, момче. Ти нещо халюцинации, а? Я да те прегледам.

"--* Ааа, ненене, няа нужа. Аз така см си доре.

"--* Чакай, чакай. Искаш ли да ти покажа нещо.

Тук детското любопитство се предаде. Ерик се спря и се замисли. Какво имаше да губи? Най-много да се отегчи и да си се качи на кулата. Даже, чудовището май вече беше заспало.

Ерик се приближи към препятствието.

"--* Кво ше показвъш? – попита малкият.

Странникът не отговори. Бръкна в якето си и извади едно прозрачно кълбо. Повъртя го в ръцете си. Детето понечи да го пипне, но онзи си дръпна ръката.

"--* Сега, това не се пипа. Само го гледаш.

"--* Но ти ка го пипъш? – отвърна му Ерик.

"--* Аз мога да го пипам. Сега, искаш ли да видиш нещо? По-точно да не видиш нищо?

Момчето бе видимо объркано за това видимо-неви\-димо нещо.

"--* Ка така?

Съществото постави топчето между двама им и почака. Постепенно ярка светлина заструи от сферата. Тя отлиташе нагоре, където сякаш се удряше в невидим свод и се спускаше по стените му. Постепенно двамата се озоваха в кълбо, отвъд което не се виждаше нищо. Тревата около тях все още се полюшваше от вятъра, който идваше незнайно откъде. Някакви малки насекоми трептяха, а Ерик гледаше с паднала от кеф мандибула. Човечето изчака още няколко секунди, хвана сферата и я прибра. Всичко беше отново нормално.

"--* И ка така? – попита отново момчето.

Препятствието се усмихна и посочи към небето.

"--* Е, там има една звезда. Ти не я виждаш в момента, но тя те наблюдава. Там има страшни чудеса, с които да се забавляваш. Това е едно от тях\ldots

"--* И що си тука? Не ти ли е скука? – прекъсна го Ерик.

"--* Търпение, момче. Аз съм търговец. Много скъпи са моите играчки, ама ти си готино хлапе. Искаш ли да направим сделка?

Малкият предчувстваше как ще срази всички с вълшебната сфера. Една тънка лига заплашваше да се сформира в единия край на устата му, но той благоразумно я засмука обратно.

"--* Хм. Мириши туй – отговори той.

"--* Добре, аз ще ти кажа какво давам и какво искам за него. Ако ти мирише, няма сделка, иначе ти ще се сдобиеш с най-прекрасната сфера-невидимка. Тази тука дето ти я показах е слаба работа. Аз имам още по-хубава, която ще ти дам. Ама, ти трябва да ми кажеш нещо.

"--* Всичко щи кажа! – почти извика Ерик.

"--* Чакай, по-тихо. Добре, къде почитате вашите богове тук? Къде яйцата крият управата на този град?

"--* Много щеш! – опъна се момчето. – Туй са два въпроса, аз на един ше отговоря. Ейцата ги запитваме в центора. В оназ голямата сграда, дето има шип на върха. Дето стига облаците, и дето е мнооооооооого, много стара.

Странникът беше доволен. Той прибра прозрачното топче и остави друго, по-хубаво пред детето, а после отлетя. Ерик постоя няколко секунди и изтича да покаже придобивката на приятелите си. Чудовището отдавна беше забравено. Пък на кулата на снайперистите щеше да се качи някой друг път.

\section*{2.}

Смпт седеше на един от последните етажи на сградата, която едно време, преди мноооооооого години, са наричали Емпайър Стейт Билдинг. Градът беше Ню Йорк, времето по обяд. По улиците нямаше трафик, а метрото отдавна беше затворено. Ако човек погледнеше към пътищата щеше да види някой друг забързан контролиран, много въоръжени хора и двама клошари, които придаваха цвят и носталгия на мястото.

Този последен етаж, в който се намираше централата на този възел от яйчената невронна мрежа, беше оголен от всичко освен колоните си. По пода се бяха разположили яйца, сформиращи мислещото ядро. Смпт координираше действия и обработваше информация.

В момента пристигаше нещо притеснително. То беше съобщено на всички яйцегени по земното кълбо. Не идваше от никой по-главен. Първият, който бе усетил опасността, бе предал притеснението си на околните. Това беше силата на общия разум изградил тази нация.

Бяха засекли силно съзнание. Само веднъж, само за някакви части от секундата, в които то се беше разсеяло и се беше замислило за яйца на очи с бекон и кисели краставички. Но винаги тези моменти решаваха събития от световно ниво.

Действието в този случай бе инстинктивно. Всички неконтролирани бяха прибрани в жилищата си. Контролираните завардиха всички важни сгради и входовете на градовете. Яйцата се включиха на вълна да следят за външна опасност и оставиха мисленето за това какво се случва в града на заден план. В крайна сметка, вече бяха обучили контролираните да се грижат сами за сигурността на техния център.

Смпт мислеше. По-точно беше включил целия етаж в дълбоки, екзистенциални размисли за това как могат да се противопоставят на това евентуално по-силно съзнание. Яйцегените бяха много, но щом съзнанието бе успяло да се скрие от тях, значи не бяха достатъчно.

``Аспержи'' – прокрадна се отнякъде. ``Какво е това?!'' – възмутено запита Смпт. ``Не бяхме ние'' – отговори колективното съзнание. ``И малко доматен сос. Да!''. ``Това е то'' – заключи Смпт. Центърът се захвана с предаването на откритието. Щом са го хванали значи то се намираше някъде над щатите и то над Ню Йорк.

Скоро дойде обратната връзка. Съзнанието на Смпт се запълни с брътвежи, мисли и чувства. Всички предаваха, но никой не слушаше. Заплахата се намираше навсякъде. Яйцата бяха изправени пред съзнание многократно надминаващо техните телепатични способности. Те трябваше да се захванат със старите и почти забравени усилватели на пи-вълни. Ако не можеха мислено, щяха да го победят технологично.

\section*{3.}

Новото се намираше на любимия си плюшен трон с много джунджурийки. До него имаше ниска масичка, на която седеше поднос с кафе и два кроасана. Новото имаше трудности с тая масичка, щото понякога му се налагаше да си извие ръката за да може да хване чашката с кафе. Господарят някак си го преживя и спря да му обръща внимание след третата счупена чаша.

Пред Новото се намираше човек от вселенските разузнавателни служби за еквилибриум (ВРъСЕ). Той беше облечен с дълго палто от козина на рошав котарак. Имаше също така леки обувки, панталони с много джобове и тайнствена усмивка. Ама, направо бе изваян по шаблон на някой стар, изтъркан филм за шпиони от бъдещето. Той държеше в ръката си онази прозрачна сфера, с която бе показал на Ерик вълшебството на скритостта.

"--* Та, както виждате тук, господине -- започна той, – детето много ясно ми каза къде се намира тяхното главно управление за този град. На тази карта – той леко натисна сферата с пръст и тя проектира Ню Йорк гледан от небето – ето тази сграда е Емпайър Стейт Билдинг. Най-стария, все още неразрушен, небостъргач в града. Аз се разходих до мястото и го разгледах доколкото можах. Проблемът беше, че периметърът е строго охраняван. Общо взето, на двеста метра от сградата се стига без проблем. Всичко по-нататък е отцепено.

"--* Имаш ли някакви идеи къде се намират те? На кой етаж? – попита Новото.

"--* Не, но не мисля, че това е проблем. Ако сринем сградата\ldots

"--* Ако, ако\ldots Акото е за мен! – сопна му се императорът.

"--* Усетих лек психически натиск.

"--* Хм. Това е интересно. Не успяха да проникнат, нали?

"--* Е, разбира се. Просто го казвам, защото много малко същества са успявали да минат пи-шлема, който ползвам. А тези яйца, те за малко да усетят, че има нов елемент в системата им.

"--* Интересно\ldots Друго имаш ли да добавиш?

"--* Не.

"--* Свободен си.

Шпионинът кимна и излезе.

Флотилията на Новото не беше много голяма. Състоеше се от две дузини кораби, които бяха разположени над по-големите градове от тоя свят. Самите те не бяха бойни, нямаше нужда. Тази раса отдавна беше осъзнала, че умственият контрол е по-силен от което и да е оръжие. Пък и при положение, че бяха мирни, нямаха нужда от бойни кораби. Откакто се помнеха, не им се беше случвало да участват във военни действия.

Новото се беше замислило, когато при него влезе един от координаторите на нападението.

"--* Господарю.

Императорът се сепна, усети леко раздразнение, но се овладя.

"--* Сподели.

"--* Засечена е интересна пи-активност от страна на яйцата. Малко след като подготвихте обяда си е почнал непрекъснат брътвеж между центровете на управление. Почти нищо не успяхме да засечем.

"--* Хммм. Малко след обяда ми? А аз защо не съм го разбрал?

"--* Апетитните рецептори засенчват телепатичните ви способности, господарю – каза координаторът.

"--* Нима? Това е лошо.

"--* Лошо е, наистина. Както и да е. Забелязва се и активност от страна на контролираните елементи на това общество. Някакви странни съоръжения биват издигани на по-високите сгради. Забелязват се и някакви огледални повърхности, които ги нямаше до преди час.

"--* Хах -- каза Новото. -- Сигурно се опитват да ни заслепят. Това е абсурдно, координаторе! Занимавате ме с ненужности.

"--* Не, не са ненужности! Става въпрос за активност, която не сме забелязали до сега\ldots

"--* Ами, да сте гледали по-внимателно, де! Сигурно имат някакъв парад или нещо.

"--* Господарю, моля ви! Пробвайте да проникнете сега в някой индивид от тази планета.

Новото се ядоса, но и погледна критично на своята избухливост. Какво пък? Можеше и наистина да представляват някаква опасност. Гордостта убива чат-пат.

Императорът насочи мисълта си към земните същества. Долови, много слабо, съвкупност от голямо количество мисловни елементи. След няколко минути го заболя и то спря.

"--* Какъв е планът за атака? – попита Новото.

"--* Никакъв. Тези устройства изцяло променят условията. Мислехме просто да се възползваме и да обърнем техните контролирани срещу яйцата. Сега няма как да направим това. Нека говоря с останалите. Трябва време за да решим какви са новите възможности.

"--* Докато мине това време, може вече да няма нови възможности!

"--* Но ако прибързаме, може да пропадне всичко!

"--* Имате време, докато слънцето залезе над този град. В полунощ почваме атаката.

"--* Ама\ldots

"--* Айде, айде, че трябва да ям – в процеса на говорене Новото избута координатора през вратата.

Господарят се затвори и седна на пухения си трон. Добара една пържола от ниската масичка и я загриза. С блаженство усещаше наплива на благотворни протеини и мазнини в собственото си тяло. Отпусна се и задряма. Събуди го почукване на металната врата. Новото стана и отвори.

"--* Господарю, измислихме го! – каза му координаторът.

"--* Нима?! Хмм, доста бързо при това! Слънцето дори не е залязло все още! Какво сега? Къде ще ходим?

"--* Тука, тука! Сега ще ви обясня – почна координаторът. – Значи, има съпротива, на която можем да разчитаме или да ги подтикнем с малко контрол от наша страна. Те са отвъд градовете, недостижими от оръжията на яйцата. Та, хващаме ги тях. Те ще бъдат примамката. После, докато яйцата са заети с тях, ние ще нападнем ментално. Целта са сградите-центрове. Успели сме да ги открием за всеки град. Установим ли контрол над тях, всичко става лесно. Останалото е въпрос на доунищожаване.

"--* Хубаво – каза Новото. – Смятаме да разрушим всичко?

"--* Да, няма смисъл от такава назадничава раса.

"--* Хм, може да имаме полза от тях. Замислял ли си се за космическа империя? – в очите на Новото пробляснаха искри.

"--* Нека не прибързваме господарю. Между другото, скоро наближава края на поредния ви жизнен цикъл. Какво ще правим ако още сме тук?

"--* Като му дойде времето, ще мислим. Сега, да почваме атаката.

\section*{4.}

Спонтанността на запаления лагерен огън бе забелязана от свободния Питър. Той седеше и се любуваше на една стара и почти неразбираема снимка от някакво списание за отегчени мъже. Изведнъж нещо беше увеличило осветлението в малкото кътче, където се беше усамотил. Той надигна поглед и видя огъня. Остави предмета на своя интерес и стана. Поогледа огнището, опита се да го пипне, но се опари. Хвана наранената си ръка и почна да я милва, псувайки. Нещо го докосна по рамото и той подскочи.

"--* Спокойно, де – заговори го едно нисичко човече.

Питър се огледа, видя едни големи, сини, блестящи очи и се мирна.

"--* Какво? Страх ли те е? Ех, ех, ех\ldots Голям човек пък се страхува от едно джудже.

"--* Кво си ти?

"--* Виж ся, аз съм си моя работа кво съм. Ама, идвам да ти дам за да ми дадеш.

"--* Ъ?

"--* Добре – продължи човечето. – Тук ви мачкат, нали?

"--* Ъъъ? Няма мачкане тука. Само си живуркаме – отговори му човекът.

"--* Ама робството не ви ли е гадно от него?

"--* А, то не ни касае. Що да ни касае? Никой не ни бий, никой не ни гони.

"--* Ама, какво е живот без свобода, човече? Какво е да не изпиташ волността на душата под светлото небе? Какво е да не можеш да достигнеш всичко, което желаеш? Нима живееш? Нима си щастлив?

"--* Дъа. Що да съм нещастен? Огиня топли, животните ранят, водата пои, децата се праят.

Тук ниското същество постоя учудено. Задачата му щеше да е по-трудна отколкото очакваше. То потърка ръце и продължи опитите си.

"--* Е, добре. Ама, не ти ли е скучно, човеко?

"--* Скучну? Кой туй? Щом ни гу знам, начи сигурну не ми й.

"--* Ела да поседнем, искам да говорим, интересен човек си ти.

"--* Оф, що? Аз не ми се сиди. Искам да ям и да ходя нанякъди – опъна му се Питър.

"--* Виж, ако поседнеш малко, ще ти дам нещо.

Тук човекът го погледна съмнително. Съществото извади една малка чашка от бистър кристал и я запокити към един камък. Тя го удари и отскочи обратно при джуджето. Питър посегна към нея.

"--* Не, първо ще седнем и ще си поговорим – спря го джуджето. – Е, не ми прави такива мили очички сега. Всичко си има ред.

Седнаха.

"--* Кажи ся – подтикна го човекът.

"--* Аз съм Джинимир.

"--* И ко?

Джинимир се замисли за тежката си съдба и проклетата мисия, която го прати на тая забутана планета.

"--* Ами, като за начало, как се казваш ти?

"--* Питър.

"--* Добре, Питър. Радвам се, че се запознахме. Сега, представи си един свободен свят.

Питър смръщи вежди, но не успя да доведе някакви смислени картини в съзнанието си. Все по-тъжно му ставаше, че не е със своята избеляла снимка на онова момиче.

"--* Нимоа – каза той. – То ква е разликътъ от тука?

"--* Можеш да се разхождаш навсякъде – започна Дж\-инимир. След малко се сети. – Представи си да имаш нещата, които яйцата притежават. Хиляди такива списания, дори движещи се. Истински!

"--* Хм!

"--* А виж, ще живееш във високи сгради, ще ядеш месо, вкусно, печено! А не тия сурови мърши!

"--* Хм! И как ше стани тва? – запита Питър.

"--* Просто трябва да свършиш едно нещо ти и всички около теб. Тази нощ, след като звездите станат ярки, а луната се приплъзне в зенита си, да отидете и да кажете на тези яйца, че си искате градовете обратно. Просто отивате до стените и почвате да викате.

"--* Те шни убият. Шни зъболи главътъ и ше плачем.

"--* Няма. Ние ще ви пазим. Ела сега, ще ти покажа.

Джинимир го хвана за ръката и го поведе. Нямайки време да се осъзнае, Питър тръгна след него. След около час бяха пред стените. До тях се виждаха още много хора подхванати от джуджета. Тих глъч се чуваше.

"--* Ето, тук сме – започна Джинимир. -- До сега да е почнало да те боли, нали? А сега не усещаш нищо. Това е, защото аз те пазя. Когато стане време, вие ще отидете, а ние ще ви пазим.

"--* Що помагате? – попита Питър

"--* Защото ни е грижа за вас. Ние сме раса, основана на мира и животолюбието. Целим да избавим всички светове от тиранията и потъпкването на свободата. Та, имаме ли сделка? Вие пристигате, когато луната е в зенита си.

"--* А вий къде шъ сти?

"--* Скрити зад дърветата. Ще ни видите като минавате.

"--* Хм!

"--* Ето чашката. Тя също е магическа. Божествена бих казал. А аз тръгвам.

В миг се изпариха всички джуджета. Хората останаха сами. Те разглеждаха талисманите, които им бяха оставени. Някои ги потъркваха, други ги почукваха, трети си ги прибираха на сигурно в джобовете. Оставаха два часа до атаката.

\section*{5.}

След едно подозрително смущение на контрола им, яйцегените бяха още по-притеснени. Към полунощ, те усетиха доста силно стълпяване на съзнания около градските стени. Това стълпяване не изглеждаше да е от същества със слаби мозъци. Смпт започна да предполага, че атаката е започнала. Първото, което направи бе да изпрати група контролирани към градските стени. Веднага след това той се концентрира върху това да следи за други психически заплахи около града.

Градът разполагаше със сто плантации, което значеше сто входа. От всеки излязоха контролирани и насочиха оръжията си към свободните. Питър се уплаши. Онези им бяха обещали защита, а сега ги нямаше. Скрили се бяха. Страхливци. Значи, в крайна сметка, щеше да го заболи? Не, трудно. Онези се колебаеха.

Смпт усети ново колебание в пи-полето си. Този път около стените. Нещо се опитваше да отнеме контрола върху пазителите му. Той веднага нареди включването на усилвателите на пи-полето. Ако можеше да чува, щеше да разбере за последвалите изстрели. 

Междувременно Новото беше задействало паралелна офанзива. Хиляди от неговите войски се спускаха към града, готови да упражнят контрол над всяко същество, което се намира там. Секундите разсейка на Смпт и на другите управители на центрове, беше спомогнала за безшумността на атаката.

Смпт усети нов прилив на психически натиск. Той беше неочакван и в пъти по-силен от предишния. Чудейки се какво да прави, той сля всички яйцегени в едно общо съзнание. Контролът беше отслабен, но не достатъчно. Вече усещаше как губи някои от контролираните. Той призова всички останали да се съсредоточат в небостъргача, където му беше най-лесно да им влияе. Това бяха наистина отчаяни последни вопли. Краят на яйчената империя се виждаше.

\section*{6.}

Настъплението беше започнало отвън към центъра. Един от екипите бе този на Джинимир. Съществата бяха седем и си пробиваха път през един от работническите квартали на Ню Йорк. Сега бяха пред една от сградите на отдавна забравения милиардер Тръмп. Тя представляваше висок, бароков хотел. Целта им беше да минат през нея, защото там беше едно от оръжията на яйцегените.

Седемте влязоха през парадния вход. За сега не усещаха никакво психическо присъствие, но смътно долавяха мислите на няколко изгубени и уплашени съзнания. Най-големият им проблем беше усилвателят на пи-вълни, който заплашваше да ги повали, ако дори за секунда се разсеят от задачата си.

Озоваха се в огромното фоайе. На рецепцията имаше усмихната статуя на човек, която предлагаше кола. Капитализмът бе афектирал и яйцата. Седемте решиха да хванат асансьора. На тази завишена гравитация въобще не им се изкачваха стълби. Обаче той побираше само по пет човека. Разделиха се на две като Джинимир и още един влязоха във втория. Очевидно тъпо действие бе да остави по-голямата част от екипа си без него, но той забеляза това, чак когато асансьорът спря между десетия и единайстия етаж.

"--* И ся кво? – запита Рамби.

"--* Я натисни тва копче – Джинимир му посочи алармата.

Изви се пронизителен писък, който принуди и двамата да си запушат ушите.

"--* Натисни го, удари го, спри го! – изкомандва Джинимир.

Рамби се паникьоса, завайка се и почна да удря по таблото. Накрая откърти копчето и алармата успя да спре. Двамата запъшкаха и се хванаха за сърцата.

"--* Оф, Рамби, Рамби\ldots Дай да видим дали не може отнякъде да се промъкнем.

"--* Там отгоре има някаква решетка – забеляза Рамби.

Джинимир се качи на раменете му, а после застана на дланите на ръцете му. Те представляваха комична гледка на люлеещ се стълб. Ръката на командира се клатеше и имаше огромни трудности с избутването на решетката. В крайна сметка той успя да я премахне, хвана се за ръба и се изкачи.

"--* Слушай, Рамби, аз не мога да те стигна, ама сега ще отидя и ще отворя асансьора.

"--* Ама, ама\ldots – притеснено почна онзи. – Сам тука? Ако ме открият?

"--* Сега, толкова висш организъм като теб да се притеснява от някакви контролирани същества. Айде, моля те.

Джинимир захлопна отвора и се огледа. Беше в една типична шахта на асансьор. Той беше застанал с лице към плъзгащата се врата. Отдясно беше шахтата на другата кабина.

Джинимир се пробва да отвори вратата, но не успя. Нагоре имаше страшно много, а надолу нямаше как да слезе. Опита се да се свърже с останалите от екипа. Долавяше само слаб сигнал. Те бяха твърде високо. Все пак излъчи няколко вълни на тревога с информация за положението си.

Отвори капака.

"--* Рамби, абе, май няма да мога да се измъкна. Няма никакви отвори.

"--* Хм\ldots

Рамби се облегна на копчетата. Нещо друсна и асансьорът почна да слиза. Джинимир отскочи нагоре и се ориентира към кабината без да иска. Почти падна върху готовия на саможертва Рамби. Анасньорът пак спря. Чу се и сигналът за излизане. Вратата се отвори. Няколко дула ги чакаха пред нея. Преди да имат възможност да стрелят, Джинимир ги накара да влязат при тях. По време на тая операция, двамата контролиращи се измъкнаха и натиснаха копчето за партера. Контролираните се шашнаха, но вратата се затвори преди да реагират.

"--* Трябва по-често да мислим за простите неща от живота – сподели командира.

Рамби само сви рамене. Имаха още доста да вървят до последния етаж. Малките крака и тежката гравитация ги измориха още преди да изкачат и половината стълби.

"--* Оф, чакай, чакай – изпъшка Джинимир. – Не мога\ldots

Той тупна на едно от стъпалата.

"--* Рамби, колко остава още?

"--* Абе, май толкова\ldots Долавям ги по-близко, но пак са далеч.

"--* Оф, оф, оф.

"--* Айде, шефе! Айде да тръгваме, че окъсняхме. Вече десет минути се тътрузим, ще изпуснем времето!

Джинимир изпъшка, Рамби му подаде ръка и двамата тръгнаха. След двайсет минути стълбите свършиха. Озоваха се на една платформа, пред която имаше врата. Рамби я побутна и надникна през нея. Нямаше никой.

"--* Хм, трябва да са някъде зад нас – каза Джинимир.

"--* Да излезем?

Командирът поведе, блъскайки вратата. Зад един ъгъл се показаха останалите петима. Те бяха седнали на приказка и наблюдаваха епичността на оръжието. Джинимир кипна и закрачи към тях.

"--* Е, кво е тва сега! – започна той. – Висите тука като цифки половин час. Ние се мъчим, потим се да стигнем по-бързо, а вие -- нищо! Не ви е срам! Като се завърнем, лично ще ви докладвам на Новото! Амаха! Айде, айде, айде! Да го махаме тва нещо оттук, че атака има да се прави! Яйцата още са живи.

"--* Чакай, шефе! – отвърна един. – Ние имаме реч да държим!

Джинимир жестоко се разочарова, подтисна гнева си и реши да ги остави да правят каквото са си наумили. Все пак, колко дълга може да е една реч?

\ldots

На двайстата минута той не издържа. Отиде, зашлеви шамар на говорещия и изтръгна контролното табло на оръжието. Почуди се нервно, позачерви се доволно, но накрая се предаде.

"--* Инженера, я ела тук и го изключи тва.

Инженерът се приближи на почетно разстояние от Джинимир. Командирът усети тъпия страх и отстъпи. Онзи поогледа таблото, дръпна червения кабел и върза зеления на негово място. Нещо изпращя и антената на усилвателя клюмна.

"--* Готово – промълви инженерът.

"--* Доообре! Айде, да вървим, че ни чака още цял квартал за преминаване, докато стигнем центъра.

\section*{7.}

Картите бяха изключително практикуван спорт сред контролираните. Яйцегените така и не бяха успели да изтръгнат този вековно внедрен навик у смъртните същества.

По, Ко, Го и За играеха белот в една малка стаичка, намираща се някъде из паркинга на Емпайър Стейт Билдинг.

"--* Терца майорна – лукаво се усмихна По.

"--* Да ти я навра отзад тая терца! – измрънка Ко.

Играеха на Всичко коз. По хвърли едно вале спатия, Ко премахна последната си слаба карта, Го наля с едно асо, а За изпсува и хвърли девятката си.

"--* Ха, вътре – каза Го след като преброи картите.

"--* Е, ше видиш ти вътре! – извика За. – Я дай да преброя!

"--* Е, как ше броиш ти! Вие викахте, ние броим! Да си ги нямаме тия!

"--* Дай тука тия карти! – каза през зъби За и изтръгна тестето от ръцете му.

След около минута броене каза.

"--* Вътре сме.

Ко обърна масата и замина да пие една студена вода. За излезе на патрул, а По и Го прибраха печалбата и размесиха картите за следващата игра.

След десет минути дойде За.

"--* Ше почваме ли?

"--* Чакаме Ко.

"--* Баси, много дълъг патрул направи нещо тоя\ldots

"--* Ми, натискат го напоследък повече. Не бяха много доволни, когато оная котка влезе и счупи китайската ваза на Смпт.

"--* Горкичкия.

Когато мина половин час, те вече бяха доволно разтревожени.

"--* Хайде, да го търсим. Кой знае\ldots Сигурно пак яде пасти зад кошарата.

Тримата взеха оръжията си и тръгнаха. Паркингът беше голямо място, с малко коли. Той даже си беше безполезен, откакто яйцата спряха всякакъв вид личен транспорт. Въпреки това, трябваше някой да го пази, защото беше най-уязвимата точка на небостъргача.

Стаичката, в която играеха карти, се намираше близо до входа. От нея те тръгнаха към вътрешността на помещението. Всичко беше добре осветено и поради тая причина нямаха проблеми със забелязването на някой случаен предмет. По и Го тръгнаха към двете срещуположни стени, а За тръгна да обикаля дълбокото, празно пространство. Скоро почти се изгубиха от поглед.

За си откъсна една малка тревичка някакси провряла се покрай една от колоните и пода. Той я задъвка и безгрижно почна да се оглежда. Както си ходеше нещо го побутна. Сърцето му затуптя и той се обърна. Едно джудже му се усмихваше.

"--* И кво искаш ся? Айде махай се преди да съм те смачкал -- каза За.

Следващото нещо, което усети бе освободените си от контрола на яйцата сетива. Почна да мърда свободно ръцете си. Усмихваше се. Някой го потупа по рамото. Той се обърна, това беше Ко.

"--* Ко, ама, ти движиш ли се?

"--* Да, човек! От малък не съм се чувствал толкова свободен! Някъв дребос\ldots – човечето го погледна. – Ъъъ същество ме потупа, аз се обърнах и то ми се усмихва. И, и, и. И после аз се движа. Говоря си каквото мисля, сгъвам ръце, крака!

"--* Същото и при мен! Баси!

"--* Хей -- каза човечето, – айде стига толкова бръщолевене. Имаме работа да вършим.

"--* Каква пък работа имаме да вършим? – запита Ко.

"--* Да не мислиш, че тая свобода идва без нищо! Яйцата трябва да изкореним. Ако не бяха те, нямаше да ви помагаме.

"--* Ама\ldots – с разбито сърце почна За.

"--* Няма ама. Това е светът на големите. Има приоритети, всеки гледа да е щастлив. Поне тука така са ви учили, де\ldots Трябва да почитаме обичаите ви.

"--* Добре, кво ше правим?

"--* Да отидем към елеватора и да се срещнем с останалите.

Там бяха По и Го. Четиримата се запрегръщаха, викаха и се радваха. Четирите човечета ги сгълчаха и По, Го, Ко и За се умълчаха.

"--* Асансьорите не работят – промълви За.

"--* Нима? Защо?

"--* Ами, щото по тях лесно може да се стигне до яйцата. Преди два века решили да ги спрат. Дори да ги пуснем, сигурно ще пропаднат в бездната отдолу.

"--* Уф\ldots И по стълбите ли?

"--* Ами, да.

"--* Ще ни носите! – изкомандва едно от човечетата.

"--* Аааа, да си ги нямаме тия! – ядоса се За. – Не стига, че ви помагаме да убиете яйцата, ами и ще ви носим. Айде, де. Това тука да не ви е бащиния!

За усети лека болка в черепоушието си.

"--* Добре де, извинявайте! Все пак ни осовободихте.

Болката спря, а той облекчено почна да се разтрива.

"--* Ще видите вие! – закани им се той.

"--* Стига За! – каза Ко. – Хайде, да тръгваме.

Те хванаха човечетата и ги сложиха на гърбовете си. Очакваше ги доста катерене. Колкото по-нагоре отиваха, толкова повече освободени тръгваха с тях. Наближавайки етажа с яйцата, човечетата усещаха все по-голяма трудност да пазят от контрол освободените. Постепенно разбираха как като стигнат на етажа с яйцегените, щяха съвсем да ги изгубят. Точно сега им трябваше силата на Новото. Те се опитаха няколко пъти да се свържат с него, но без успех.

\section*{8.}

Междувременно Новото се кипреше. То имаше желание да е перфектно за вечерта на победата. Обу си изящни, сребристи пантофки. Върху мускулестото си тяло сложи перлена туника. На средния пръст на лявата си ръка се сдоби с платинен пръстен, инкрустиран с диаманти. Дългата си коса, пълна с обем, върза на плитка, която много прилежно прекара през едната страна на врата си. Сложи малко пудра и избели зъбите си.

Беше готов. Само едно смътно, гадно чувство се прокрадваше. Чувството на гняв и желанието за гибел. Скоро щеше да настъпи цикълът. То трябваше да бърза.

Спря се за минута и започна да се чуди какво щеше да прави. Съзнанието му вече не бе окупирано с нищо друго и това му помогна да хване сигнала на войниците си.

"--* Мамка му! – изрева императорът. – Нищо ли не могат да свършат тия като хората?!

Повъртя се на едно място, отиде към вратата и излезе.

"--* Синапс! Синапс! Къде си бе?!

"--* Тук съм, тук съм, господарю – отговори слугата.

"--* Веднага искам совалка към земята. Онези не могат да справят се едни яйца\ldots Ами по другите градове ако почнат така да ме викат? Къде ще му се види краят? Аз да не съм дядо Коледа, да обикалям тука из света за една нощ\ldots

Говорейки, двамата крачеха към мястото за излитане.

"--* Чакай, чакай – каза Новото. – Я да видя другите как се справят.

Минутата мълчание мина бързо.

"--* Слава богу, че техните генерали са се взели в ръце\ldots Айде, кво чакаме сега? – сопна се Новото.

"--* Веднага, господарю.

Синапс отвори вратата. Новото влезе в тясната кабинка, сложи предпазните ремъци и се приготви.

"--* Пазете се, господарю – каза Синапс.

Новото само изсумтя, затвори кабината и изчака Синапс да излезе за да изравни налягането в помещението. Когато всичко стана готово, то натисна едно голямо червено копче, на което пишеше ``Изстреляй се''. Совалката пропадна под кораба и се запъти към Ню Йорк. Новото имаше предостатъчно време да се замисли за екзистенциалния смисъл на всички тези звезди. Той ги преброи, изчисли разстоянието до тях, на базата на звездната им величина и успя да направи карта на това небе, спрямо родната си планета отчитайки посоката, от която бяха дошли.

Те се намираха в тази галактика – Млечен път. Същата като земната. Все още никой не бе успял да прекоси разстоянието между две галактики, колкото и близо да бяха. Но един ден, надяваше се Новото, щяха да го направят. Сега само слънчевата система и родната му планета бяха негови. След два цикъла планираше да е господар на галактиката\ldots Планираше да има поданици, да му се кланят, да няма армия, която да може да му се противопостави. Планираше да съчетае психическата и физическата сила в едно непобедимо състояние. Но и се притесняваше. Щом той се бе родил в тази галактика, нямаше ли шанс да има друг като него? А какво щеше да стане, ако отиде в друга система, по-далечна, по-силна? Щеше да се бори. Той бе непобедим.

От мислите му го изтръгна заходът за приземяване. Совалката кацна в подножието на една висока сграда, на няколко пресечки от небостъргача. Новото излезе и се отупа. Огледа с удоволствие своята земя и закрачи.

Около него често се случваше да се намерят разни индивиди. Повечето контролирани. То просто ги приспиваше. Ако човек беше в този район в момента, щеше да види как нещата просто заспиват в някакъв радиус около Новото. 

Господарят стигна до входа на небостъргача. Бутна вратата и влезе в хубаво осветено фоайе. След секунди се озова пред асансьора. Натисна копчето, но нищо не стана. Натисна го още няколко пъти в яростта си, но пак нямаше успех. Въздъхна и се запъти към стълбите.

Докато се изкачваше, то усещаше все повече яйченото присъствие. Яйцата наистина имаха способности, но те бяха незначителни спрямо неговите. Вече се чудеше и как са се уплашили поданиците му. Та, нима едно толкова слабо и неразвито пи-поле, можеше да свърши нещо? След около трийсет минути катерене, стълбите свършиха. Новото се поспря пред вратата, която водеше някъде си. То я отвори и се озова в една голяма, остъклена зала, от която се виждаше целият Ню Йорк. Господарят се разходи из нея и се опита да засече пи-полето на яйцата. Усети, че то бе два етажа по-надолу. Поколеба се малко, защото два етажа по-надолу нямаше врата. После мерна една врата от срещуположния край на залата и реши да я пробва. Пред него се откри някакво стълбище. Новото заслиза и скоро наистина стигна до врата, която се намираше два етажа по-надолу. То я отвори и се озова пред поданиците си.

Човечетата го посрещнаха и преди да има време да им отговори, го въведоха в етажа. Той имаше много стаи и два коридора, пресичащи се на кръст. Всички се бяха разположили в една централна конферентна зала. Седяха и гледаха тъпо.

"--* Какво става сега? Къде са тези яйца? – запита Новото.

"--* Ами, господарю, не знаем\ldots Усещаме ги страшно близо, но няма и следа от тях.

"--* Хм, а другите екипи, по другите градове? Те къде са ги открили?

"--* Не сме се свързвали с тях, господарю – каза Джинимир.

"--* Идиоти! Какво чакате, тогава?! Да дойда аз да ви свърша работата? Айде, чуйте се с тях и докладвайте!

След няколко минути Новото получи информацията. Всички нападатели бяха в същото положение – не знаеха къде се намират яйцата. Всички седяха и чакаха.

"--* Въобще някой търсил ли ги е? – попита Новото. – Или просто сте видели, че ги няма и оп, не знаете къде са\ldots

"--* Ами, търсихме ги, господарю – започна Джинимир. -- Под път и над път, както казват, ги търсихме\ldots

"--* Не можем ли да установим контрол над тях и те сами да се покажат? – отвърна Новото.

"--* Не, имат някаква странна защита. Единствения начин да ги спрем е да ги унищожим физически – каза Джинимир.

"--* Лошо\ldots Ами, ще търсим пак. Ама тоя път почваме от последния етаж. Айде всички!

Групата атакуващи се запъти по стълбите нагоре. Сред пъшкания, вопли и стенания, успяха да изкачат двата етажа. Новото изби вратата с крак и отиде към средата на голямата зала.

"--* Яйца! Командва ви Новото! Покажете се и ще процедираме мирно – естествено това той не го извика, а го предаде по всички пи-честоти. В отговор усети само едно мяукане на котка, която очевидно бе почнала да развива телепатични способности. – Само една котка отговори\ldots

Едно от човечетата почти се засмя. Новото видя това и го скастри мисловно. Онова се опита да застане сериозно, но не му се отдаде особено. Новото се отказа.

"--* Добре, тръгваме по ъглите и нишите. Като видите врата я отваряте и разхвърляте всичко, което видите вътре.

Всички се разшетаха. След един час се събраха отново в залата.

"--* И кво? – възмути се Новото. – Не можем да намерим едни шибани яйца\ldots Превзехме цяла планета и единственото, което ни спира са някакви си елипсоиди, пълни с качествени протеини и мазнини\ldots Не знам кво да правим, просто. Страшно много се ядосвам!

При последното си изречение той тропна с крак. Една от плочките под него се разцепи и хлътна надолу. Новото очакваше тя да пропадне напълно, но за негово съжаление тя се закрепи. Все пак, трябваше да бъде неговото – той я удари с юмрук. Ръката му се обля в някаква гадна желирана течност. Той я изтръска и избърса в хубавата си тога. Погледна пак към дупката. Там имаше три размазани яйца.

"--* Ето къде били иззедниците! – каза Новото. – Трошете всяка една плочка, докато не остане живо яйце! Предадете това на другите!

Трошенето на плочки почна. Новото отстъпи в страни и зачака с доволна усмивка на лице. След петнайсет минути Джинимир дойде.

"--* Всичко е изчистено, господарю!

Новото се заслуша. Не успя да долови пи-лъчения.

"--* Браво! Как е при другите екипи?

"--* И те са почти готови. Господарю -- драматично започна Джинимир, – мисля че можем да обявим тази планета за наша!

Една сълзичка потече по бузата на Джинимир. Новото сдържа чувствата си и само кимна.

"--* Все още имаме да правим много, моето момче. Скоро ще дойде моят цикъл. Трябва да намерим къде да ме заключите.

"--* В кораба ви има специално пригодена зала за това.

"--* Не, искам на тази планета. Искам да изпитам разрушението върху ѝ, за да мога после да я построя, както аз желая.

"--* Но господарю\ldots 

"--* Казах.

И той каза. След три дни го наместиха в някакъв бункер, който бил използван много, много отдавна. Надали щеше да издържи, но поне бе под земята – Новото нямаше къде да ходи. Пък и подсилиха стените на бункера с техните метали. Получи се доста сигурно.

\chapter*{Епилог}

Та, така\ldots Новото освободи тази изтерзана планета и започна да властва над нейните същества. Честно казано, те нямаха много време да мислят върху факта на предателство. Те не можеха и да се чувстват предадени, щото не знаеха какво е свобода. Един лидер се сменяше с друг. Слабият яйчен контрол се сменяше със завършения и силен пи-механизъм на Новото.

Хората пак си продължиха по старому, само че нито имаше свободни, нито неконтролирани. Имаше само контролирани и упражняващи контрол.

Новото изкара поредния си изблик на ярост в подсиления бункер. После отиде да си почива на родната планета и остави земята под властта на Джинимир. Следващият план на императора бе една планета от другата страна на галактиката. Но, като всеки добър владетел, Новото знаеше, че армията има нужда от почивка. То щеше да си почине, да изчака следващия цикъл и чак тогава да продължи завоевателната си кампания. Една доста велика империя се разрастваше. Непоклатима монархия с подчинени поданици.

Единственото, което притесняваше Новото бе да не му писне. То бе почти безсмъртно и всяка сутрин ставаше, надявайки се да не се отегчи от този безкраен живот.

\begin{flushright}
30 дек 2014 г.

Плевен
\end{flushright}


\end{document}
